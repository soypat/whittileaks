\documentclass{article}
\usepackage[spanish, mexico]{babel}
\usepackage[utf8]{inputenc}
\usepackage{booktabs,makecell}
\usepackage{xcolor}
\definecolor{darkblue}{rgb}{.01 , .05, .5}

\usepackage[urlcolor=blue,colorlinks=true,linkcolor=darkblue,
citecolor=red]{hyperref}
% Hyperrefs before geometry! Always!
\usepackage[a4paper,left=3cm,right=3cm,top=2cm,bottom=2cm]{geometry}

\newcommand{\ul}[1]{\underline{#1}}

%% Header
\usepackage{fancyhdr}
\setlength{\headheight}{15.2pt}
\pagestyle{fancy}
\lhead{ \href{http://www.whittileaks.com}{\textsf{whittileaks.com}} }
\rhead{Pato --- Cana}
\cfoot{\thepage}

%End Header


\begin{document}
{
\centering
{\Huge \bf \sc Derecho al Infierno\par}
\vspace{.1cm}
{\large Exámenes resueltos \par}
}
\pagestyle{plain}
\tableofcontents

\clearpage
\section{Final Derecho 2015}
\pagestyle{fancy}
\subsection{Explique cuáles son las consecuencias de la presentación en concurso y de la quiebra.}
\label{sec:ConsecuenciasConcursoQuiebra}

\textit{Efectos de concurso}
\begin{enumerate}
\item Se conserva la admin. del patrimonio, bajo supervisión
\item No puede hacer actos a título gratuito
\item Debe ser autorizado por juez a realizar distintos actos
\item Suspensión de intereses
\item No viajar al exterior
\end{enumerate}
\textit{Efectos de quiebra}
\begin{enumerate}
\item Todos los acreedores quedan sometidos al proceso
\item Todos verifican su crédito
\item Hipotecarios reclaman el pago mediante la venta de la cosa hipotecada en cualquier momento
\item Se consideran vencidas todas las obligaciones que tenían plazo
\item Se suspende el curso de intereses de todo tipo
\item El juzgado atrae todas las acciones judiciales iniciadas contra el fallido
\item El deudor es desapoderado de los bienes y de su administración
\item El síndico cobra las obligaciones a favor que tenía el deudor, dando recibos.
\item Se venden los bienes perecederos
\item El deudor queda inhabilitado para ejercer comercio, ser administrador, gerente, síndico…
\end{enumerate}

\subsection{Explique qué actos requieren autorización del Tribunal de Defensa de la Competencia y cuál es la posible resolución del trámite de autorización.}
La toma del control de una o varias empresas, siempre que el volumen de operaciones supere los $200$ Mpe:
\begin{enumerate}
\item Fusión entre empresas
\item Transferencia de fondos de comercio
\item Adquisición de propiedad o acciones que directa o indirectamente den control o influencia sustancial sobre la persona que emita tales derechos.
\item Cualquier otro acto que transfiera los activos de una empresa o le otorguen a una persona influencia determinante en las decisiones.
\end{enumerate}

\subsection{Explique qué modalidades de contratación laboral existen y  explíquelas a la luz de la posible utilización.}

\begin{itemize}
\item Plazo Fijo
\item Contratación eventual
\item Contrato de temporada
\item Trabajo por equipo
\item Contrato a tiempo parcial
\end{itemize}
Modalidades:
\textbf{Plazo fijo}
\begin{itemize}
	\item Plazo fijo tiene una duraci\'on máxima de cinco años. 
	\item No puede utilizarse para reemplazar a un empleado permanente
	\item Debe ser realizado por escrito y tener establecido el plazo
	\item Debe preavisarse con 30 dias de anticipación
	\item Abona una ndemnización al 50\% de que si fuera por despido del contrato por tiempo indeterminado si la antigüedad supera el año
\end{itemize}

\textbf{Contratación eventual}
\begin{itemize}
	\item Se realiza cuando no se determina plazo
	\item Circunstancias extraordinarias como:
	\begin{itemize}
		\item Determinada obra
		\item Aumento circunstancial del trabajo
	    \item cubrir ausencia temporaria personal
	\end{itemize}
\item No paga preaviso ni integracion de mes de despido
\item No da derecho indemnizatorio
\item No hay periodo de prueba
\end{itemize}

\textbf{Contrato de temporada}
\begin{itemize}
	\item Celebrado por un empleador y un grupo representado por intermedio de un delegado
	\item El intermediario pacta condiciones y el precio del grupo
	\item El intermediario puede cambiar a los integrantes o solicitarlo 
	\item El personal extra contratado tiene una vinculación aparte
	\item No es igual al trabajo por equipos
\end{itemize}

\textbf{Contrato a tiempo parcial}
\begin{itemize}
	\item Horas del día, semana o mes inferiores a los dos tercios de la jornada habitual
	\item Rige periodo de prueba
	\item No existen horas extra. Siempre se pagan como hora simple
	\item Aportes, contribuciones, indemnizaciones proporcionales al salario, con excepción de la obra social, que deberá pagarse sobre el total del sueldo de tiempo completo.
\end{itemize}



%\item Se devenga aunque no se preste servicio; con la \textit{mera puesta a disposicón de su fuerza de trabajo} se hace acreedor al trabajo de su salario aunque no se le encargue y realize, efectivamente, nada.
%\item El 80\% del pago debe ser en dinero efectivo y el resto puede ser imputado a pagos en especie
%\item El trabajo se presume remunerado
%\item SMVyM ...
\subsection{Explique qué es y qué integra un fondo de Comercio. Explique cómo puede transmitirse.} \label{sec:QueEsUnFondoDeComercioYTransmision}
Fondo de comercio(ley 11.867): Conjunto de fuerzas productivas, derechos y cosas que, tanto interior como exteriormente, se presentan como un organismo, con perfecta unidad para la obtención de beneficios comerciales e industriales. 
2do Articulo de Fondos de Comercio: Toda transmisión del fondo de comercio se tiene que hacer entre terceros, previo anuncio en el boletín oficial de la capital federal, o provincia respectiva. Anuncio debería especificar todos los datos del negocio.

\subsection{Explique: ¿qué es una ‘obligación’? ¿Qué es el patrimonio? ¿Cómo se integra?}
\textbf{Obligación}: Un vínculo jurídico en el cual el acreedor tiene derecho a exigir del deudor una prestación. Puede ser de la clase de \textit{dar}, \textit{hacer} o \textit{no hacer}. 
\textbf{Patrimonio}: Un conjunto de bienes
Los patrimonios de personas humanas son bienes susceptibles de valor económico.
Los patrimonios de personas jurídicas no son de los miembros, pues los miembros se consideran como personas enteramente distintas. Los socios (miembros) poseen ciertos derechos respecto de la sociedad pero no son dueños en forma directa de su patrimonio.


\section{Final Derecho 14 Dic. 2018}
\subsection{Formas de contrato laboral}
N/A
\subsection{Defina los tipos de personas que hay en la argentina y como es la capacidad y régimen para cada una.}
Existen dos tipos de personas: \textit{humanas} y personas \textit{jurídicas}. La persona \textit{humana} comienza su existencia desde la concepción. Si nace sin vida es como si nunca hubiese existido. Las personas \textit{jurídicas} pueden ser del carácter \underline{público}: el Estado nacional, las provincias etc o de carácter \underline{privado}: Sociedades, asociaciones civiles, iglesias, etc

\subsection{Explique los efectos de los concursos preventivos y compararlos con los de quiebra. Como puede llegar una persona a presentarse en concurso preventivo y como a una quiebra}
Efectos de concurso y quiebra en \ref{sec:ConsecuenciasConcursoQuiebra}. Una persona se presenta a un concurso cuando se encuentra en \textit{estado de cesación de pagos} de una obligación.  Consecuencias: [\ref{sec:ConsecuenciasConcursoQuiebra}]
\subsection{Explique todo lo referente a la Sociedad Anónima vista en clase}
(Art 1 de 19.550) Sociedades son compuestas por una o más personas que realicen aportes para aplicarlos a la producción o intercambio de bienes o servicios.
\begin{table}[htb!]
{\footnotesize
\begin{tabular}{@{}lccc@{}}
\toprule
\textbf{}                & \textbf{S.H.}                                         & \textbf{S.R.L}                                                                                                    & \textbf{S.A}                                                                           \\ \midrule
\textbf{Causa}           &\makecell{ Constitución informal.\\ No se registra\\ (puede en AFIP)} & \makecell{Constitución formal a través de\\ un contrato social. Por \\ instrumento publico o privado \\con firma  certificada (IGI)} & \makecell{ Constitución formal de un\\ contrato social por instrumento\\ publico (cédula verde) (IGI) }\\ \hline
\textbf{Socios}& 2 o más sin Límites& 2 o más hasta 50& \makecell{1 o más sin limites.Por lo \\ menos 1 persona humana. }\\ \hline
\textbf{Respnbld.} & Ilimitada. Solidaria. Subsidaria & Limitada a los cuota parte& Limitada a las acciones \\\hline
\textbf{Gobierno}        & Acuerdo de socios& Gerente & Directorio (Presidente) \\\hline
\textbf{Registros}       & Libro de compra\textbackslash{}ventas & \makecell{ Libro de actas y \\ diario inventario\textbackslash{}balance} & \makecell{Libro diario, registro de \\ accionistas inventario y balance\\ acta de asambleas }  \\ \bottomrule
\end{tabular}
}
\end{table}
\subsection{Régimen de marcas. Que derechos otorga? Que se puede registrar como marca?}
El régimen de marcas protege la inversión que una persona hace para que sea conocida su marca, pudiendo ésta oponerse al uso de ésta por terceros no autorizados, o al registro de marcas similares. Se puede registrar como marcas: Dibujos, emblemas, monogramas, grabados, estampados, sellos, imágenes, bandas, combinaciones de colores aplicados en lugares determinados de los productos o de los envases, los envases, las combinaciones de letras y números, las letras y números por su dibujo especial, combinación de dos o más colores, relieves, 

\section{Final 2019 Diciembre 1ra fecha}
\subsection{Explique las cuatro componentes de la funcion resarcitoria. Comente cada una.}
Resuelto en \ref{ssec:funcresarcitoria}.

\subsection{Que es un contrato de seguros, poliza, cuantas partes intervinientes? Cuando hay que comunicar el hecho, cuales son las condiciones? amplie.}


\subsection{Que es el abuso de posic\'on dominante? Explique como se determina.}

La ley de defensa de la competencia seg\'un Dr. Farina ``tiene por finalidad la protección de la libre concurrencia (y evitar la actividad monopólica u oligopólica)"{}. El art\'iculo 1 de la ley 22.262:

\begin{itemize}
	\item Se prohíbe y sanciona en la actividad de produccion e intercambio de bienes y servicios:
	\begin{itemize}
		\item[1)] Los actos o conductas que limiten, restrinjan o distorsionen la competencia
		\item[2)] Los actos o conductas que constituyan abuso de una \textit{posición dominante} en un mercado, de modo que pueda resultar perjuicio para el interés económico general
	\end{itemize}
\end{itemize}

El art\'iculo 4 expresa cuando existe una posición dominante:

\begin{itemize}
	\item Cuando para un determinado producto o servicio existe una sola empresa oferente o demandante dentro del mercado nacional
	\item Cuando dos o m\'as empresas gozan de una posición dominante seg\'un el item anterior pero no existe competencia efectiva entre ellas
\end{itemize} 

Quien se encuentra en una posición dominante deberá probar que no efectuó un abuso de \'esta en caso de conflicto. (Responsabilidad. Parte especial)

\subsection{Consecuencias de la extinci\'on de contrato laboral.}



\subsection{Que introdujo el articulo 14bis?}
Resuelto en \ref{ssec:arti14bis}.

\section{Final 2019 Diciembre Segunda fecha}
\subsection{Acerca la relaciones laborales. Cuales son los requerimientos?}

\subsection{???}

\subsection{Acerca la ley de defensa al consumidor escriba lo que sabe de garantías. (documentos, validez, plazos)}

\subsection{Que es un patente? Que requerimientos existen para la patente de un invento?}

\subsection{Defina el derecho. Que diferencias tiene con la moral?}

\section{Parcial 2015 Com. A}
\subsection{Defina el Derecho. Escriba todo lo que pueda sobre las normas jurídicas}
\label{sec:DefDerechoYNormaJuridica}
El \textbf{derecho} es un conjunto de normas que regulan el comportamiento humano, impuestas por el Estado y orientadas hacia el ideal de justicia. 

La \textbf{norma jurídica} debe describir y prescribir una conducta, y establecer las consecuencias de su incumplimiento. Se ordenan según su jerarquía (CN+Tratados, Leyes, Decretos+Resoluciones, Contratos). Deben tener vigencia, cumplida de forma efectiva.
La norma (jurídica) debe describir y prescribir una conducta, y a continuación establecer las consecuencias de que esa conducta no sea cumplida (sanción) [Kelsen - iuspositivistas]. 
La norma (jurídica) debe ser una prescripción de la razón en vista del bien común y promulgada por el que tiene al cuidado la comunidad [Santo Tomás - iusnaturalistas]. 

\subsection{Que es un cheque? Cuales son los tipos de cheques?}
El cheque es una orden de pago mediante el cual una persona que tiene cuenta corriente en un banco ordena a éste pagarle una cierta cantidad de dinero a un tercero. Los tipos son (ver también \ref{ssec:chequesprimeratipificacion} 
\begin{enumerate}
\item Cheque comun: orden de pago a la vista. Pagadera de hoy a +30days
\item Cheque de pago diferido: Pagadera desde la fecha que figura hasta +30days
\item Transmisión. Los endosantes asumen responsabilidad sobre el efectivo cobro del cheque en línea descendente. 
\end{enumerate}

También existen otras formas de clasificar los cheques...
\begin{itemize}
	\item Cheque cruzado. Solo se puede depositar en cuenta.
	\item Cheque al portador. No tiene especificado beneficiario. Puede ser cobrado por cualquiera que lo tenga en su poder.
	\item Cheque no a la orden. Solo puede cobrar el beneficiario especificado.
	\item  Cheque certificado. El banco certifica que la cuenta a pagar tiene fondos en el plazo especificado.
	\item Cheque de caja. Expedido por institucion de credito para ser pagado en sus propias sucursales.
	\item Cheque de viajero. Expedido por institución bancaria para ser pagado en alguna de sus sucursales dentro del pais o en el exterior.
	\item Cheque para acreditar en cuenta. Tiene que ser depositado en cuenta y se indica en que banco tiene que ser depositado.
	\item Cheque en blanco. Solo requiere la firma del librador. El resto de los campos los puede llenar el beneficiari+o.
\end{itemize}
\subsection{Que son y cuales son los efectos de los procedimientos concursales?}
El concurso preventivo es el proceso por el cual el deudor logra un acuerdo con sus acreedores, por 

\subsection{En patentes, que es un invento?}
Se considera un invento a toda creación humana que permita transformar materia o energía para su aprovechamiento por el hombre.

\subsection{Que es una obligación? Que tipos hay? Cuales son los elementos esenciales de una obligación?}
\label{sec:QueEsObligacionTiposDeOblElementosDeOblYDeudaGarantia}
art. 724: La obligación es una relación jurídica en virtud de la cual el acreedor tiene el derecho a exigir del deudor una prestación destinada a satisfacer un interés licito y, ante el incumplimiento, a obtener forzadamente la satisfacción de dicho interés.
	Clases de Obligaciones: \textit{Dar}, \textit{hacer} o \textit{no hacer}. 
	Las obligaciones tienen una causa que les da origen de conformidad con el ordenamiento jurídico. Los tres elementos esenciales de una obligación son:
\begin{enumerate}
\item Los sujetos con quien contrata y sus capacidades.
\item Que se contrata. (La prestación objeto de las obligaciones)
\item Que hecho idóneo lo genera (Causa.)
\end{enumerate}
Sin alguno de estos, la obligación no existiría.
\textit{Garantía}: El patrimonio del sujeto deudor. \textit{Deuda}: Ese deber de cumplir que impulsa a pagar

\section{Parcial 2015 Com. B}
\subsection{Escribir todo lo que sepa sobre seguros.}
El seguro es un contrato que establece la obligación de una compañía aseguradora de dar al asegurado una cantidad determinada de dinero si sufre algún daño en su persona o en sus cosas. 
\subsection{Derechos intelectuales. Los plazos de cada uno}
\label{sec:DerechosAutoryPlazos}
Los derechos intelectuales son aquellos derechos de dominio que tiene la persona sobre su obra. Éstos se distinguen entre patrimoniales y extrapatrimoniales. Los primeros se pueden usar para realizar contratos y obtener lucro, mientras que los segundos tratan de los derechos personales: paternidad de la obra, integridad y la divulgación de ésta. 
Los plazos son distintos según se trate de obras, marcas y patentes. 

\begin{itemize}
\item En  obras escritas o artísticas los derechos de autor duran hasta 70 años después de la muerte del autor, momento a partir del cual pasan a dominio público.
\item  Las obras cinematográficas el plazo es hasta 50 años después de la muerte del último colaborador. 
\item Las fotografías duran hasta 20 años después de su primera publicación
\item Las marcas son renovables ilimitadamente cada 10 años, sujeto a la condiciones que tienen que ver con el aprovechamiento de la marca.
\item Las patentes permiten explotar la invención de forma exclusiva 20 años. 
\item Las patentes de adición duran hasta que vence la patente original.
\item Los modelos de utilidad duran 10 años improrrogables.
\item Los diseños industriales duran 5 años renovables por 5 años más.
\end{itemize}

\subsection{Explicar cuales son las fuentes del derecho}
\label{sec:FuentesDeDerechoExplicadas}
\textbf{FUENTE DE DERECHO:} Aquello que origina la norma jurídica.

\textit{La Ley}: Conjunto de normas que incluyen leyes, decretos, resoluciones etcetera. Creadas por el órgano del estado, el cual les otorga validez.

\textit{La Costumbre}: El ser humano se mueve en el ambiente respetando ciertas construcciones sociales. Cuando estas son resaltadas por la mayoría de la sociedad se empiezan a considerar obligatorios, se transforman, por su repetición, en costumbres. Dicho de otra forma:  \textbf{Derecho consuetudinario} son normas jurídicas que no están escritas pero se cumplen porque en el tiempo se han hecho costumbre cumplirlas. Tiene fuerza y se recurre a él cuando \textit{no existe ley} (o norma jurídica escrita) aplicable a un hecho.

\textit{La Jurisprudencia:} Se conocen con el término jurisprudencia las decisiones del órgano judicial del Estado. Los jueces tiene la posibilidad de apoyarse en los fallos de cámaras anteriores. En argentina no es obligatoria seguir los fallos anteriores excepto en el caso de un fallo plenario.

\textit{La Doctrina}: Reunión de opiniones  de los autores expertos en derecho que se han volcado en diversos libros y artículos sobre materia jurídica. No tiene ninguna obligatoriedad.

\subsection{Cuales son las formas para extinguir obligaciones}
\begin{enumerate}
\item Por pago: el cumplimiento de la prestación objeto de la obligación.
\item Por la novación: el reemplazo de la obligación por otra
\item Compensación: Cuando las partes son acreedoras y deudoras al mismo tiempo.
\item Transacción: Negociación, concesión recíproca. 
\item Confusión: Cuando las dos partes se reúnen en una sola persona
\item Renuncia de derechos: El acreedor abdica de algún derecho subjetivo.
\item Remisión de deuda: El acreedor se considera pagado ficticiamente. 
\item Imposibilidad del pago: Por caso fortuito o fuerza mayor
\item Dación en pago: Acreedor acepta un prestación distinta a la original
\end{enumerate}


\subsection{Que es una persona? Que capacidades tienen las personas jurídicas?}

Una persona según el código civil puede ser \textit{humana} o \textit{jurídica} y estas disponen de derechos según su clasificación. Para personas jurídicas la \textit{capacidad de derecho} está limitada a solamente realizar actos que estén contemplados en su objeto social (especificado en su acta constitucional). 

\section{Parcial del 02/10:}
\subsection{¿Cómo estructuraría a las normas jurídicas? Fundamente.}
La norma jurídica debe describir y prescribir una conducta, y a continuación establecer las consecuencias de que esa conducta no sea cumplida (sanción).
	Por arriba de la \textbf{Constitución Nacional} (CN) de un país no hay normas jurídicas si no un hecho de fuerza que otorga legitimidad y monopolio de fuerza al estado. Por debajo de la CN están las \textbf{leyes} dictadas por el poder legislativo, luego los \textbf{decretos} dictados por el poder ejecutivo, luego las \textbf{resoluciones} y por debajo otras normativas originadas de otros órganos inferiores y por último las convenciones de los particulares.
\subsection{¿Qué derechos y garantías se incluyen en el artículo 14 bis de la CN?} \label{ssec:arti14bis}
Se incluyen los derechos de segunda generación, los relacionados al trabajo. 
En cuanto al trabajador se menciona:
\begin{itemize}
\item condiciones dignas y equitativas de labor
\item jornada limitada
\item descanso y vacaciones pagados
\item retribución justa
\item salario mínimo vital móvil
\item igual remuneración por igual tarea
\item participación en las ganancias de las empresas, con control de la producción y colaboración en la dirección;
\item protección contra el despido arbitrario
\item estabilidad del empleado público
\item organización sindical libre y democrática reconocida por la simple inscripción en un registro especial.
\end{itemize}
Queda garantizado a los gremios:
\begin{itemize}
\item Concertar convenios colectivos de trabajo
\item recurrir a la conciliación y al arbitraje
\item el derecho de huelga. 
\item Los representantes gremiales gozarán de las garantías necesarias para el cumplimiento de su gestión sindical y las relacionadas con la estabilidad de su empleo.
\end{itemize}

El Estado otorgará los beneficios de la seguridad social, que tendrá carácter de integral e irrenunciable. En especial, la ley establecerá: 

\begin{itemize}
\item el seguro social obligatorio 
\item jubilaciones y pensiones móviles;
\item la protección integral de la familia; 
\item la defensa del bien de familia; 
\item la compensación económica familiar
\item el acceso a una vivienda digna.
\end{itemize}


\subsection{¿Qué es una marca? ¿Qué tipos hay? ¿Cómo se registran?}
\textit{Marca}: Signo utilizado para distinguir un servicio o un producto de otros similares o de igual naturaleza. Se puede registrar palabras, dibujos, emblemas, grabados, hologramas etc. por periodos de diez años, con posibilidad de renovación. 
	No se pueden registrar nombres, palabras y signos que constituyan la designación necesaria o habitual del producto o servicio a distinguir.

\subsection{¿Qué es un fondo de comercio?}
Ver \ref{sec:QueEsUnFondoDeComercioYTransmision}.
\subsection{Explique la extinción por pago y por renuncia.}
Las obligaciones tienen una duración. Tienen una fuente y un final. 

\textbf{Extinción por pago}: Cumplimiento de la prestación del deudor, el cual era objeto de la obligación al acreedor.

\textbf{Por renuncia:} El acreedor puede renunciar sus derechos a la prestación.


\section{Parcial 2016 Comision Jueves}
\subsection{Que es el derecho? Cuales son los derechos y garantías individuales?}
Definición Derecho en \ref{sec:DefDerechoYNormaJuridica}. En la primera parte del artículo 14 encontramos derechos que no son absolutos [\ref{sec:Art14}] (las leyes reglamentan su ejercicio).
\subsection{Que es una obligación? Cuales son las componentes. Explique deuda y garantía.}
Obligaciones y deudas,garantías en \ref{sec:QueEsObligacionTiposDeOblElementosDeOblYDeudaGarantia}.

\subsection{Que es un cheque? Describa los distintos tipos.} \label{ssec:chequesprimeratipificacion}
El cheque es una orden de pago mediante la cual una persona que tiene cuenta corriente en un banco ordena a éste pagarle una cierta cantidad de dinero a un tercero. Los cheques se pueden transmitir de diferentes modos dependiendo del tipo de cheque.
Existen dos tipos de cheques a grandes rasgos (No es la única clasificación):
\begin{itemize}
\item \textbf{Cheque común:} Orden de pago a la vista hasta 30 días después de su emisión.
\item \textbf{Cheque de pago diferido}: Es una orden de pago librada para una fecha en el futuro y se realiza en un plazo de 30 días. Entre la fecha de libramiento y de pago puede mediar un periodo mayor a 365 días.
\end{itemize}



\subsection{Que requerimientos hay para que algo sea considerado como un invento? Como son los plazos para las patentes?}
\label{sec:ReqParaConsiderarUnInventoyPlazosPatentes}

(Art. 4 Ley 24.481) 
\begin{itemize}
    \item Debe constituir una novedad absoluta (Su conocimiento no debe encontrarse en el ámbito técnico[\textit{estado de la técnica}])
    \item Uso práctico con aplicación industrial
    \item debe poder se patentable por no violar el universo normativo de la ley 24.481 
\end{itemize}

\subsection{Cuales son las fuentes del derecho. Explique todo lo que sepa de las leyes.}
Fuentes del derecho: [\ref{sec:FuentesDeDerechoExplicadas}]. Las leyes pueden ser clasificadas en \textit{imperativas} o \textit{supletorias}. 
\label{sec:LeyesTodo}
Las leyes imperativas son consideradas imprescindibles en su aplicación a sujetos jurídicos. Al hacer un contrato estas leyes no pueden ser dejadas a un lado.

Las leyes supletorias no poseen obligatoriedad en el caso mencionado anteriormente y vienen a subsanar la falta de regulación de derechos y obligaciones.

Hay un tipo de ley especial llamada \textit{código}. Se trata de un conjunto de normas abocadas a reglar una rama especifica del derecho en forma sistemática. Existen códigos de \textit{fondo} y \textit{forma}. Las leyes de fondo son reservadas para la Constitución Nacional para ser dictadas por el Congreso de la Nación y tener aplicación en todo el país. En cambio las leyes de forma se refieren al procedimiento para llevar a cabo la aplicación de las leyes de fondo. Siendo que las idiosincrasias, cantidad de habitantes y extensión del territorio de cada provincia son distintos. Esto significa que cada provincia dicta su código de procedimientos.


\section{Parcial Escrito a mano}
\subsection{Que es una obligación? Cuales son las implicaciones de obligación de medio y resultados}

\subsection{Pirámide jurídica. Explicarla y graficarla. Hablar de la vigencia y validez de las normas jurídicas}
La \textit{validez} es otorgada por el cuerpo que la cree y su cumplimiento con normas de rango superior, como por ejemplo, la Constitución Nacional.
La \textit{vigencia} se refiere a la la promulgación de la norma por cuerpos jurídicos. Una norma puede existir pero no ser controlada, esto le quita la obligatoriedad, lo cual es esencial a la norma.
\subsection{Art. 14} \label{sec:Art14}
Todo habitante de la argentina tiene derecho a\textbf{ ejercer su profesión} (lícita);\textbf{peticionar} a las autoridades; \textbf{entrar}, \textbf{permanecer} y \textbf{transitar} el territorio; \textbf{publicar ideas} sin ser censurado; disponer y usar su \textbf{propiedad}; \textbf{profesar} su culto y \textbf{aprender} libremente.
\subsection{Que es lo que conoce el juez para entrar en concurso de quiebra? Cuales son las formas de llegar a la quiebra?}
\label{sec:QueConoceJuezEnConcursoQuiebra}
En toda quiebra el síndico debe informar al juez lo siguiente

\begin{itemize}
    \item La posibilidad de mantener la explotación sin contraer nuevos pasivos, salvo los mínimos para mantener la empresa
    \item La ventaja que resultaría para los acreedores de la enajenación de la empresa en marcha
    \item La ventaja que pudiere resultar para terceros del mantenimiento de la actividad
    \item El plan de explotación acompañado de un presupuesto de recursos, debidamente fundado
    \item Los contratos en curso de ejecución que deben mantenerse
    \item En su caso, las reorganizaciones o modificaciones que deben realizarse en la empresa para hacer económicamente viable su explotación
    \item Los colaboradores que necesitará para la administración de la explotación
    \item Explicar el modo en que se pretende cancelar el pasivo preexistente
\end{itemize}
El deudor se encuentra en un \textit{estado de cesación de pagos} de una obligación. El juez conoce
\begin{itemize}
\item Las causas concretas de la situación patrimonial del deudor y la razón de la cesación de pagos
\item Todo acerca los activos y pasivos de la sociedad en cuestión, e incluso el patrimonio
\item Balances u otros contables
\item Todos los acreedores del deudor en cuestión y los detalles de todas las deudas
\item Libros de comercio que lleve el deudor
\item Si hubo concursos anteriores y si está inhibido a concurso
\item Datos relevantes acerca los empleados y si hay deuda laboral
\end{itemize}


\subsection{Marcas. Para que sirve y enunciar 5 prohibiciones del recurso marcario.}
\label{sec:ParaQueSirvenMarcasYProhibiciones}
Utilizadas para distinguir un servicio o un producto de otros de similar o igual naturaleza.
(Art. 2):
\begin{itemize}
\item No son registrables nombres, palabras y signos que constituyan la designación necesaria o habitual del producto o servicio habitual a distinguir
\item Frases, signos que hayan pasado al uso general antes de su solicitud de registro
\item La forma que se dé a los  productos
\item El color natural o intrínseco de los productos o un solo color aplicado sobre los mismos.
\end{itemize}
(Art. 3):  No incluye todas
\begin{itemize}
\item Marca idéntica a una registrada o solicitada con anterioridad para distinguir los mismos productos/servicios
\item Las marcas similares a otras ya registradas o solicitadas para distinguir los mismos productos/servicios
\item Las denominaciones (nombre de país, región, lugar o área geográfica) de origen nacionales o extranjeras para designar un producto originario de ellos.
\item Palabras/signos contrarios a la moral y a las buenas costumbres
\item El nombre o seudónimo o retrato de una persona (sin consentimiento de el o el de sus herederos hasta cuarto grado inclusive). que hayan pasado al uso general antes de su solicitud de registro
\end{itemize}


\section{Parcial 18 de abril 2018}
\subsection{Enumere cuatro fuentes del derecho. }
Ver \ref{sec:FuentesDeDerechoExplicadas}.

\subsection{Explique cómo está integrado el congreso nacional y cuales son sus efectos.}
El Congreso de la Nación Argentina es el órgano que ejerce el poder legislativo federal de la República Argentina, compuesto por el senado y la cámara de diputados.  Los integrantes incluyen 3 senadores por cada distrito electoral (23 provincias y CABA) dando un total de 72 senadores y una \textit{representación proporcional por listas} de diputados. 

\subsection{Explique Como se puede llegar a una quiebra y cuales son sus efectos}
Consecuencias: \ref{sec:ConsecuenciasConcursoQuiebra}.
\subsection{Explique que defiende la ley de derecho de autor? Que derechos otorga? Cuanto dura}
	Los \textbf{derechos de propiedad intelectual} defienden la posibilidad de hacer contratos o obtener lucro por medio de obras cuyas creación es atribuida al autor.
    
	El \textbf{derecho de autor} defiende la obra, abarcando la expresión de ideas, procedimientos, métodos de operación, conceptos matemáticos; pero no a las ideas, métodos y conceptos en sí mismos.
	Películas western pueden ser similares entre sí pero todas cuentan con diferencias en expresión: diferentes estilos, palabras, melodías etcétera.
	Duración: \ref{sec:DerechosAutoryPlazos}

\subsection{Explique los presupuestos de la responsabilidad civil con relaciona la función resarcitoria.} \label{ssec:funcresarcitoria}
\label{sec:ResponsabilidadCivilyFuncionResarcitoria}
La \textbf{Responsabilidad Civil} trata el tema de ``¿quien debe \textit{responder} ante las obligaciones?'' y tiene tres funciones: \textit{preventiva}, \textit{punitiva} y \textit{resarcitoria}. 

La función \textit{resarcitoria} consiste en la reparación del daño generado por la violación de ``no dañar'' o el incumplimiento de una obligación. Consiste en los siguientes presupuestos:

\begin{enumerate}
    \item \textbf{Antijuridicidad}: La acción u omisión es contraria a lo previsto en el ordenamiento jurídico. Sin embargo, existen excepciones...
    \item \textbf{Factor de atribución}: Implica dos tipos de imputación al provocador de daño: \textit{Objetiva} y \textit{Subjetiva}.
    \begin{enumerate}
        \item Objetiva: La única forma de eximirse de la responsabilidad es cortando el nexo casual (si, sin importar la intención, el acreedor no cumple con lo prometido)
        \item Subjetiva: Forma de eximirse mediante el \textit{dolo} y la \textit{culpa}
    \end{enumerate}
    \item \textbf{Previsibilidad Contractual}: Cuanto más grande o crítico sea el deber, mayor es la diligencia exigible al agente y la valoración de la previsibilidad de las consecuencias
    \item \textbf{Relación de Casualidad}: Busca analizar la causa efecto del daño
\end{enumerate}

\section{Parcial Whatsapp Agucho}
\subsection{Nombrar Fuentes del derecho y explicarlas}
\ref{sec:FuentesDeDerechoExplicadas}
\subsection{Que tener en cuenta a la hora de registrar una marca para antes y para el futuro}

\subsection{Que dice la constitución sobre como se sancionan las leyes}
ds
\subsection{Definir persona humana y jurídica y decir las diferencias en cuanto a responsabilidades entre ellas}
sd
\subsection{Que provoca el concurso respecto los bienes del dueño de la empresa tanto como los bienes personales?}
sd


\section{18 de octubre de 2016 - Post-Peronismo Day}
\subsection{Defina derecho y moral y  compárelos}
sd
\subsection{Explique función resarcitoria}

\subsection{Defina cheque y formas de emitir uno}
ds
\subsection{Explique cómo está dividido y enumere los derechos del artículo 14 bis}
ds
\subsection{Requisitos de una patente. Duración. Patente de adición y modelos de utilidad.}
dcd


\section{Parcial 27.05.2019 Com. Lunes}
\subsection{Explique y compare el Derecho y la moral }

\subsection{Compare los efectos de concurso y quiebra. Explique cómo se puede ingresar al concurso y y como a una quiebra}

\subsection{Escriba todo lo que sabe de S.R.L.}

\subsection{Explique el régimen de patentes}

\subsection{Explique las partes esenciales de una obligación y compárelas con las de un contrato}

\setcounter{section}{-1}
\section{Random}
\subsection{Que es una invención y que es el estado de la técnica}

\subsection{Relación entre moral y derecho.}

\subsection{Explicar la extinción por transacción y compensación.}

\subsection{Explicar los tipos de minas y su relación con el propietario superficial}

\subsection{Funciones del poder legislativo. ¿Dónde ubicaría en la pirámide jurídica lo mencionado en el inciso 22 de este artículo?}

\subsection{¿Qué requisitos debe cumplir un invento según el artículo 4 de la ley de patentes?}

\subsection{Explicar los derechos consagrados en el art. 17 de la CN.}

\subsection{Definir qué es una obligación, cuáles son sus fuentes y sus efectos. Elementos que existen en la obligación y explicarlos.}

\subsection{Nombrar los participantes de una contratación de un seguro. Ejemplificar para los casos de un seguro de saldo de una tarjeta de crédito y para una responsabilidad civil con un tercero involucrado.}

\subsection{Explicar la diferencia y los plazos de protección legal para una patente de adición y un modelo de utilidad.}

\subsection{Enumerar algunas atribuciones de alguno de los poderes.}

\subsection{Art. 16}
La Nación Argentina no admite prerrogativas de sangre, ni de nacimiento: No hay en ella fueros personales ni títulos de nobleza. Todos sus habitantes son iguales ante la ley, y admisibles en los empleos sin otra condición que la idoneidad. La igualdad es la base del impuesto y de las cargas públicas.


% \section{examen}
% \subsection{Explique}

% \subsection{Explique}

% \subsection{Explique}

% \subsection{Explique}

% \subsection{Explique}




\end{document}
