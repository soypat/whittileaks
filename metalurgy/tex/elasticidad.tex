\section{Comportamiento elástico}

\subsection{Comportamiento elástico perfecto y anelasticidad}
Si un material tiene \textbf{comportamiento elástico perfecto} entonces la \textit{deformación depende exclusivamente de la tensión}. En términos matemáticos existe una relación biunívoca entre la tensión y la deformación \eqref{eq:elasticidad}

\begin{equation} \label{eq:elasticidad}
	\varepsilon = f(\sigma)
\end{equation}

Consecuencias:
\begin{itemize}
	\item No hay deformación remanente
	\item No hay dependencia de la deformación con el tiempo. Las deformaciones están en fase con las tensiones
	\item No hay energía neta absorbida por el material (la deformación si tiene una energía asociada)
\end{itemize}


Los materiales cuya deformación depende del tiempo tienen \textbf{comportamiento elástico imperfecto o anelástico} \eqref{eq:anelasticidad}. Durante el ciclo de carga/descarga el material absorbe cierta energía, lo que contribuye a su capacidad de amortiguamiento. En metales a baja temperatura es despreciable el amortiguamiento anelástico.

\begin{equation} \label{eq:anelasticidad}
	\varepsilon = f(\sigma,t)
\end{equation}

El régimen elástico en metales es invariablemente pequeña ($\varepsilon$ menor a 5\%) y por eso se puede aproximar mediante una recta que tiene la pendiente de la fuerza neta de la curva Condon-Morse (figura \ref{fig:curvas_condonmorse}). La pendiente de esta curva se denomina el \textbf{módulo de Young} o constante elástica del material.

\subsection{La elasticidad como propiedad}
Algunos puntos a remarcar de la elasticidad

\begin{description}
	\item[Material] Depende del material y composición química
	\item[Red cristalina] Hay diferentes módulos elásticos normales y transversales para una red cristalina dependiendo de la dirección (\textbf{anisotropía}). 
	\item[Defectos vs. elasticidad] Las constantes elásticas son propiedades insensibles a la estructura de defectos del material y como tales \textbf{no son modificables mediante procesos} como la deformación plástica o tratamientos térmicos
	\item[Depende de prop. termodinámicas] Depende de la temperatura y (en menor medida) la presión
	\item[Monocristales] Al tener un monocristal se tiene el caso de anisotropía (diferentes módulos elásticos dependientes de la dirección). Puede haber grandes diferencias en diferentes direcciones
	\item[Policristales y textura] La mayoría de los metales son policristales. Un \textbf{policristal no-texturado} tiene una gran cantidad de granos orientados al azar dando así un promedio único de las rigideces en todas las direcciones. Un \textbf{policristal texturado} la rigidez vuelve a ser una propiedad anisótropa pues hay una orientación preferencial de grano
\end{description}



\subsubsection{Rigidez intrínseca vs. estructural}

Hasta ahora se habló de la rigidez de un material y como se relaciona a las curvas de Condon-Morse. Estas propiedades son insensibles a los cambios de la estructura. Esta rigidez se denomina \textbf{rigidez intrínseca}.

La rigidez de una pieza o estructura depende de su geometría. Esta es la \textbf{rigidez estructural}. En general suele ser más efectivo cambiar la rigidez estructural (cambiar dimensiones/geometría de una pieza) a cambiar la rigidez intrínseca (material usado).

