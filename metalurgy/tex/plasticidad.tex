\section{Deformación plástica en metales}

La \textbf{deformación plástica} es aquella deformación remanente en el material una vez retiradas las cargas que produjeron la deformación. Cabe destacar que las \textbf{únicas} tensiones que producen deformación plástica son las de \textbf{corte}, es decir, en estado de tensión hidrostático no puede haber deformación plástica.

Es un fenómeno mecánicamente irreversible (a diferencia de la elasticidad) y en metales puede alcanzar grandes valores ($\varepsilon \approx  0.5\%$ hasta $90\%$)

\subsection{Consecuencias de la deformación plástica}

\begin{description}
	\item[Endurecimiento por deformación]  La tensión de fluencia aumenta 5 veces al deformar un acero inoxidable (301) al 60\%. Método económico para lograr altas resistencias
	\item[Aumento de tenacidad] La capacidad de absorber grandes deformaciones implica un aumento de la energía que puede absorber. Esto se traduce a resistencia a propagación de fisuras y resistencia ante impactos
	\item[Propiedades intrínsecas] La densidad prácticamente no cambia. El parámetro de red permanece constante 
\end{description}

\subsection{Mecanismo de deformación plástica}

Si se quiere deformar plásticamente a un monocristal se requiere deslizar un plano de átomos. La tensión teórica necesaria para lograr esto es \textbf{4 órdenes de magnitud mayor} a los resultados experimentales. La razón por esto es por la presencia de defectos en la red, particularmente, \textbf{dislocaciones}.

Al aplicar una carga que produzca tensiones de corte adecuadas, se rompen los enlaces de una hilera (sobre una linea de átomos), lo que requiere mucha menos tensión que la rotura de los enlaces sobre un plano entero. De esta forma la dislocación avanza un paso a la vez hasta llegar a una superficie libre (generando un escalón) o anclarse en un átomo sustitucional. Cada desplazamiento será idéntico a la longitud y dirección del vector Burgers.

Estas dislocaciones interactúan entre sí y defectos puntuales debido a la distorsión local de la red. La distorsión genera tensiones de compresión, tracción o corte, generando así atracción y repulsión entre defectos.

\subsection[Planos y direcciones de deslizamiento]{Planos y direcciones de deslizamiento -- Tensión crítica }



