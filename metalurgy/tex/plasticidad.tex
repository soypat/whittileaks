\section{Deformación plástica en metales}

La \textbf{deformación plástica} es aquella deformación remanente en el material una vez retiradas las cargas que produjeron la deformación. Cabe destacar que las \textbf{únicas} tensiones que producen deformación plástica son las de \textbf{corte}, es decir, en estado de tensión hidrostático no puede haber deformación plástica.

Es un fenómeno mecánicamente irreversible (a diferencia de la elasticidad) y en metales puede alcanzar grandes valores ($\varepsilon \approx  0.5\%$ hasta $90\%$)

\subsection{Consecuencias de la deformación plástica}

\begin{description}
	\item[Endurecimiento por deformación]  La tensión de fluencia aumenta 5 veces al deformar un acero inoxidable (301) al 60\%. Método económico para lograr altas resistencias
	\item[Aumento de tenacidad] La capacidad de absorber grandes deformaciones implica un aumento de la energía que puede absorber. Esto se traduce a resistencia a propagación de fisuras y resistencia ante impactos
	\item[Propiedades intrínsecas] La densidad prácticamente no cambia. El parámetro de red permanece constante 
\end{description}

\subsection{Mecanismo de deformación plástica}

Si se quiere deformar plásticamente a un monocristal se requiere deslizar un plano de átomos. La tensión teórica necesaria para lograr esto es \textbf{4 órdenes de magnitud mayor} a los resultados experimentales. La razón por esto es por la presencia de defectos en la red, particularmente, \textbf{dislocaciones}.

Al aplicar una carga que produzca tensiones de corte adecuadas, se rompen los enlaces de una hilera (sobre una linea de átomos), lo que requiere mucha menos tensión que la rotura de los enlaces sobre un plano entero. De esta forma la dislocación avanza un paso a la vez hasta llegar a una superficie libre (generando un escalón) o anclarse en un átomo sustitucional. Cada desplazamiento será idéntico a la longitud y dirección del vector Burgers.

Estas dislocaciones interactúan entre sí y defectos puntuales debido a la distorsión local de la red. La distorsión genera tensiones de compresión, tracción o corte, generando así atracción y repulsión entre defectos.

\subsection[Sistemas de deslizamiento]{Sistemas, planos y direcciones de deslizamiento -- Tensión crítica }


El deslizamiento o cizallamiento en un metal se produce sólo sobre ciertos planos cristalinos y en determinadas direcciones cristalinas contenidas en dichos planos. Cada familia de planos y direcciones posee una determinada distribución de átomos. La \textbf{mínima} tensión de corte necesaria para provocar el deslizamiento en determinado plano y dirección se llama {\it \bf tensión de corte crítica}. {\it Aquellos planos donde la tensión crítica sea mínima serán los preferenciales para que ocurra la deformación plástica.}

Tal vez le resulte poco intuitivo al lector, pero los planos de \textbf{tensión crítica mínima son los planos de alta densidad}. Esto es porque el recorrido de la dislocación es menor (vector de Burger es menor).

La combinación de un plano de alta densidad y una dirección de alta densidad se lo denomina \textbf{sistema de deslizamiento}. {\it En un policristal, los sistemas de deslizamiento que están alineados con la máxima tensión de corte son los que actúan y dan luz a la deformación plástica.}

\begin{description}
	\item[Redes FCC y HCP] Tienen planos y direcciones de máxima densidad denominados \textbf{compactos}. Estos forman los sistemas de deslizamiento de estas redes.
	\item[Red BCC] Contienen \textbf{direcciones compactas} pero \textbf{no planos compactos}, sin embargo hay varias familias de planos de alta densidad que contienen a la dirección compacta. Estos conforman los sistemas de deslizamiento de la red BCC.  
\end{description}


\subsubsection{Máxima tensión de corte en un ensayo de tracción} \label{sssec:max_corte_ensayotraccion}
En una probeta sometida a tracción se puede aplicar una transformación de ejes para obtener el corte $\tau$ en un nuevo sistema $(x_\theta, y_\theta)$ a un ángulo $\theta$. Partiendo de la suposición de un estado de tensiones planos se tiene:
\footnote{Se suele hablar de la ley de Schmid aquí también, la cual tiene la forma: $\tau = \sigma \cos \theta \cos \chi$}

\begin{equation} \label{eq:plane_stress_transform}
	\tau_{x_\theta y_\theta} = - \left( \frac{\sigma_{xx} - \sigma_{yy}}{2} \right) \sin 2\theta + \tau_{xy} \cos 2 \theta
\end{equation}

Para el estado de tensiones de un ensayo de tracción 

\begin{equation}
	\sigma_{xx} = \tau_{xy} = 0,\qquad \sigma_{yy} = \frac{F}{S_0} 
\end{equation}
donde $F$ es la fuerza de tracción y $S_0$ es la sección nominal de la probeta. La tensión de corte se maximiza para $\theta =  45^\circ$.

\subsubsection{Microplasticidad}

Siguiendo la lógica de la sección \ref{sssec:max_corte_ensayotraccion}, en un policristal sometido al ensayo de tracción la deformación plástica va comenzar con los granos que tienen su sistema de deslizamiento alineado con con el ángulo a $45^\circ$ de la tracción. El resto de los granos seguirán en el rango elástico pues la tensión de corte sobre sus sistemas de deslizamiento no llegó a la tensión crítica. Este fenómeno se denomina \textbf{microplasticidad}, pues el material sigue estando en su rango elástico (si se retiran las cargas, la pieza vuelve a su forma y dimensión original. Cuando la casi totalidad de los granos ha comenzado a deformar plásticamente, recién se puede detectar macroscópicamente la deformación plástica.

La microplasticidad juega un rol importante en la \textbf{fatiga}, un fenómeno caracterizado por la nucleación de fisuras a una tensión mucho menor que la de fluencia.

En el ensayo de tracción se define la tensión de fluencia en forma convencional (deformación remanente del 0,2\%).

\subsubsection{Deformación por maclado}

Además del cizallamiento entre planos de alta densidad, existe un mecanismo alternativo para la deformación plástica llamado \textbf{maclado}. Este mecanismo requiere \textbf{mucha energía} y la \textbf{deformación a la que puede conducir es muy limitada}, lo cual lo deja en un puesto de poca importancia con respecto al cizallamiento.





\begin{description}
	\item[BCC] Por debajo de una temperatura o a altas velocidades de deformación, el \textbf{maclado comienza a dominar en metales BCC}. 
	\item[FCC] Los metales FCC poseen varios sistemas de deslizamiento y por lo tanto no suelen maclar por deformación, pero si pueden maclar durante la recristalización del metal, denominado \textbf{maclas de recocido}.
	\item[HCP] Tienen solo 3 sistemas de deslizamiento por lo que es muy difícil encontrar alguno bien orientado para deslizar. Estos metales son los de \textbf{mayor tendencia a maclar}
\end{description}


\subsection{Endurecimiento por deformación en frío}

Durante la deformación plástica, las dislocaciones se mueven en varios sistemas de deslizamiento, interactuando entre sí debido a los campos de tensiones. Muchas de estas dislocaciones ven su movimiento dificultado por otras dislocaciones e incluso algunas se ``traban''. A partir de estas interacciones se generan \textbf{fuentes de dislocaciones} las cuales aumentan la cantidad de dislocaciones por unidad volúmen. 

Esto se convierte en un ciclo que se realimenta positivamente con las nuevas dislocaciones generadas tendiendo a anclarse más frecuentemente por la densidad de dislocaciones que solo aumenta. A medida que aumente la densidad de dislocaciones, también aumentará la dificultad en moverlas. Esto conlleva con un aumento en la tensión necesaria para deformar el metal plásticamente. Esto es conocido como el \textbf{endurecimiento por deformación en frío}.
