\section{Recocido -- Annealing}

Cuando se deforma un material plásticamente más de 90\% de la energía se convierte en calor irreversiblemente. Menos de 10\% queda almacenada en el material en forma de defectos. \textbf{Esto deja al material en un estado inestable donde $G$ no es mínima.} Cuando se caliente el material de modo que aumente la movilidad de los átomos, los {\bf defectos comenzarán a disminuir en cantidad}.


Esto da origen a los llamados \textbf{procesos de restauración}. Los mismos tratan de restaurar las propiedades del material antes de la deformación plástica. Durante los mismos hay una liberación de energía y el material pasa a un estado más estable ($G$ disminuye). Esta diferencia de entalpía libre ($\Delta G$) se denomina \textbf{fuerza impulsora} del proceso.

\subsection{Recuperación}
Al subir la temperatura de un metal previamente deformado comienza la recuperación:
\begin{itemize}
	\item Disminución de defectos puntuales como vacancias, defectos de Frenkel. Los mismos difunden hacia las dislocaciones, borde de grano y la superficie libre (sumideros de defectos)
	\item Redistribución de las dislocaciones. Se agrupan para disminuir la energía total
	\item Este proceso continua hasta formar ``paredes de dislocaciones''. Las paredes de dislocaciones son en realidad bordes de grano de bajo ángulo o \textbf{bordes de subgrano}. Este proceso se llama \textbf{poligonización}.
\end{itemize}

Durante esta etapa la forma y tamaño de los granos no cambia. Los subgranos solo pueden ser apreciados con un microscopio electrónico. Para aceros este proceso ocurre a $\approx 580 \grad < A_1$.

Lo más afectado por la recuperación es la conductividad eléctrica (aumenta) y la relajación de tensiones residuales. Aún así, la densidad de dislocaciones es alta y el metal puede seguir bajando su energía.


\subsection{Recristalización}

Si la deformación plástica previa superó cierto valor y dado suficiente tiempo y temperatura, luego de la recuperación se produce nucleación y crecimiento de nuevos granos con baja densidad de dislocaciones. Esto disminuye drásticamente la entalpía libre $G$ del material (la fuerza impulsora es grande).

Los nuevos granos nuclean en los antiguos borde de grano y subgrano. Este fenomeno se denomina \textbf{recristalización}.

\subsubsection{Influencia en propiedades}
La recristalización juega un rol muy importante en la metalurgia debido a los siguientes puntos

\begin{itemize}
	\item Tensiones residuales desaparecen
	\item Disminuye aún más la resistividad eléctrica
	\item Al bajar la cantidad de dislocaciones en varios ordenes de magnitud se \textbf{recupera la dureza, resistencia mecánica y ductilidad} que tenía el material antes de la deformación
\end{itemize}

La eliminación de dislocaciones por deformación permite seguir deformando el material sin riesgo de fisuración.

\subsection{Velocidad de recristalización}

La velocidad de recristalización es un parámetro fundamental y depende de la fuerza impulsora, la cual depende a la vez de tres variables
\begin{itemize}
	\item La temperatura (la de mayor influencia y más controlable)
	\item La deformación plástica previa
	\item El tamaño de grano previo a la recristalización (Mucho menos influyente que las dos anteriores)
\end{itemize}

Se define la \textbf{temperatura de recristalización} como aquella para la cual la recristalización se completa en 1 hora para un material con gran deformación previa ($\varepsilon > 75\%$). Se verifica que se relaciona a la temperatura de fusión para la gran mayoría de metales y aleaciones.


A menor tamaño de grano antes de la deformación, la recristalización se acelera pues los bordes de grano son sitios de nucleación de los nuevos granos. Además, el tamaño de las celdas formadas en el proceso de recuperación guarda relación con el tamaño original.


\subsection{Deformación crítica}

Para que la recristalización ocurra debe existir una energía mínima acumulada en el metal en forma de defecto, de lo contrario no habrá suficiente energía para la nucleación de nuevos granos. La deformación plástica mínima necesaria para que ocurra la recristalización se denomina \textbf{deformación crítica}.





