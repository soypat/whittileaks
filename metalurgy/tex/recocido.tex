\section{Recocido -- Annealing}

Cuando se deforma un material plásticamente más de 90\% de la energía se convierte en calor irreversiblemente. Menos de 10\% queda almacenada en el material en forma de defectos. \textbf{Esto deja al material en un estado inestable donde $G$ no es mínima.} Cuando se caliente el material de modo que aumente la movilidad de los átomos, los {\bf defectos comenzarán a disminuir en cantidad}.


Esto da origen a los llamados \textbf{procesos de restauración}. Los mismos tratan de restaurar las propiedades del material antes de la deformación plástica. Durante los mismos hay una liberación de energía y el material pasa a un estado más estable ($G$ disminuye). Esta diferencia de entalpía libre ($\Delta G$) se denomina \textbf{fuerza impulsora} del proceso.

\subsection{Recuperación}
Al subir la temperatura de un metal previamente deformado comienza la recuperación:
\begin{itemize}
	\item Disminución de defectos puntuales como vacancias, defectos de Frenkel. Los mismos difunden hacia las dislocaciones, borde de grano y la superficie libre (sumideros de defectos)
	\item Redistribución de las dislocaciones. Se agrupan para disminuir la energía total
	\item Este proceso continua hasta formar ``paredes de dislocaciones''. Las paredes de dislocaciones son en realidad bordes de grano de bajo ángulo o \textbf{bordes de subgrano}. Este proceso se llama \textbf{poligonización}.
\end{itemize}

Durante esta etapa la forma y tamaño de los granos no cambia. Los subgranos solo pueden ser apreciados con un microscopio electrónico. Para aceros este proceso ocurre a $\approx 580 \grad < A_1$.

Lo más afectado por la recuperación es la conductividad eléctrica (aumenta) y la relajación de tensiones residuales. Aún así, la densidad de dislocaciones es alta y el metal puede seguir bajando su energía.


\subsection{Recristalización}

Si la deformación plástica previa superó cierto valor y dado suficiente tiempo y temperatura, luego de la recuperación se produce nucleación y crecimiento de nuevos granos con baja densidad de dislocaciones. Esto disminuye drásticamente la entalpía libre $G$ del material (la fuerza impulsora es grande).

Los nuevos granos nuclean en los antiguos borde de grano y subgrano. Este fenomeno se denomina \textbf{recristalización}.

\subsubsection{Influencia en propiedades}
La recristalización juega un rol muy importante en la metalurgia debido a los siguientes puntos

\begin{itemize}
	\item Tensiones residuales desaparecen
	\item Disminuye aún más la resistividad eléctrica
	\item Al bajar la cantidad de dislocaciones en varios ordenes de magnitud se \textbf{recupera la dureza, resistencia mecánica y ductilidad} que tenía el material antes de la deformación
\end{itemize}

La eliminación de dislocaciones por deformación permite seguir deformando el material sin riesgo de fisuración.

\subsection{Velocidad de recristalización}

La velocidad de recristalización es un parámetro fundamental y depende de la fuerza impulsora, la cual depende a la vez de tres variables
\begin{itemize}
	\item La temperatura (la de mayor influencia y más controlable)
	\item La deformación plástica previa
	\item El tamaño de grano previo a la recristalización (Mucho menos influyente que las dos anteriores)
\end{itemize}

Se define la \textbf{temperatura de recristalización} como aquella para la cual la recristalización se completa en 1 hora para un material con gran deformación previa ($\varepsilon > 75\%$). Se verifica que se relaciona a la temperatura de fusión para la gran mayoría de metales y aleaciones.


A menor tamaño de grano antes de la deformación, la recristalización se acelera pues los bordes de grano son sitios de nucleación de los nuevos granos. Además, el tamaño de las celdas formadas en el proceso de recuperación guarda relación con el tamaño original.


\subsection{Deformación crítica}

Para que la recristalización ocurra debe existir una energía mínima acumulada en el metal en forma de defecto, de lo contrario no habrá suficiente energía para la nucleación de nuevos granos. La deformación plástica mínima necesaria para que ocurra la recristalización se denomina \textbf{deformación crítica}.

La deformación crítica no sólo depende, sino también de la forma en que se deforme el metal (tracción, torsión, compresión, etc) y oscila entre 2 y 20\%.

A mayor deformación aplicada al metal (siempre que supere la crítica), es menor el tamaño de grano recristalizado pues existen más sitios para la nucleación de granos nuevos. 

Dependiendo de la deformación aplicada, el tamaño de grano puede ser mayor o menor que el tamaño de grano original.

\subsection{Crecimiento de grano}

Una vez finalizada la recristalización el material queda con baja entalpía libre $G$. Sin embargo aún queda la posibilidad de seguir disminuyendo la misma si se reducen la cantidad de borde de grano.

Si hay suficiente temperatura y tiempo entonces el \textbf{tamaño promedio de grano aumenta}. A diferencia del crecimiento de grano durante la recristalización, este proceso tiene como fuerza impulsora la energía acumulada en los bordes de grano. Por ende, este crecimiento de grano puede ocurrir aún si no hay deformación previa existente (la cual era la fuerza impulsora para la recristalización)!

Este proceso elimina los granos pequeños (borde de grano con curvatura alta=energía alta) mientras que aumentan los granos grandes (energía baja).

Este crecimiento trae algunos inconvenientes

\begin{itemize}
	\item Disminución de la tensión de fluencia (ley Hall-Petch \eqref{eq:hall_petch})
	\item Baja la tenacidad y sube la temperatura de transición dúctil-frágil
	\item Afecta otras propiedades como la susceptibilidad a la fisuración por temple
\end{itemize}
debido a esto se intenta evitar el crecimiento de grano. Se debe tener un control muy preciso de la temperatura y tiempo a la cual es recocido el metal.

Cuando en el metal existe una \textbf{dispersión fina y homogénea de partículas de otra fase}, entonces estas interfieren con el movimiento de los bordes de grano efectivamente ralentizando el crecimiento. Se llega a un tamaño de grano límite o de equilibrio que está dado por 

\begin{equation}
	d_{\max} = \frac{4 r}{3 f}
\end{equation}
donde $d_{\max}$ es el tamaño de grano que estará en equilibrio con una fracción en volumen $f$ de partículas de tamaño promedio $r$. Esto logra controlar el tamaño máximo del grano incluso a altas temperaturas.


\subsection{Recocido de recristalización}

Las tres etapas estudiadas ocurren durante el tratamiento térmico de \textbf{recocido de recristalización}, también denominado recocido de procesamiento.

Tiene el objetivo de restituir las propiedades necesarias para seguir conformando el material en frío sin riesgo de fisurar el material.

Existe la posibilidad de deformar al material en caliente y que ocurra la recristalización durante el mismo proceso de deformación. A esto se denomina el \textbf{deformación en caliente}. Puede ser \textbf{dinámica} (recristalización durante la deformación) o \textbf{estática} (recristalización inmediatamente después de la deformación).

Puntos a remarcar de \textbf{deformación en caliente}
\begin{itemize}
	\item Permite grandes deformaciones sin cambiar resistencia y ductilidad
	\item Requiere de un cierto consumo de energía en los hornos de recalentamiento
	\item A igualdad de deformación, la energía requerida es sensiblemente menor que en el conformado en frío
	\item Se pierde cierta cantidad de material por oxidación
	\item Las tolerancias dimensionales deben ser grandes (orden del mm)
	\item Mala terminación superficial
\end{itemize}

Puntos a remarcar de \textbf{deformación en frío}
\begin{itemize}
	\item Limitada cantidad de deformación. Se pierde ductilidad y aumenta la resistencia a deformación
	\item No requiere calentamiento previo
	\item A igualdad de deformación, la energía requerida es sensiblemente mayor que en el conformado en caliente
	\item Elevadas fuerzas de conformado (miles de toneladas)
	\item No se produce oxidación
	\item Tolerancias chicas y terminación superficial buena
\end{itemize}


En comparación con la estructuras obtenidas en la colada, las estructuras de un material trabajado en caliente:
\begin{itemize}
	\item Tienen granos más finos y uniformes que los obtenidos en colada
	\item Composición química más homogénea
	\item Ausencia de poros y microrechupes
	\item Ausencia de segregaciones (macro y micro)
	\item Propiedades más uniformes aunque pueden ser anisótropas
\end{itemize}

\subsection{Fibrado mecánico}

La deformación plástica excesiva que involucra casi cualquier conformado en caliente trae un efecto de ``fibrado''. Durante la deformación se estiran y deforman algunos tipos de inclusiones las cuales se alinean según la dirección de estiramiento principal. Lo más común es que las inclusiones sean zonas de microsegregación de lingotes (zonas interdendríticas donde se concentra el aleante secundario).

Este fibrado le da propiedades direccionales a la pieza, las cuales pueden ser adversas en varios casos.