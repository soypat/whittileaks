\section{Tensiones residuales}

Las tensiones residuales son las tensiones que subsisten en el material aún en ausencia de cargas externas. Consecuentemente, deben estar en equilibrio estático y ser \textbf{tensiones elásticas}.

Causas:
\begin{itemize}
	\item Gradientes térmicos
	\item Transformaciones de fases en presencia de gradientes térmicos
	\item Deformación plástica inhomogénea
\end{itemize}

\subsection{Gradientes térmicos}

En piezas grandes es más difícil evitar las tensiones residuales por gradiente térmico. Se debe enfriar la pieza lentamente para evitarlas por completo, algo que puede resultar costoso.

Estas tensiones se generan porque al enfriarse una pieza muy caliente hay una contracción del exterior de la pieza que se enfrió rápidamente mientras que el interior de la pieza sigue caliente. Esto causa un perfil de tensiones: el exterior se tracciona y el interior se comprime. Si el gradiente es lo suficientemente alto el interior de la pieza se deforma plásticamente debido a esta tracción.

A medida que sigue enfriando, el interior comienza a contraerse. Si hubo deformación permanente en la etapa posterior entonces el interior de la pieza va contraerse a tal punto de someter sus alrededores a tracción (la superficie de la pieza). Esto causa una inversión de las tensiones, ahora el exterior está comprimido y el interior sometido a tracción.

\subsection{Transformaciones de fase}
En el caso en que el metal tenga transformaciones de fases, las tensiones térmicas y las residuales estarán influidas también por la contracción o expansión que involucren dichas fases.

Por ejemplo, la transformación martensítica durante el temple del acero causa tracción en la superficie de la pieza. Esta tensión residual es más propensa a la fisura e incluso tiene un nombre asociado con el fenómeno: \textbf{fisuración durante el temple}.

\subsection{Deformación plástica inhomogénea}

Cualquier proceso que deforme plásticamente el material, produce tensiones residuales en la pieza.

\begin{description}
	\item[Laminación en frío] Si los rodillos laminadores tienen un diámetro pequeño comparado al espesor de la chapa entonces solo deforman la superficie plásticamente generando tensiones residuales de compresión en la superficie y de tracción en el corazón
	\item[Granallado (shot peening)] Un proceso que consiste en impulsar pequeñas partículas con un chorro de aire sobre una superficie. Las partículas impactan la superficie deformándola plásticamente. Genera tensiones residuales de compresión altas, endurece por deformación plástica y aumenta levemente la rugosidad superficial. Usado para aumentar \textbf{la resistencia a la fatiga }
\end{description}






