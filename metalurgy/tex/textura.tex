\section{Textura cristalina}

En un policristal se denomina textura cristalina o simplemente \textbf{textura} a la orientación preferencial (de la red cristalina) de los granos del policristal.

La causa de textura puede ser {\bf deformación plástica, solidificación columnar o recristalización (texturas de recocido).} 

\subsection{Consecuencias de la textura}
La textura \textbf{no} está relacionada con la forma de los granos.

\begin{itemize}
	\item Anisotropía
\end{itemize}

\subsection[Usos de la textura]{Usos de la textura -- Embutido}
La textura está presente en la mayoría de los metales y a veces no es buscada pues trae inconvenientes. En ciertos procesos es deseada, como por ejemplo en chapas para embutido profundo y chapa eléctrica.

En el embutido profundo el material del ala (parte de la chapa sin embutir) sufre compresión en la dirección tangencial, tracción radial y una ligera compresión en el espesor. El material de la pared (parte de chapa en contacto con los costados de la matriz) sufre principalmente tracción biaxial en el plano que la contiene y una ligera compresión en el espesor. El material del fondo está sometido a tracción biaxial también.

El material buscado para el proceso del embutido requiere tener \textbf{alta ductilidad}, debe deformarse fácilmente ante tensiones de compresión. Para embutidos profundos la reducción de espesor debe ser baja mientras que la reducción de ancho debe ser alta, es decir, debe tener comportamiento anisótropo. Esto se mide mediante el coeficiente de anisotropía \eqref{eq:coef_anisotropia}

\begin{equation} \label{eq:coef_anisotropia}
	R = \frac{\varepsilon_w}{\varepsilon_t}
\end{equation}

Las chapas para embutidos profundos deben tener una composición química adecuada y son fabricadas mediante un proceso de deformación por laminación (para otorgarle textura, y consecuentemente, anisotropía) seguido de recocidos posteriores. La textura producida es una textura de recocido y no solo de deformación. Los granos de este tipo de chapa son equiaxiales (no son alargados).

Existen dos otros coeficientes relacionados al embutido, el coeficiente de anisotropía normal ($R_p$) que define la profundidad máxima del embutido, y el coeficiente de anisotropía plana ($\Delta R$)

\begin{equation}
	R_p = \frac{R_{0^\circ} + 2R_{45^\circ} + R_{90^\circ}}{4}, \qquad \Delta R = \frac{R_{0^\circ} - 2R_{45^\circ} + R_{90^\circ}}{2}
\end{equation}
Se desea chapas de alto $R_p$ y bajo $\Delta R$ para embutido profundo. A mayor $\Delta R$ mayor será el orejado y por ende, mayor material perdido.

La relación de embutido se define como el diámetro inicial de la chapa sobre el diámetro del fondo de la pieza embutida:

\begin{equation}
	\beta = \frac{D_0}{D}
\end{equation}


