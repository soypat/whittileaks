% !TeX spellcheck = es_ES
% !TeX root = ../metalurgy.tex
\part{Aceros para construcción Mecánica}
Para fabricación de piezas de maquinaria. Cadena cinemática, transmisión de potencia, elementos roscados, herramientas manuales, industria automotriz, agrícola, aeronáutica, ferroviaria, etc.

Los \textbf{requerimientos de servicio} para un acero incluye propiedades mecánicas como $R_m$, $R_{p0,2}$, tenacidad, resistencia a la fatiga, resistencia al desgaste, resistencia a cargas de contacto. Luego se pueden exigir \textbf{caracteristicas de fabricación} como formabilidad, maquinabilidad y ductilidad. Los requerimientos de servicio y características de fabricación suelen obtenerse de tratamientos térmicos.

\section{Clasificación}

Existen varios tipos de clasificaciones de aceros para contrucción mecánica:

\subsection*{Según aleantes}
\begin{description}
	\item[Aceros al C] Más baratos, fácil fabricación y de menor templabilidad
	\item[Ac. de baja aleación] Buena templabilidad, menor severidad de temple requerida, buena maquinabilidad, buena resistencia al revenido
\end{description}


\subsection*{Clasificación por \%C}
\begin{description}
	\item[Bajo C ($<0,25\%$)] Para cementación/carburización (engranajes, árboles, cadenas)
	\item[Medio C ($0,25 \textrm{ a }0,5\%$)] Aceros para bonificado. Durezas $\approx 30$ a $40$ HRC
	\item[Bajo C ($>0,5\%$)] Máxima resistencia y dureza. Se usan bonificados, martemperados+revenidos, austemperados. Costo alto de fabricación de piezas por baja maquinabilidad y formabilidad.
\end{description}

\subsection*{Clasificación por templabilidad}
Puede ser baja, media o alta. El requerimiento de templabilidad va estar en función a la solicitación.

\begin{description}
	\item[Tracción/corte puro] 90\% mínimo de martensita en el centro. $R_{p0,2}>1200$ MPa. Bulones y tornillos de alto grado
	\item[Flexión/Torsión pura] 50\% martensita en el centro. Árboles, ejes y resortes
\end{description}

\section{Resistencia a la fatiga}
Se determina con un ensayo en la maquina de Moore (flexión rotativa $\sigma_m=0$;$\sigma_a = |\sigma_{\max}|$) definiéndose así las zonas de bajo y alto ciclado.

\subsection*{Como maximizar la resistencia a la fatiga}
En aceros al C de baja aleación con dureza 45 HRC se cumple que $R_f \approx 0,35 - 0,6R_u$ (Shigley toma $0,5R_u$). Esto es valido para una probeta lisa sin tensiones residuales bajo flexión alternativa.

Si se tiene una entalla
\[
q = \frac{k_f -1}{k_t - 1}
\]
$q$ indica cuanta tensión es relevada. $q=1 \Rightarrow$ muy sensible a la entalla.  Para maximizar la resistencia a la fatiga se necesita alta ductilidad y alta $R_m$. Se puede obtener por temple y revenido, o austemperado.

\section{Aceros al boro}
El boro promueve templabilidad a bajo costo en aceros de bajo y medio carbono. El boro (intersticial en la austenita) segrega a los bordes de grano retrasando la nucleación de ferrita proeutectoide. Se corren las curvas CCT a la derecha, ganándose mejor templabilidad.

El boro se puede encontrar en estos aceros en solución sólida, como óxido, nitruros, boruros. Se corre el peligro que al agregar demasiado boro vaya a formar un compuesto que no cumpla ser soluto.

\section{Aceros de corte libre}

Son aquellos con algunos elementos (P y S en general) que se usan cuando la pieza requiere mucho mecanizado y el costo es más importante que las propiedades finales. El azufre hace que se entrecorte la viruta y funcione como lubricante.  El fósforo en solución sólida hace que se fragilice la ferrita logrando viruta entrecortada. SAE 1100$\rightarrow$S; SAE 1200$\rightarrow$S,P.
