\section{Introducción a aceros}

Porque aceros?

\begin{itemize}
	\item Bajo costo (25\% de energía requerida comparado con aluminio)
	\item Alta rigidez ($E$ 3 veces más rígido que aluminio)
	\item Versatilidad. Gran cantidad de aleaciones
	\item Respaldo. Se vienen usando aceros de alta aleación hace un siglo
\end{itemize}

Hay 3 variedades alotrópicas de acero más comunes
\begin{description}
	\item[Ferrita $\alpha$] Fase bland, dúctil y tenaz con $\Tdf$ (Temperatura transición dúctil frágil). BCC y ferromagnética
	\item[Austenita $\gamma$] Contiene carbono en solución. Red FCC. Blanda, dúctil, y tenaz sin $\Tdf$. Es amagnética.
	\item[Cementita $\cementita$] Intermetálico 6,7\% carbono. Muy dura   
\end{description}

En general se tiene una matriz de ferrita que aporta tenacidad y ductilidad y una proporción de cementita para la dureza.

Las temperaturas de importancia para una aleación de acero son (ver figura \ref{fig:diagAceros} del diagrama meta-estable)

\begin{description}
	\item[\Aone] Temperatura del eutectoide
	\item[\Athree] Temperatura límite del campo de $\alpha + \gamma$ y $\gamma$
	\item[\Atwo] Temperatura de transición de ferrita magnética a ferrita amagnética  (no visible en la figura \ref{fig:diagAceros})
\end{description}

Elementos relevantes a la metalurgia del acero

\begin{description}
	\item[Si] Calma acero evitando efervescencia ($\approx0,05\%$)
	\item[Al] Controla crecimiento de grano. Granos más finos y calma acero.
	\item[S] El azufre es una impureza. En ausencia de manganeso (Mn) forma FeS, el cual derrite a 988\grad. Si se sobrepasa este límite se corre riesgo de que segregue a los bordes de grano al volver a solidificar. Esto disminuye resistencia y aumenta fisuras en trabajo en frío. El azufre aumenta maquinabilidad
	\item[P] Impureza. Endurece y fragiliza
	\item[Mn] Contrarresta S para formar MnS, el cual tiene alto punto de fusión con efectos negativos menores al FeS porque no precipita en los bordes    
\end{description}

