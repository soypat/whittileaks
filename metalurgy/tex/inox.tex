% !TeX spellcheck = es_ES
% !TeX root = ../metalurgy.tex
\part{Aceros inoxidables}
Grupo de aceros de alta aleación diseñados para tener resistencia a la corrosión en un amplio rango de servicio (buena $R_m$ a altas temperaturas).

Las aplicaciones principales de estos aceros son en plantas generadoras de energía, centrales hidráulicas, plantas petroquímicas, industria farmacéutica, piezas de maquinas, arquitectura (decoración), aplicaciones marítimas.

\section{Corrosión}
Deterioro de un material por la acción química del medio que lo rodea. A excepción de los nobles, los metales son termodinámicamente inestables al contacto con el aire y forman óxidos. Si la reacción es ``lenta"{} (desde un punto de vista cinético) la corrosión puede llegar a aceptarse. Si es ``rápida"{} y limita la vida del componenta surgen problemas de costo y fiabilidad mecánica.

\subsection{Clasificación}
\begin{description}
	\item[Generalizada] A nivel macroscópico, la superficie es atacada uniformemente. Es predecible y controlable. Se puede prevenir si se diseña con un sobre-espesor uniforme. El daño es proporcional a la cantidad de material removido.
	\item[Localizada] Daño localizado progresa rápido. Resulta difícil predecir, controlar y detectar. El daño puede ser muy grande aunque la cantidad de material sea mínima. Es muy problemático. A continuación se tienen 4 tipos de corrosión localizada:
	\begin{description}
		\item[Picado/Pitting] Promovido por la presencia de iones $Cl^{-1}$, alta temperatura y baja velocidad de circulación del fluido. Si la capa pasivante se rompe localmente y no se forma rápido otra la corrosión avanza y se genera pozos en forma de túneles que pueden perforar el material en corto tiempo aumentando el riesgo de avance de fisura por fatiga.
		\item[Rendijas] Se da en zonas donde el fluido no llega a circular. Cuando se consume el oxigeno, se localiza la corrosión en la rendija. Evitable con buen diseño de pieza, reduciendo iones de cloro y reduciendo la temperatura
		\item[Bajo tensión] Puede dar lugar  fallas catastróficas a partir de fisuración provocada por la combinación del \textbf{medio} y \textbf{tensión de tracción}. Em general la fisura rompe por fractura frágil. Los aceros inoxidables austeníticos y martensiticos son los más susceptibles
		\item[Granular] Asociada a fenómeno de precipitación de carburos ricos en Cr en los bordes de grano de las microestructuras.
	\end{description}
\end{description}


\subsection{Pasividad}
La pasivación de un metal se refiere a la formación de una delgada capa de óxido al ser sometido a un diferencial de potencial ($\Delta V$) mayor al potencial de pasivación. Esta capa aísla al metal del medio y hace que disminuya la velocidad de corrosión en varios ordenes de magnitud. Para que sea efectiva la capa esta debe ser fina, continua, no porosa, insoluble en el medio, y debe poder regenerarse rápidamente al ser dañada (ralladura, mecanizado). 

Ek aluminio, por ejemplo, no es electronegativo pero es fuertemente pasivado. El hierro puede pasivarse pero a un alto $\Delta V$. La inclusión de cromo (Cr) en un 12\% o más hace que el metal pueda oxidarse fácil

\subsubsection*{Variables en la estabilidad de la capa pasivante}
\begin{description}
	\item[Composición química del acero] Factor principal es el contenido de Cr (12\% mínimo) para pasivar con soluciones acuosas neutras. En general la heterogeneidades (segregaciones, precip. de carburos) hacen que disminuya la estabilidad.
	\item[Composición del medio] Los inoxidables se pasivan solo cuando el medio es altamente oxidante (acuosos, $HNO_3$). La presencia de iones de halógenos es la principal razón por fallas. Estos iones desestabilizan y generan Pitting. Un medio básico (pH) es más estabilizante para una capa pasivante.
	\item[Variables operativas] En general, la resistencia a la corrosión disminuye con el aumento de la temperatura. La velocidad relativa entre el medio y la superficie afecta la formación de la capa. A mayor velocidad hay mayor aporte de $O_2$, aumentando la velocidad de oxidación.
	\item[Factores de diseño] La presencia de rendijas o alta rugosidad crean zonas que favorecen la oxidación localizada.
\end{description}

\section{Aceros inoxidables austeníticos}
Se trata del grupo de aceros inoxidables que mejor combina propiedades mecánicas y tecnologías con resistencia a la corrosión a precio razonable. Constituyen 70\% de la producción total de aceros inoxidables. 

La composición típica de un acero austenítico inoxidable:
\begin{description}
	\item[Cr] 16 a 30\%
	\item[Ni] 8 a 30\%
	\item[Mo] Hasta 4\%
	\item[Mn] $\approx$ 2\%
	\item[C] Menor de 0,1\%
\end{description}

\subsubsection*{Elección de gamágeno}
El Cr es un elemento alfágeno, entonces se necesita de un gamágeno para estabilizar la austenita a temperatura ambiente.
\begin{description}
	\item[C] Debe ser bajo porque favorece la precipitación de carburos de Cr, lo cual favorece la corrosión
	\item[N] No se usa ya que no estabiliza la austenita hasta bajas temperaturas
	\item[Mn] Se usa en casi todos los austeníticos pero no es el gamágeno principal. EStabiliza la austenita a temperatura ambiente pero no baja tanto $M_s$
	\item[Ni] Estabiliza austenita hasta bajas temperaturas. Baja la $M_s$ (reduciendo la formación de martensita, cosa indeseable), mejora la tenacidad y aumenta la resistencia a la corrosión. A mayor cantidad de Cr, Mo y otros alfágenos se requiere de mayor Ni para balancear.
\end{description}

\subsubsection*{Ventajas de la estructura FCC}
\begin{itemize}
	\item Ductilidad y baja $R_{p0,2}$. Alta capacidad de deformación en frío
	\item No hay transición dúctil frágil, alta tenacidad
	\item Resistencia al fenómeno del creep (fluencia a altas temperaturas)
	\item Los intersticiales tienen más solubilidad y menor difusividad. El Cr precipita más lento
	\item Amagnética
\end{itemize}

A temperatura ambiente un acero inoxidable austenítico contiene muy poca martensita por el $M_s$ bajo, poca ferrita y algunos carburos.

\subsubsection{Precipitación de carburos de Cromo}
El \%C disminuye la temperatura a la cual precipitan los carburos del Cromo y la ralentiza. Los carburos de Cromo nuclean en los borde de grano y bordes de maclas de la austenita. Alrededor de los carburos queda una zona empobrecida en Cr, lo cual lleva a la \textbf{sensibilización}.

La sensibilización da origen a la corrosión intergranular y ocurre frecuentemente en aceros austeníticos ya que cuando hay un proceso de alta temperatura el carbono se disuelve y re-precipita al enfriarse.
