
\section*{Glosario}
Subíndice $1$ indica la entrada al rotor, $2$ indica salida del rotor, $B_s$ indica el estado $B$ para el proceso isoentrópico.
\par

\glossentry{\speedsound}{La velocidad del sonido en un fluido}


\glossentry{c}{Velocidad absoluta del fluido}
% \glossentry{\cax{}=V}{Velocidad axial del fluido}
\glossentry{\ctan{}}{Velocidad tangencial del fluido}
\glossentry{\cax{}}{Velocidad axial del fluido}

\glossentry{\crad{}}{Velocidad radial del fluido}
\glossentry{w}{Velocidad relativa del fluido con respecto al rotor}
\glossentry{U}{Velocidad del impulsor}
\glossentry{\cteuniversal=8,314 \si{\joule \per \kelvin\per \mole}}{Constante universal para gases ideales}
\glossentry{R=\molarmass\cteuniversal}{Constante especifica de un gas ideal \sib{\joule \per \kilogram \per \kelvin} donde \molarmass{} es la masa molar \sib{\kilogram \per \mole}}
\glossentry{\degreeofreaction}{Grado de reacción}
\glossentry{\relcomp}{Relación de compresión}
\glossentry{r,D}{Radio y diámetro, respectivamente}
\glossentry{\Mach=\frac{c}{\speedsound}}{Número de Mach}
\glossentry{\alpha}{Ángulo entre velocidad absoluta con respecto al plano meridional}
\glossentry{\attackangle}{Ángulo de ataque del alabe}
\glossentry{\beta}{Ángulo entre velocidad relativa con respecto al plano meridional}
\glossentry{\delta}{Ángulo de desviación del alabe o espesor de capa límite}
% \glossentry{\slipangle}{Ángulo de deslizamiento}
\glossentry{\psi}{Coeficiente}
\glossentry{\phi}{Coeficiente de flujo $\frac{\crad{2}}{U_2}$}
\glossentry{\slip}{Coeficiente de deslizamiento}
\glossentry{\Omega}{Velocidad angular del rotor}


\clearpage
% end of glossary
%------------------
\setcounter{section}{-1}