% !TeX program = pdflatex
% !TeX spellcheck = es_ES
% !TeX encoding = utf8
\documentclass[11pt, a4paper, twoside, openright, openany]{book}
% Math text
\usepackage[utf8]{inputenc}
\usepackage[a4paper,
            top=2cm,bottom=2cm,
            margin=3.5cm,
            headheight=27pt]{geometry}
\usepackage[spanish, mexico]{babel}
%\usepackage[utopia]{mathdesign}
\usepackage{amsmath,amssymb,array,siunitx}
% HEADER
\usepackage{fancyhdr,framed}
\setlength{\headheight}{44pt}
\pagestyle{fancy}
\usepackage[T1]{fontenc}
\usepackage[utf8]{inputenc}
\usepackage{soulutf8,xcolor}
\usepackage{svg}
\usepackage{graphicx}

% !TeX root = control_rapido.tex
%% DIFFERENTIAL OPERATOR
\makeatletter
\providecommand*{\diff}%
{\@ifnextchar^{\DIfF}{\DIfF^{}}}
\def\DIfF^#1{%
	\mathop{\mathrm{\mathstrut d}}%
	\nolimits^{#1}\gobblespace}
\def\gobblespace{%
	\futurelet\diffarg\opspace}
\def\opspace{%
	\let\DiffSpace\!%
	\ifx\diffarg(%
	\let\DiffSpace\relax
	\else
	\ifx\diffarg[%
	\let\DiffSpace\relax
	\else
	\ifx\diffarg\{%
	\let\DiffSpace\relax
	\fi\fi\fi\DiffSpace}


\newcommand{\dimfont}[1]{\ensuremath{#1}}
\newcommand{\Cme}[1]{\mathbf{#1}}
\newcommand{\Mme}[1]{\mathbf{#1}}
\newcommand{\Ts}{T_s}

\newcommand{\eye}{\Mme{I}}
%transpose
\def\dt{\Delta t}
\def\tp{^{\top}}
% Small
\def\MA{\Mme{A}}
\def\MB{\Mme{B}}
\def\MC{\Mme{C}}
\def\MD{\Mme{D}}
\def\ME{\Mme{E}}
\def\MK{\Mme{K}}
\def\MW{\Mme{W}}
\def\MX{\Mme{X}}
\def\MV{\Mme{V}}
\def\ctrb{\Mme{Y}}
\def\Mzero{\Mme{0}}
\def\obsv{\Mme{\mathcal{O}}}

\newcommand{\ts}[2]{\left. #1\right|_{\tiny #2}}

\def\error{\varepsilon}
\def\Cx{\Cme{x}}
\def\Cf{\Cme{f}}
\def\Cy{\Cme{y}}
\def\Cu{\Cme{u}}
\def\Cn{\Cme{n}}
\def\Cd{\Cme{d}}
\def\Czero{\Cme{0}}
\def\Cv{\Cme{v}}
\def\Cz{\Cme{z}}
\def\Cw{\Cme{w}}
\def\Jcost{\Cme{\mathcal{J}}}
% RK4
\def\Ca{\Cme{a}}
\def\Cb{\Cme{b}}
\def\Cc{\Cme{c}}


\def\noise{{n}}
\def\disturb{{d}}
\newcommand{\di}{\ensuremath{\textrm{d}}}

\newcommand{\Matlab}{{\sc Matlab}}

\newcommand{\spartial}[2]{\frac{\partial {#1}}{\partial {#2}}}
\newcommand{\dpartial}[2]{\frac{\partial^2 #1}{\partial #2 ^2}}

\usepackage{amsthm}
\theoremstyle{definition}
\newtheorem{definition}{Definition}[chapter]
\newtheorem{theorem}{Teorema}[chapter]
\newtheorem{exercise}{Ejercicio}[chapter]
%\usepackage{bigfoot} % to allow verbatim in footnote
\usepackage[numbered,framed]{matlab-prettifier}

\let\ph\mlplaceholder % shorter macro
\lstMakeShortInline"
\renewcommand{\lstlistingname}{Código}
\renewcommand{\lstlistlistingname }{Códigos \Matlab}
\lstset{
  style              = Matlab-editor,
  basicstyle         = \mlttfamily,
  escapechar         = ",
  mlshowsectionrules = true,
  numbers = none,
  tabsize=4,
  literate = {-}{-}1,
}


\newcommand{\dimin}{\dimfont{p}}
\newcommand{\dimout}{\dimfont{q}}
\newcommand{\dimdisturb}{\dimfont{r}}
\newcommand{\dimss}{\dimfont{n}}

\begin{document}
%	% !TEX encoding = UTF-8 Unicode
% !TEX root = ../control.tex
\begin{titlepage}
	\onehalfspacing
	\enlargethispage{0.65\baselineskip}
	\par
%	\begin{large}
		\vspace{-0.1cm}
		\noindent \textbf{DEPARTAMENTO DE INVESTIGACIÓN Y DOCTORADO}\par
		\vspace{6cm}
		\begin{center}
			{\Huge \textbf{Title}\par}
			{\huge \textbf{Subtitle}\par}
		\end{center}
		\vfill
		\noindent \textbf{AUTOR:}  \par
		\vspace{1cm}
		\noindent \textbf{DIRECTOR:}  \par
		\noindent \textbf{CO-DIRECTOR:} \par	
		\vspace{1cm}
		\noindent{TESIS PRESENTADA PARA OPTAR AL TÍTULO DE}\par
		\noindent\textbf{\@degree}\par
		\vspace{1cm}
		\noindent\textbf{Jurado}\par
		\begin{normalsize}		
			\noindent%
			\makebox[0.33\textwidth][l]{Evaluator}

		\end{normalsize}
		\vspace{1cm}
		\begin{center}
			\textbf{CIUDAD AUTÓNOMA DE BUENOS AIRES}\\
			\textbf{\date}\par
		\end{center}
%	\end{large}
\end{titlepage}'
\tableofcontents
\part{Control Castellano - Bootcamp}
\chapter{Sistemas lineales}

El exponencial matricial es poco práctica para calcular con la matriz \(\Mme{A}\).

En cambio, lo que se hace en la practica es usar los autovalores y autovectores para efectuar una transformación de coordenadas de las coordenadas de $\Cx$ a las coordenadas de algún autovector donde es mas fácil escribir el exponencial matricial y facilita entender el sistema también.

Un \textbf{autovector} $\Cme{\xi}\in\mathbb{C}^\dimss$ cumple con la siguiente igualdad. $\lambda \in \mathbb{C}$ son los \textbf{autovalores} del sistema.
\[
\Mme{A}\Cme{\xi} = \lambda \Cme{\xi}
\]
Una forma de visualizar esto es que el producto entre la matriz $\Mme{A}$ y el autovector mantiene la dirección del autovector.

\[
\Mme{T} = \left[ \xi_1,\, \xi_2,\, \ldots\, \xi_\dimss \right]
\]

\[
\Mme{D}= \begin{bmatrix}
\lambda_1 & & & 0 \\
 & \lambda_2 & & \\
  & & \ddots & \\
 0 & & & \lambda_\dimss 
\end{bmatrix}
\]

Es posible diagonalizar el sistema siempre que no se tengan dos autovectores cuasi-paralelos o un sistema degenerado (autovectores generalizados) (entre otros casos). 

Esto nos deja escribir la relación

\[
\Mme{A} \Mme{T} = \Mme{T} \Mme{D}
\]

\[
\Cme{x} = \Mme{T}\Cme{z} \Rightarrow \Mme{T}^{-1} \Mme{A} \Mme{T} = \Mme{D}
\]

\[
\Mme{T} \dot{\Cme{z}} = \Mme{A} \Mme{T} \Cme{z} \Rightarrow \dot{\Cme{z}} = \Mme{T}^{-1} \Mme{A} \Mme{T} \Cme{z}
\]

Se obtienen entonces un sistema de ecuaciones desacoplado! El cambio de la variable $z_i$ depende de si misma
\[
\boxed{\dot{\Cme{z}} = \Mme{D} \Cme{z}}
\]

Se pueden obtener estas matrices en \Matlab~ en una linea:
\begin{lstlisting}
[T, D] = eig(A);
\end{lstlisting}

La solución del sistema va ser simple

\[
\Cme{z}(t) = \begin{bmatrix}
e^{\lambda_1 t} & & & 0 \\
 &e^{\lambda_2 t}& &  \\
 & &\ddots &  \\
  0& & & e^{\lambda_\dimss t} \\
\end{bmatrix} \cdot \Cme{z}(0)
\]

Es de interes poder mapear entre los dos espacios. Usando la expresión \( \Mme{A} = \Mme{T} \Mme{D} \Mme{T}^{-1}\) se puede simplificar la exponencial matricial empleando conocimientos de algebra lineal

\[
e^{\Mme{A}t} = e^{\Mme{T} \Mme{D} \Mme{T}^{-1}t} = \Mme{T} e^{\Mme{D}t} \Mme{T}^{-1}
\]
Cabe destacar que es barato calcular \( e^{\Mme{D}t}\) en términos computacionales. 

Reescribimos la solución al sistema recordando \(\Cme{x} = \Mme{T}\Cme{z}\)

\[
\Cme{x}(t) = \Mme{T}  \underbrace{e^{\Mme{D}t} \underbrace{\Mme{T}^{-1} \Cme{x}(0)}_{\Cme{z}(0)}}_{\Cme{z}(t)}
\]
La igualdad de arriba se usa para computar la evolución de $\Cme{x}$ en el tiempo aprovechando la simplicidad del cálculo de \(e^{\Mme{D}t}\). 

Que hicimos?
\begin{itemize}
	\item Descubrimos que si sabemos los autovectores/valores de \(\Mme{A}\) podemos transformar el sistema a un sistema de coordenadas donde es más facil resolver el sistema y estudiar su dinámica
\end{itemize}
 
El próximo paso es agregar la matriz de control y el vector de entrada para empezar a controlar el sistema.

\chapter{Estabilidad y autovalores}


Para el estudio de estabilidad podemos primero mirar a la igualdad

 \[
\Cme{x} = \Mme{T} e^{\Mme{D}t} \Mme{T}^{-1} \Cme{x}(0)
\]
Si uno de los valores diagonal de $e^{\Mme{D}t}$ se va a infinito entonces la combinación resultante que iguala a \(\Cme{x}\) también se va ir al infinito. Recordemos que $\lambda \in \Bbb{C}$

\[
\lambda = a + ib \quad \Rightarrow \quad e^{\pm \lambda t} = e^{at}\left[\cos(bt)\pm i\sin (bt)\right]
\]
Esto cuenta la siguiente historia
\begin{IEEEeqnarray}{tt}
si \(a>0\) & El sistema aumenta hasta llegar a infinito (\textbf{Inestable}) \\
si \(a<0\) & El sistema converge a cero a tiempo infinito (\textbf{Estable}) \\
\end{IEEEeqnarray}


Esto significa que tal vez comenzemos con un sistema inestable, es decir, que nuestra matriz \(A\) de la siguiente ecuación tiene autovalores con $a>0$
\[
\dot{\Cme{x}} = \Mme{A} \Cme{x}
\]

Esto se puede remediar agregando el término $\Mme{B}\Cme{u}$ de tal forma que lleve los autovalores de la zona inestable ($a>0$) a la zona estable ($a<0$).

\section{Evolución discreta}

\[
\Cme{x}_{k+1} = \Mme{\tilde{A}} \Cme{x}_k, \qquad \Cme{x}_k
 = \Cme{x}(k\Delta t)\]
 donde \(\Mme{\tilde{A}} = e^{\Mme{A} \Delta t}\). Sabiendo el vector de estado inicial podríamos calcular el estado para cualquier otro momento
 
\[
\Cme{x}_{N} = \Mme{\tilde{A}}^{N}\Cme{x}_0
\]

En coordenadas de autovector, cada vez que multiplicamos la matriz estamos elevando nuestros autovalores de $\Mme{\tilde{A}}$ a una potencia. Estos pueden agrandarse o achicarse dependiendo de su `radio'
\[
\lambda^N = R^{N}e^{i \,N\theta }
\]
si el radio $R$ es menor a uno, la magnitud va decaer a medida que pasa el tiempo. Si el radio es mayor que uno crecerá sin cota superior.

\begin{lstlisting}
[Tt, Dt] = eig(At);
aval = diag(Dt);
inestables = aval(aval>1);
\end{lstlisting}

\chapter{Linealizando un sistema}

\begin{enumerate}
	\item Encontramos los puntos fijos $\Cme{\bar{x}}$ tal que \(f(\Cme{\bar{x}})=\Cme{0}\)
	\item Linealizamos alrededor de $\Cme{\bar{x}}$
\end{enumerate}

Para un sistema \(2\times 2\) el jacobiano es
\[
\spartial{\Cme{f}}{\Cme{x}}=
\begin{bmatrix}
\spartial{f_1}{x_1} &\spartial{f_1}{x_2} \\
\spartial{f_2}{x_1} & \spartial{f_2}{x_2}
\end{bmatrix}
\]


\[
\dot{\Cme{x}} = f(\Cme{x}) = f(\Cme{\bar{x}}) + \left. \spartial{\Cme{f}}{\Cme{x}}\right|_{\Cme{\bar{x}}}\cdot (\Cme{x}-\Cme{\bar{x}}) + \left.\dpartial{\Cme{f}}{\Cme{x}} \right|_{\Cme{\bar{x}}}\cdot (\Cme{x}-\Cme{\bar{x}})^2 \ldots
\]
Como linealizamos alrededor de un entorno reducido, los términos no-lineales van a ser muy pequeños si $\Cme{x}$ es cercano a $\Cme{\bar{x}}$. Y dado que es un punto fijo, $f(\Cme{\bar{x}})=0$.

Nuestro sistema linealizado va quedar así
\[
\Delta \dot{\Cme{x}} = \left.\spartial{\Cme{f}}{\Cme{x}}\right|_\Cme{\bar{x}} \Delta x \Rightarrow \boxed{\Delta \dot{\Cme{x}} = \Mme{A} \Delta \Cme{x} }
\]

\begin{theorem}[Hartman--Grobman]
Si los autovalores de $\Mme{A}$ tienen todos parte real entonces se puede describir el sistema como lineal en un vecindario de $\Cme{\bar{x}}$.
\end{theorem}

\begin{exercise}
Determinar la estabilidad de un péndulo en su posición normal e invertida. Factor de fricción $\delta$.

\[
\ddot{\theta} = -\frac{g}{\ell} \sin(\theta) - \delta \dot{\theta}
\]

\[
\Cme{x} = \begin{Bmatrix}
\theta \\ \dot{\theta}
\end{Bmatrix}, \qquad \dot{\Cme{x}} = \begin{bmatrix}
x_2 \\
-\frac{g}{\ell}\sin(x_1)-\delta x_2
\end{bmatrix}
\]

Nuestro jacobiano es
\[
\spartial{\Cme{f}}{\Cme{x}} = \begin{bmatrix}
0 & 1 \\
-\frac{g}{\ell} \cos(x_1) & -\delta 
\end{bmatrix}
\]

Los puntos fijos son 
\[
\Cme{\bar{x}} = \begin{bmatrix}
0\\0
\end{bmatrix},
\begin{bmatrix}
\pi\\0
\end{bmatrix}
\]

Matriz del sistema péndulo en posiciones normal $d$ e invertida $u$:

\[
\Mme{A}_d = \begin{bmatrix}
0 & 1 \\
-\frac{g}{\ell} & -\delta
\end{bmatrix},\qquad
\Mme{A}_u = \begin{bmatrix}
0 & 1 \\
\frac{g}{\ell} & -\delta
\end{bmatrix}
\]

Los autovalores son
\[
\lambda_{d}=\begin{cases}
-\frac{\ell \,\delta +\sqrt{-\ell\,\left(4\,g-\ell\,{\delta }^2\right)}}{2\,\ell} \\
-\frac{\ell\,\delta -\sqrt{-\ell\,\left(4\,g-\ell\,\delta ^2\right)}}{2\,\ell} 
\end{cases}, \qquad \lambda_u = \begin{cases}
-\frac{\ell\,\delta +\sqrt{\ell\,\left(\ell\,\delta ^2+4\,g\right)}}{2\,\ell } \\
-\frac{\ell\,\delta -\sqrt{\ell\,\left(\ell\,{\delta }^2+4\,g\right)}}{2\,\ell}
\end{cases}
\]

Si se estudia el caso de un péndulo con valores $\frac{g}{\ell}=1$ y $\delta = 0,1$

\[
\lambda_{d}=
-0,05 \pm 0,9987i
, \qquad \lambda_u = \begin{cases}
-1,0512 \\
0,9512
\end{cases}
\]

Podemos ver que los autovalores del péndulo normal tienen parte real menor a cero. Esto significa que es un sistema \textbf{estable}, tiende a cero la solución. En cambio, el péndulo invertido tiene un autovalor mayor que cero, característica de un sistema \textbf{inestable}.

\end{exercise}

\chapter{Controlabilidad}

donde $\Cme{x}\in \Bbb{R}^\dimss $ y $\Cme{u}\in \Bbb{R}^\dimin$. El control `óptimo'~ para un sistema lineal se logra realimentando $-\Mme{K}\Cme{x}$, es decir:
\[
\qquad \Cme{u} = - \Mme{K} \Cme{x} \quad \text{Control `óptimo'}
\]

Entonces nuestro sistema se describe de la siguiente forma
\begin{IEEEeqnarray*}{c}
\dot{\Cme{x}} = \Mme{A} \Cme{x} + \Mme{B} \Cme{u} \\
\dot{\Cme{x}} = \Mme{A} \Cme{x} - \Mme{B}\Mme{K} \Cme{x} \\
\boxed{\dot{\Cme{x}} =\left( \Mme{A} - \Mme{B}\Mme{K}   \right)  \Cme{x} }
\end{IEEEeqnarray*}

Nuestro objetivo ahora es elegir $\Mme{K}$ para modificar las propiedades de mi sistema, como por ejemplo, la estabilidad. Si nuestro sistema es \textbf{controlable} va ser posible hacer estas modificaciones.

\begin{definition}
	A grosso modo, mi sistema es controlable si puedo elegir $\Cme{u} = -\Mme{K} \Cme{x}$ y así poner mis autovalores de $\Mme{A} - \Mme{B}\Mme{K}$ en cualquier lugar del plano complejo. Si puedo elegir la posición de mis autovalores entonces se puede controlar la evolución del \textit{state-space}  eligiendo $\Cme{u}$. 
\end{definition}

Para determinar si un sistema es controlable se construye la matriz de controlabilidad. El sistema es controlable si y soli si se verifica que la cantidad de columnas linealmente independientes sea igual a \dimss. 
\begin{definition}{Matriz de controlabilidad}
\[
\ctrb = \begin{bmatrix}
\MB & \MA \MB & \MA^2 \MB & \ldots & \MA^{\dimss-1}\MB 
\end{bmatrix}
\]	
En \Matlab~ se usa la función \texttt{ctrb} para obtener la matriz de contrabilidad
\begin{lstlisting}[caption={Obtención del rango de \(\ctrb\)}]
r = rank(ctrb(A,B));
\end{lstlisting}
\end{definition}

\section{Grados de contrabilidad y gramianes}
Mirar el rango de la matriz de contrabilidad nos da un valor binario de la contrabilidad del sistema. Hay estudios más ricos que se pueden hacer para conocer que tan controlable es el sistema.


\[
\Cx (t) = e^{\MA t} \Cx(0) + \int_0^t e^{\MA(t-\tau)} \MB \Cu(\tau) \di \tau
\]

\begin{definition}{Gramian}
\[
\MW_t = \int_0^t e^{\MA\tau} \MB \MB\tp e^{\MA\tp \tau} \di \tau  \quad \in \mathbb{R}^{\dimss\times\dimss}
\]

\end{definition}

\part{Control English}
\chapter{Introduction}

\section{Control theory}
\subsection{Basics}
Taylor expansion (Linearization) of two-variable nonlinear equation.
\[
f(x,y) = f(\overline{x},\overline{y}) + \left[ \frac{\partial f}{\partial x} (x-\overline{x}) +\frac{\partial f}{\partial y} (y-\overline{y}) \right] + \ldots
\]


Matlab command to convert state space to transfer function \verb|[num,den]=ss2tf(A,B,C,D,iu)| where \verb|iu| must be specified for systems with more than one input.








\subsection{State space}
\(\Cme{u}(t)\) is inputs vector and is of size \(\dimin\times 1\) for a given system, i.e: \( \Cme{u}(t)=\begin{bmatrix}
u_1 \\ u_2
\end{bmatrix}\) for two input system, \(\dimin=2\).

\(\Cme{y}(t) \) is the output vector of size  \(\dimout\times 1\).

\(\Cme{z}(t)\) is the \textit{disturbance input}. Only applies to dynamical systems and is of size \(\dimdisturb \times 1\)

Thus we define the \textbf{state space variables} so that system output is purely in function of current system state variables and input variables.
\[
\text{System Output} = f\left( \text{Current System State, System Input} \right)
\]

We will define \(X\) or \(\Cme{x}\) as our system state variables. There are some important aspects to note about state space variables such as
\begin{itemize}
	\item System output \(\Cme{y}(t)\) will be a function of them
	\item They change over time
	\item They are internal to the system
	\item They may include system outputs (outputs will be a function of themselves in part)
	\item Their selection is inherent part of the system design process and there are different methods of selecting them.
	\item We will assume there is a minimal quantity of state variables that is sufficient to accurately describe the system
	\item If all system inputs \(u_j\) are defined beforehand for \(t\geq t_0\) then \(\Cme{x}(t)\) defines all system states for time \(t \geq t_0\)
\end{itemize}


The mathematical representation of state space variables will be will be that of the \textbf{state vector} \(\Cme{x}(t)\) of size \(\dimss \times 1\).

To model our system we then define the equations that govern it in \textbf{state space}\footnote{State space can be thought of an \dimss -dimensional space whose axes are the state variables (\(x_1,x_2\ldots\))}. These are the \textbf{state-space equations} of the system. For a dynamic system these must include a variable that serves as memory of inputs for \(t \geq t_1\). \textit{Integrators} serve as memory devices for \textit{continuous-time} models, however, our state-space representation is discrete! This is when state-space variables come in handy: The outputs of integrators can be considered as the variables that define the internal state of the dynamic system (Ogata). 


For a system of size \(\dimin=\dimout=\dimss=1\) one has the state-space representation defined as:
\[
\dot{x}(t)=g\left[t_0,t,x(t),x(0),u(t)\right] ,\qquad y=h\left[t,x(t),u(t)\right]
\]

For a \textit{linear time-variant dynamical system} of arbitrary size it is convenient to represent it in it's linearized form

\begin{equation}
\Cme{\dot{x}} (t) = \Mme{A}(t) \Cme{x}(t) + \Mme{B}(t) \Cme{u}(t) + \Mme{E}(t) \Cme{z}(t)
\end{equation}
\begin{equation}
\Cme{y} (t) = \Mme{C}(t) \Cme{x}(t) + \Mme{D}(t) \Cme{u}(t)
\end{equation}

where 
\begin{itemize}
	\item[\(\Mme{A}_{\dimss \times\dimss }\)] System matrix. Relates future state change with current state. May be zero. Also referred to as the state matrix in some bibliographies.
	\item[\(\Mme{B}_{\dimss \times\dimin}\)] Control/input matrix. How system input influences state change. May be zero. 
	\item[\(\Mme{C}_{\dimout\times\dimss }\)] Output matrix. How system state influences system output.
	\item[\(\Mme{D}_{\dimout\times\dimin}\)] Feed forward or feedthrough matrix. How system input influences system output. Is usually zero for most physical systems.
	\item[\(\Mme{E}_{\dimss \times\dimdisturb}\)] Input matrix for disturbances. Applies only for dynamical systems.
\end{itemize}
the system is said to be \textbf{time-invariant} if the above matrices are not dependent of time. An example of a \textbf{time-variant} system is a spacecraft, whose mass changes due to fuel consumption.

One method of state space variable selection is \textbf{physical selection}. This method is based on energy accumulators. It can be said that \textit{the minimum number of state-space variables needed to model the system accurately is equal to the number of independent energy accumulators.} When state-space variable is not a energy variable it is said to be an augmented variable.

The general solution to the linear differential equation of state:
\[
\dot{\Cme{x}}(t) = \Mme{A}~ \Cme{x} (t) + \Mme{B}~ \Cme{u} (t)
\]
es
\[
\Cme{x}(t) = e^{\Mme{A}(t-t_0)}\Cme{x})0 +  \int^{t}_0 e^{\Mme{A}(t-\tau)} \Mme{B}~ \Cme{u} (\tau) \di \tau
\]

\begin{definition}[Matrix Exponential]
\begin{IEEEeqnarray}{c}
	e^{\Mme{A} t} = \eye + \sum_{i=1}^{\infty} \frac{(\Mme{A}t )^ i}{i!}
\end{IEEEeqnarray}
\end{definition}

A \Matlab~ function is provided to calculate the matrix exponential. 
\lstinputlisting[caption = {matrixexponential.m}]{code/matrixexponential.m}



\end{document}