%!TeX root=../control_rapido.tex
%!TeX spellcheck=es_ES

\chapter{Sistemas lineales}
El planteo de la evolución de un sistema de primer orden es dado por \eqref{eq:systemCtrb}. Su solución analítica requiere evaluar exponencial de la matriz $\MA$
\begin{IEEEeqnarray}{rc}
\Cme{\dot{x}} (t) = \Mme{A}(t) \Cme{x}(t) + \Mme{B}(t) \Cme{u}(t) + \Cw_{\disturb}\IEEEyessubnumber \label{eq:systemCtrb}\\
\Cme{y} (t) = \Mme{C}(t) \Cme{x}(t) + \Mme{D}(t) \Cme{u}(t) + \Cw_{\noise}\IEEEyessubnumber \label{eq:systemObsv}
\end{IEEEeqnarray}
donde $\Cw_{\disturb}=\ME(t)\Cz(t)$ son las perturbaciones sobre el sistema y $\Cw_{\noise}$ es el ruido en la medición.

\begin{figure}[htb!]
	\centering
	\begin{tikzpicture}[auto, node distance=2cm]
	% We start by placing the blocks
	\node [input, name=input] {};
	\node [sum, right of=input] (sum) {};
	\node [block, right of=sum] (controller) {Controlador};
	\node [block, right of=controller, pin={[pinstyle]above:Peturbaciones ($\ME$)},
	node distance=3cm] (system) {Sistema};
	%% We draw an edge between the controller and system block to 
	%% calculate the coordinate u. We need it to place the measurement block. 
	\draw [->] (controller) -- node[name=u] {$\Cu$} (system);
	\node [output, right of=system] (output) {};
	\coordinate [below of=u] (tmp);
	% Once the nodes are placed, connecting them is easy. 
	\draw [->] (input) -- node {$\Cme{r}$} (sum);
	\draw [->] (sum) -- node {$\Cme{e}$} (controller);
	\draw [->] (system) -- node [name=y] {$\Cy$}(output);
	\draw [->] (y) |- (tmp) -| node[pos=0.99] {$-$} 
	node [near end] {$\Cy+\Cw_{\noise}$} (sum);
	\end{tikzpicture}
	\caption{Esquema de un sistema de control genérico a lazo cerrado.}
\end{figure}

La exponencial de una matriz \eqref{eq:matrixExponential} es poco práctica para calcular con la matriz \(\Mme{A}\) y muy costosa numéricamente.

\begin{definition}{Exponencial de una matriz}
	\begin{IEEEeqnarray}{c}\label{eq:matrixExponential}
	e^{\MX} = \eye + \sum_{k=1}^{\infty} \frac{\MX^ k}{k!}
	\end{IEEEeqnarray}
donde $\eye$ es la matriz identidad.
\end{definition}

En cambio, lo que se hace en la practica es usar los autovalores y autovectores para efectuar una transformación de coordenadas de las coordenadas de $\Cx$ a las coordenadas de algún autovector donde es mas fácil escribir la exponencial de una matriz y facilita entender el sistema también.

Un \textbf{autovector} $\Cme{\xi}\in\mathbb{C}^\dimss$ cumple con la siguiente igualdad. $\lambda \in \mathbb{C}$ son los \textbf{autovalores} del sistema.
\[
\Mme{A}\Cme{\xi} = \lambda \Cme{\xi}
\]
Una forma de visualizar esto es que el producto entre la matriz $\Mme{A}$ y el autovector mantiene la dirección del autovector.

\[
\Mme{T} = \left[ \xi_1,\, \xi_2,\, \ldots\, \xi_\dimss \right]
\]

\[
\Mme{D}= \begin{bmatrix}
\lambda_1 & & & 0 \\
 & \lambda_2 & & \\
  & & \ddots & \\
 0 & & & \lambda_\dimss 
\end{bmatrix}
\]

Es posible diagonalizar el sistema siempre que no se tengan dos autovectores cuasi-paralelos o un sistema degenerado (autovectores generalizados) (entre otros casos). 

Esto nos deja escribir la relación

\[
\Mme{A} \Mme{T} = \Mme{T} \Mme{D}
\]

\[
\Cme{x} = \Mme{T}\Cme{z} \Rightarrow \Mme{T}^{-1} \Mme{A} \Mme{T} = \Mme{D}
\]

\[
\Mme{T} \dot{\Cme{z}} = \Mme{A} \Mme{T} \Cme{z} \Rightarrow \dot{\Cme{z}} = \Mme{T}^{-1} \Mme{A} \Mme{T} \Cme{z}
\]

Se obtienen entonces un sistema de ecuaciones desacoplado! El cambio de la variable $z_i$ depende de si misma
\[
\boxed{\dot{\Cme{z}} = \Mme{D} \Cme{z}}
\]

Se pueden obtener estas matrices en \Matlab~ en una linea:
\begin{lstlisting}
[T, D] = eig(A);
\end{lstlisting}

La solución del sistema va ser simple

\[
\Cme{z}(t) = \begin{bmatrix}
e^{\lambda_1 t} & & & 0 \\
 &e^{\lambda_2 t}& &  \\
 & &\ddots &  \\
  0& & & e^{\lambda_\dimss t} \\
\end{bmatrix} \cdot \Cme{z}(0)
\]

Es de interes poder mapear entre los dos espacios. Usando la expresión \( \Mme{A} = \Mme{T} \Mme{D} \Mme{T}^{-1}\) se puede simplificar la exponencial de una matriz empleando conocimientos de algebra lineal

\[
e^{\Mme{A}t} = e^{\Mme{T} \Mme{D} \Mme{T}^{-1}t} = \Mme{T} e^{\Mme{D}t} \Mme{T}^{-1}
\]
Cabe destacar que es barato calcular \( e^{\Mme{D}t}\) en términos computacionales. 

Reescribimos la solución al sistema recordando \(\Cme{x} = \Mme{T}\Cme{z}\)

\[
\Cme{x}(t) = \Mme{T}  \underbrace{e^{\Mme{D}t} \underbrace{\Mme{T}^{-1} \Cme{x}(0)}_{\Cme{z}(0)}}_{\Cme{z}(t)}
\]
La igualdad de arriba se usa para computar la evolución de $\Cme{x}$ en el tiempo aprovechando la simplicidad del cálculo de \(e^{\Mme{D}t}\). 

Que hicimos?
\begin{itemize}
	\item Descubrimos que si sabemos los autovectores/valores de \(\Mme{A}\) podemos transformar el sistema a un sistema de coordenadas donde es más facil resolver el sistema y estudiar su dinámica
\end{itemize}
 
El próximo paso es agregar la matriz de control y el vector de entrada para empezar a controlar el sistema.

