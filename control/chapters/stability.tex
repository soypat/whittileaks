%!TeX root=../control_rapido.tex
%!TeX spellcheck=es_ES

\chapter{Estabilidad y autovalores} \label{chap:estabilidadAutovalores}


Para el estudio de estabilidad podemos primero mirar a la igualdad

 \[
\Cme{x} = \Mme{T} e^{\Mme{D}t} \Mme{T}^{-1} \Cme{x}(0)
\]
Si uno de los valores diagonal de $e^{\Mme{D}t}$ se va a infinito entonces la combinación resultante que iguala a \(\Cme{x}\) también se va ir al infinito. Recordemos que $\lambda \in \mathbb{C}$

\[
\lambda = a + ib \quad \Rightarrow \quad e^{\pm \lambda t} = e^{at}\left[\cos(bt)\pm i\sin (bt)\right]
\]
Esto cuenta la siguiente historia
\begin{IEEEeqnarray}{tt}
si \(a>0\) & El sistema aumenta hasta llegar a infinito (\textbf{Inestable}) \\
si \(a<0\) & El sistema converge a cero a tiempo infinito (\textbf{Estable}) \\
\end{IEEEeqnarray}


Esto significa que tal vez comenzemos con un sistema inestable, es decir, que nuestra matriz \(A\) de la siguiente ecuación tiene autovalores con $a>0$
\[
\dot{\Cme{x}} = \Mme{A} \Cme{x}
\]

Esto se puede remediar agregando el término $\Mme{B}\Cme{u}$ de tal forma que lleve los autovalores de la zona inestable ($a>0$) a la zona estable ($a<0$).

\section{Evolución discreta}

\[
\Cme{x}_{k+1} = \Mme{\tilde{A}} \Cme{x}_k, \qquad \Cme{x}_k
 = \Cme{x}(k\Delta t)\]
 donde \(\Mme{\tilde{A}} = e^{\Mme{A} \Delta t}\). Sabiendo el vector de estado inicial podríamos calcular el estado para cualquier otro momento
 
\[
\Cme{x}_{N} = \Mme{\tilde{A}}^{N}\Cme{x}_0
\]

En coordenadas de autovector, cada vez que multiplicamos la matriz estamos elevando nuestros autovalores de $\Mme{\tilde{A}}$ a una potencia. Estos pueden agrandarse o achicarse dependiendo de su `radio'
\[
\lambda^N = R^{N}e^{i \,N\theta }
\]
si el radio $R$ es menor a uno, la magnitud va decaer a medida que pasa el tiempo. Si el radio es mayor que uno crecerá sin cota superior.

\begin{lstlisting}
[Tt, Dt] = eig(At);
aval = diag(Dt);
inestables = aval(aval>1);
\end{lstlisting}

\section{Linealizando un sistema}

\begin{enumerate}
	\item Encontramos los puntos fijos $\Cme{\bar{x}}$ tal que \(f(\Cme{\bar{x}})=\Cme{0}\)
	\item Linealizamos alrededor de $\Cme{\bar{x}}$
\end{enumerate}

Para un sistema \(2\times 2\) el jacobiano es
\[
\spartial{\Cme{f}}{\Cme{x}}=
\begin{bmatrix}
\spartial{f_1}{x_1} &\spartial{f_1}{x_2} \\
\spartial{f_2}{x_1} & \spartial{f_2}{x_2}
\end{bmatrix}
\]


\[
\dot{\Cme{x}} = f(\Cme{x}) = f(\Cme{\bar{x}}) + \left. \spartial{\Cme{f}}{\Cme{x}}\right|_{\Cme{\bar{x}}}\cdot (\Cme{x}-\Cme{\bar{x}}) + \left.\dpartial{\Cme{f}}{\Cme{x}} \right|_{\Cme{\bar{x}}}\cdot (\Cme{x}-\Cme{\bar{x}})^2 \ldots
\]
Como linealizamos alrededor de un entorno reducido, los términos no-lineales van a ser muy pequeños si $\Cme{x}$ es cercano a $\Cme{\bar{x}}$. Y dado que es un punto fijo, $f(\Cme{\bar{x}})=0$.

Además, proponemos un cambio de variable $\Delta x_i =  x_i - \bar{x}_i$ el cual vá representar el distanciamiento de $x_i$ desde el punto donde linealizamos. Nuestro sistema linealizado va quedar así
\[
\Delta \dot{\Cme{x}} = \left.\spartial{\Cme{f}}{\Cme{x}}\right|_\Cme{\bar{x}} \Delta \Cx \Rightarrow \boxed{\Delta \dot{\Cme{x}} = \Mme{A} \Delta \Cme{x} }
\]

\begin{theorem}[Hartman--Grobman]
Si los autovalores de $\Mme{A}$ tienen todos parte real entonces se puede describir el sistema como lineal en un vecindario de $\Cme{\bar{x}}$.
\end{theorem}

\begin{exercise}
Determinar la estabilidad de un péndulo en su posición normal e invertida. Factor de fricción $\delta$.

\[
\ddot{\theta} = -\frac{g}{\ell} \sin(\theta) - \delta \dot{\theta}
\]

\[
\Cme{x} = \begin{Bmatrix}
\theta \\ \dot{\theta}
\end{Bmatrix}, \qquad \dot{\Cme{x}} = \begin{bmatrix}
x_2 \\
-\frac{g}{\ell}\sin(x_1)-\delta x_2
\end{bmatrix}
\]

Nuestro jacobiano es
\[
\spartial{\Cme{f}}{\Cme{x}} = \begin{bmatrix}
0 & 1 \\
-\frac{g}{\ell} \cos(x_1) & -\delta 
\end{bmatrix}
\]

Los puntos fijos son 
\[
\Cme{\bar{x}} = \begin{bmatrix}
0\\0
\end{bmatrix},
\begin{bmatrix}
\pi\\0
\end{bmatrix}
\]

Matriz del sistema péndulo en posiciones normal $d$ e invertida $u$:

\[
\Mme{A}_d = \begin{bmatrix}
0 & 1 \\
-\frac{g}{\ell} & -\delta
\end{bmatrix},\qquad
\Mme{A}_u = \begin{bmatrix}
0 & 1 \\
\frac{g}{\ell} & -\delta
\end{bmatrix}
\]

Los autovalores son
\[
\lambda_{d}=\begin{cases}
-\frac{\ell \,\delta +\sqrt{-\ell\,\left(4\,g-\ell\,{\delta }^2\right)}}{2\,\ell} \\
-\frac{\ell\,\delta -\sqrt{-\ell\,\left(4\,g-\ell\,\delta ^2\right)}}{2\,\ell} 
\end{cases}, \qquad \lambda_u = \begin{cases}
-\frac{\ell\,\delta +\sqrt{\ell\,\left(\ell\,\delta ^2+4\,g\right)}}{2\,\ell } \\
-\frac{\ell\,\delta -\sqrt{\ell\,\left(\ell\,{\delta }^2+4\,g\right)}}{2\,\ell}
\end{cases}
\]

Si se estudia el caso de un péndulo con valores $\frac{g}{\ell}=1$ y $\delta = 0,1$

\[
\lambda_{d}=
-0,05 \pm 0,9987i
, \qquad \lambda_u = \begin{cases}
-1,0512 \\
0,9512
\end{cases}
\]

Podemos ver que los autovalores del péndulo normal tienen parte real menor a cero. Esto significa que es un sistema \textbf{estable}, tiende a cero la solución. En cambio, el péndulo invertido tiene un autovalor mayor que cero, característica de un sistema \textbf{inestable}.

\end{exercise}