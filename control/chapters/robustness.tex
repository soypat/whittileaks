%!TeX root=../control_rapido.tex
%!TeX spellcheck=es_ES

\chapter{Robustez y el dominio frecuencia}
Un paper escrito por John C. Doyle en 1978 en su paper \textit{Guaranteed margins for LQG regulator} determino por contraejemplo que no existen márgenes estables para un sistema con controlador LQG y que una peturbación arbitrariamente chica puede ocasionar una pérdida de control completa \citep{doyle1978guaranteed}. Esto se debe a que el sistema LQG es vulnerable a incertidumbres del sistema descrito por las ecuaciones \eqref{eq:systemCtrb}, \eqref{eq:systemObsv} y por \textit{time delay} de las observaciones. Entonces a pesar de poder posicionar los autovalores con un regulador cuadrático para aumentar la estabilidad del sistema, no podemos asegurar que nuestro sistema sea \textbf{robusto}.

Aquí es donde aparecen dos conceptos duales llamados \textbf{robustez} y \textbf{performance}. 

\begin{description}
	\item[Performance] Los autovalores del lazo cerrado y que tan rápido estimar $\hat{\Cx}$ son medidas de performance
	\item[Robustez] Medida de que tan sensible es el sistema a perturbaciones, incertidumbres del sistema y retardo de mediciones
\end{description}

En las próximas secciones se verá como determinar si un sistema es robusto y como diseñar un sistema de control para que tenga buena robustez y performance.

\section{Representaciones equivalente de sistemas lineales}

\begin{figure}[htb!]
	\centering
\begin{tikzpicture}[auto, node distance=2cm]
	\node[block,label=above:Sistema] (sys) {$G(s)$};
	\node[input,left of=sys] (in) {};
	\node[output,right of=sys] (out) {};
	\draw[->] (in)  --node[name=y] {$\Cy$} (sys);
	\draw[->] (sys)  --node[name=y] {$\Cu$} (out);
\end{tikzpicture}
\caption{Se puede relacionar la salida y entrada de un sistema con una función transferencia $G$.}
\end{figure}

Existen 3 representaciones equivalentes para sistemas lineales
\begin{description}
	\item[Espacio de estados]  La vista hasta ahora. Ecuaciones \eqref{eq:systemCtrb}, \eqref{eq:systemObsv}.
	\item[Dominio de frecuencia] $G(s) = \MC (s\eye -\MA)^{-1}\MB$
	\item[Dominio de tiempo	] Respuesta al impulso $h(t)$. Integral de convolución: $\Cy(t) = \int_0^t h(t-\tau)\Cu(t) \diff \tau$
\end{description}


\begin{figure}[htb!]
	\centering
\resizebox{0.49\textwidth}{!}{
\begin{tikzpicture}
\begin{axis}[
	title={Input $\Cu$},
%	xlabel=$x$,
	ylabel=$\sin(\omega t)$,
	grid=major,
	xmin=0,xmax=2*\numpi,
	ymin=-1,ymax=1,
	xtick={-0,\numpi,2*\numpi},
]
\addplot[domain=0:2*\numpi,samples=50] { sin((x)r) };
\end{axis}
\end{tikzpicture}
}
\resizebox{0.49\textwidth}{!}{
	\begin{tikzpicture}
	\begin{axis}[
	title={Salida $\Cy$},
	grid=major,
%	xlabel=$x$,
	ylabel=$A\sin(\omega t +\varphi)$,
	xmin=0,xmax=2*\numpi,
	ymin=-1,ymax=1,
	xtick={-0,\numpi,2*\numpi},
	]
	% r para convertir radianes a grados que es lo que toma sin(x)
	\addplot[domain=0:2*\numpi,samples=50] { 0.45*sin((x-.7)r) }; 
	\end{axis}
	\end{tikzpicture}
}
\caption{Para un sistema lineal una entrada $\Cu$ sinusoidal va generar una salida $\Cy$ sinusoidal de la misma frecuencia pero con un cambio de amplitud y desfasaje.}
\end{figure}

La función de transferencia $G$ es una función compleja donde la variable de entrada es la frecuencia del input. De aquí podemos obtener la respuesta del sistema de amplitud y desfasaje ante un input $\Cu$ de frecuencia $\omega$.
\begin{itemize}
\item $|G(i\omega)|=A$
\item $\angle G(i\omega) = \varphi$
\end{itemize} 

La función de transferencia nos dará una buena idea de si un sistema es robusto 

\section{Ejemplo de respuesta en frecuencia}
\begin{figure}[htb!]
	\centering
	\tikzset{boxed/.style={draw,outer sep=0pt,thick}}
	\begin{tikzpicture}
	
	\begin{scope}[xshift=7cm]
	\node (M) [boxed,minimum width=1cm, minimum height=2.5cm] {$m$};
	
	\node (ground) [ground,anchor=north,yshift=-0.25cm,minimum width=1.5cm] at (M.south) {};
	\draw (ground.north east) -- (ground.north west);
	\draw [thick] (M.south west) ++ (0.2cm,-0.125cm) circle (0.125cm)  (M.south east) ++ (-0.2cm,-0.125cm) circle (0.125cm);
	
	\node (wall) [ground, rotate=-90, minimum width=3cm,yshift=-3cm] {};
	\draw (wall.north east) -- (wall.north west);
	
	\draw [spring] (wall.170) -- node[above] {$k$} ($(M.north west)!(wall.170)!(M.south west)$);
	\draw [damper] (wall.10) -- node[above,pos=.4] {$c$} ($(M.north west)!(wall.10)!(M.south west)$);
	
	\draw [-latex,ultra thick] (M.east) ++ (0.2cm,0) -- node[above] {$u$}  +(1cm,0);
	\end{scope}
	\end{tikzpicture}
	\caption{Sistema masa-resorte-amortiguador.}
	\label{fig:dampedmasssystem}
\end{figure}

La ecuación que describe el movimiento del cuerpo en la figura \ref{fig:dampedmasssystem}
\[
m\ddot{x} + c\dot{x} + kx = u
\]
donde $u$ son las fuerzas externas actuantes. La ecuación se puede modelar como un sistema lineal si introducimos $x$ y $\dot{x}$ como variables de estado. Se puede también tomar la transformada de Laplace, de hecho, es lo vamos hacer ahora. Suponiendo que $m=k=1,\,c=\frac{1}{2}$ y despreciando los términos transitorios $\xbar(0)$


\[
\lapc{\ddot{x} +\tfrac{1}{2}\dot{x} +x} = s^2\xbar(s) + \tfrac{1}{2}s \xbar(s) + \xbar(s) = \ubar(s)
\]
donde $\xbar(s) = \lapc{x(t)}$.

Es de interés obtener una relación de input sobre respuesta, la cual va ser la transferencia del sistema. Si agrupamos la función $\xbar$ y despejamos obtenemos la función transferencia del sistema. Esta relaciona la variable de entrada $\ubar$ al estado del sistema $\xbar$
\[
G(s)= \frac{\xbar}{\ubar} = \frac{1}{s^2 + \tfrac{1}{2} s + 1}
\]

Como vimos anteriormente, con esta fnción podemos calcular la amplitud y desfasaje del sistema para una frecuencia $\omega$. Podemos representar la respuesta del sistema en un gráfico llamado Bode para entender mejor el sistema
\begin{figure}[htb!]
	\centering
	% BODE
		\begin{tikzpicture}[trim axis right]
		\begin{semilogxaxis}[bodeamp]
		\addplot table[x=x, y=bode, col sep=comma, mark=none] {plots/bd_dampedmass.csv};
		\end{semilogxaxis}	
		\end{tikzpicture}
		\begin{tikzpicture}[trim axis right]
		\begin{semilogxaxis}[bodephase]
		\addplot table[x=x, y=phase, col sep=comma, mark=none, color=red] {plots/bd_dampedmass.csv};
		\end{semilogxaxis}	
		\end{tikzpicture}
		\caption{Bode para la función de transferencia $G(s) = \frac{1}{s^2 + \frac{1}{2}s + 1}$ }
		\label{plt:bodedampedmasssystem}
\end{figure}

Del Bode podemos obtener
\begin{itemize}
	\item Las frecuencias resonantes o naturales del sistema. En el gráfico \ref{plt:bodedampedmasssystem} observamos una frecuencia de resonancia para $\omega = 0,9$ radianes por segundo
	\item La frecuencia de entrada $\omega$ para la cual el sistema empieza a decaer en amplitud (frecuencia de corte)
\end{itemize}


\section{Transformada Laplace de un sistema lineal}
A continuación se obtiene la transformada de la ecuación de un sistema lineal
\begin{IEEEeqnarray*}{c}
\lapc{\dot{\Cx} =\MA \Cx + \MB \Cu } \\
 s \bar{\Cx}(s) - \Cx(0) = \MA \bar{\Cx}(s) + \MB \bar{\Cu}(s)\\
 (s\eye - \MA) \bar{\Cx} = \MB \bar{\Cu} + \Cx(0) \\
 \bar{\Cx} = \left(s\eye - \MA\right)^{-1}\MB \bar{\Cu} + \left(s\eye - \MA\right)^{-1}\Cx(0)
\end{IEEEeqnarray*}
el objeto $\left(s\eye - \MA\right)$ es un polinomio idéntico al polinomio característica de la matriz $\MA$. Es decir, tiene raíces en los autovalores de $\MA$. Debido a esto la función transferencia de un sistema lineal se va a infinito cuando el input es un autovalor de $\MA$.

La ecuación \eqref{eq:systemObsv}
\begin{IEEEeqnarray*}{c}
\bar{\Cy}(s)=\MC \bar{\Cx}(s)
\end{IEEEeqnarray*}

Reemplazando la transformada de \eqref{eq:systemCtrb} teniendo en cuenta que el término transitorio muere con el tiempo $\Cx(0)=0$, se tiene

\begin{IEEEeqnarray}{c}
\boxed{
\bar{\Cy} =\underbrace{ \MC \left(s\eye - \MA\right)^{-1}\MB}_{=G(s)} \bar{\Cu} \label{eq:linearsysTransform}
}
\end{IEEEeqnarray}

La ecuación \eqref{eq:linearsysTransform} me dice que puedo obtener la salida en dominio \textbf{continuo} con solo anti-transformar el producto de la función transferencia con la transformada de la entrada.  

Es interesante notar que la anti transformada de la función transferencia nos dá la respuesta del sistema ante un impulso. 
\[
\Cy(t) = \ilapc{G(s)} \quad \text{donde} \quad \Cu(t)=\vec{\delta}(t)
\]
dicho de otra manera, la transformada de la respuesta ante el impulso es la función transferencia. Se puede aproximar así la función transferencia a partir de ensayos hechos en laboratorios donde se mide la respuesta de un sistema.











