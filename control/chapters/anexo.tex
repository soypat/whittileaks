%!TeX root = ../control_rapido.tex
%!TeX spellcheck = es_ES
\chapter{Anexo}

\section{Métodos numéricos para resolución de SEDO}\label{sec:metodosNumericos}

\subsection{Runge--Kutta orden 4 multivariable} 
El método de Runge Kutta tiene su ventajas sobre otros métodos numéricos por ser simple de programar y tener bajo error ($\error = O(\dt^4)$), donde $\dt$ es el paso elegido. El primer paso consiste en describir el sistema de $\dimss$ variables de forma similar a como hacemos con nuestros sistemas lineales,

\begin{IEEEeqnarray}{c}
\Cx^{\prime} = \Cf (\Cx)
\end{IEEEeqnarray}
que es lo mismo que escribir
\begin{IEEEeqnarray*}{lc}
x_1^{\prime} &= f_1(x_1,x_2,\ldots,x_{\dimss}) \\
x_2^{\prime} &= f_2(x_1,x_2,\ldots,x_{\dimss}) \\
&\vdots \\
x_\dimss^{\prime} &= f_\dimss(x_1,x_2,\ldots,x_{\dimss}) \\
\end{IEEEeqnarray*}

Ahora se definen dos instantes de tiempo $\ts{\Cx}{k}$ de valores conocidos y $\ts{\Cx}{k+1}$ desconocido. Para calcular los valores de $\ts{\Cx}{k+1}$ el algoritmo Runge--Kutta hace lo siguiente

\begin{IEEEeqnarray}{c}
\begin{cases}
\ts{\Ca}{k} &= \Cf(\ts{\Cx}{k}) \\
\ts{\Cb}{k} &= \Cf(\ts{\Cx}{k} + \tfrac{h}{2} \ts{\Ca}{k} ) \\
\ts{\Cc}{k} &= \Cf(\ts{\Cx}{k} + \tfrac{h}{2} \ts{\Cb}{k} ) \\
\ts{\Cd}{k} &= \Cf(\ts{\Cx}{k} + h \ts{\Cc}{k} ) \\
\end{cases} \\
\ts{\Cx}{k+1} = \ts{\Cx}{k} + \frac{h}{6} \left( \ts{\Ca}{k} + 2\ts{\Cb}{k} + 2\ts{\Cc}{k} + \ts{\Cd}{k}\right)
\end{IEEEeqnarray}

El nuevo vector $\ts{\Cx}{k+1}$ te da el estado del sistema después de un pequeño paso $h$.