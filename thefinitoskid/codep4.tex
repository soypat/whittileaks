 \clearpage
 \subsection*{Código \Matlab{} Finitos II}
    %%%%------------------------------------------------------------------CODE
    \subsubsection*{Código de Placas}
    Se va tratar con un método para el almacenado eficiente de las funciones de forma usando structures de \Matlab{}. Esto acorta el tiempo de corrido en un factor mayor a 100 para problemas medianos/grandes. Código en anexo.

    \begin{code}[Funciones de Forma Mindlin Q4]
    \begin{verbatim}

clear dN N dNaux X dNxx dNyy dNxy 
syms ksi eta real
X = [1 ksi eta ksi.*eta];
Xdx = diff(X,ksi);
Xdy = diff(X,eta);
uNod = [-1 -1;1 -1;1 1;-1 1];
Nnodporelem = length(uNod);
Ndofpornod = 3; %Placas Mindlin
Ndofporelem = Nnodporelem*Ndofpornod;
A = zeros(Nnodporelem,length(X));
for i=1:Nnodporelem
  ksi=uNod(i,1); eta = uNod(i,2);
  A(i,:) = double(subs(X));
end

syms ksi eta real
shapefuns = X*inv(A);
N = shapefuns;
dNx = diff(shapefuns,ksi);
dNy = diff(shapefuns,eta);
B =  sym('noImporta',[5,Ndofporelem]);
for i = 1:Nnodporelem
  B(:,(i*3-2):(i*3)) = [0 dNx(i) 0;0 0 dNy(i);0 dNy(i) dNx(i);
	-dNx(i) N(i) 0;-dNy(i) 0 N(i)];
end
dN = [dNx;dNy];
    \end{verbatim}
    \end{code}

\begin{code}[Dofinitions y elemDof]
	\begin{verbatim}

[Nelem, Nnodporelem]= size(elementos);  
[Nnod, Ndim] = size(nodos); % Numero de nodos, Numero de dimensiones del problema (es 2-D)

Ndofpornod = 3; %Para placa Kirchoff
dof = Nnod*Ndofpornod;
DOF = (1:dof)'; %vector columna

n2d = @(nodo) [nodo*3-2, nodo*3-1, nodo*3]; % Función Node a DOF. Obtiene indices de dof de un nodo. Si hay mas/menos de 3 dof por nodo entonces cambia
elemDof = zeros(Nelem,Ndofpornod*Nnodporelem);
for e = 1:Nelem
for n = 1:Nnodporelem
  elemDof(e,n2d(n)) = n2d(elementos(e,n));
end
end
	\end{verbatim}
\end{code}

\begin{code}[Constitutiva]
	\begin{verbatim}
	
F = E*t^3/(12*(1 - NU^2)); %Rigidez ante la flexion
G = E/(2+2*NU); % Rigidez a la torsion
Cb = [F NU*F 0;
NU*F F 0 
0 0 (1-NU)*F/2];
Cs = 5/6*[G*t 0;0 G*t];
C = blkdiag(Cb,Cs);
	\end{verbatim}
\end{code}

\begin{code}[Acople matriz rigidez]
	\begin{verbatim}
	
Kg = sparse(dof,dof);
for e = 1:Nelem
  Kb = zeros(Ndofporelem);
  dofIndex = elemDof(e,:);
  storeTo = elemDof(e,:);
  nodesEle = nodos(elementos(e,:),:);
  Bb = zeros(3,Ndofporelem);
  Bs = zeros(2,Ndofporelem);
  for ipg = 1:npg2
    ksi = upg2(ipg,1); eta = upg2(ipg,2);
    jac = dNs2{ipg}*nodesEle;
    dNxy = jac\dNs2{ipg};   % dNxy = inv(jac)*dN
    for i = 1:Nnodporelem 
     Bb(:,(i*3-2):(i*3)) = [0 dNxy(1,i) 0;0 0 dNxy(2,i);0 dNxy(2,i) dNxy(1,i)];
    end
    Kb = Kb + Bb'*Cb*Bb*wpg2(ipg)*det(jac);
  end
  jac = dNs1*nodesEle;
  dNxy = jac\dNs1;   % dNxy = inv(jac)*dN
  for i = 1:Nnodporelem 
    Bs(:,(i*3-2):(i*3)) = [-dNxy(1,i) Ns1(i) 0;-dNxy(2,i) 0 Ns1(i)];
  end
  Ks = Bs'*Cs*Bs*wpg1*det(jac);

  Kg(storeTo,storeTo) = Kg(storeTo,storeTo) + Kb+ Ks;
end
	\end{verbatim}
\end{code}

\begin{code}[Cargas]
	\begin{verbatim}
	
p0 = -0.05e6; %MPa
R = zeros(dof,1);
for e = 1:Nelem
   storeTo = elemDof(e,1:3:end);
   nodesEle = nodos(elementos(e,:),:);
   for ipg = 1:npg2
     jac = dNs2{ipg}*nodesEle;
     Q = Ns2{ipg}'*p0*wpg2(ipg)*det(jac);
     R(storeTo)=R(storeTo)+Q;
  end
end
	\end{verbatim}
\end{code}

\begin{code}[Recuperación de tensiones]
	\begin{verbatim}
	
Sb = zeros(Nnodporelem,3,Nelem);
Ss = zeros(Nnodporelem,2,Nelem);
% FuncFormas en nodos
Nsn = cell(Nnodporelem,1);
dNsn = cell(Nnodporelem,1);
for inod = 1:Nnodporelem
  ksi = uNod(inod,1); eta = uNod(inod,2);
  dNsn{inod} = double(subs(dN));
  Nsn{inod} = double(subs(N));
end

for e = 1:Nelem
  storeTo = elemDof(e,:);
  nodesEle = nodos(elementos(e,:),:);
  Bb = zeros(3,Ndofporelem);
  Bs = zeros(2,Ndofporelem);
  for  inod = 1:Nnodporelem
    ksi = uNod(inod,1); eta = uNod(inod,2);
    Nder=dNsn{inod};
    jac = Nder*nodesEle;
    dNxy = jac\Nder;  % dNxy = inv(jac)*dN     
    for i = 1:Nnodporelem % Armo matriz B de bending
        Bb(:,(i*3-2):(i*3)) = [0 dNxy(1,i) 0
        0 0 dNxy(2,i)
        0 dNxy(2,i) dNxy(1,i)];
    end
    for i = 1:Nnodporelem 
      Bs(:,(i*3-2):(i*3)) = [-dNxy(1,i) Nsn{inod}(i) 0
     -dNxy(2,i) 0 Nsn{inod}(i)]; % Ver cook 15.3-3
    end
    Sb(inod,:,e) = Cb*(Bb*D(storeTo));
    Ss(inod,:,e) = Cs*(Bs*D(storeTo));
  end
end
	\end{verbatim}
\end{code}

\begin{code}[Plot Von Mises y desplazamientos]
	\begin{verbatim}
	
Svm1 = (((Sb(:,1,:)-Sb(:,2,:)).^2+(Sb(:,3,:).^2 + ...
   Ss(:,1,:).^2+Ss(:,2,:).^2) )./2).^(.5);
bandplot(elementos,nodos,squeeze(Svm1)',[],'k')

Dz = zeros(divisionesx,divisionesy); % Matriz superficie
xv =[];
yv = [];
for n=1:Nnod
   xv = [xv nodos(n,1)];yv = [yv nodos(n,2)];
   Dz(n) = D(n*3-2);
end
Dz=reshape(Dz,[],1)';
scatter3(xv,yv,Dz)
	\end{verbatim}
\end{code}

\subsubsection{Vibraciones}
La rutina para acoplar a la matriz de rigidez y la matriz es la misma.
\begin{code}[Matriz de masa  viga 2 dof]
	\begin{verbatim}
	
Me= m/420 * [156    22*Le     54      -13*Le
   22*Le    4*Le^2    13*Le    -3*Le^2
   54      13*Le     156     -22*Le
  -13*Le   -3*Le^2   -22*Le   4*Le^2];
	\end{verbatim}
\end{code}


\begin{code}[El problema de autovalores]
	\begin{verbatim}
	
A = Mg(isFree,isFree)\Kg(isFree,isFree);
[Vr, eigVal]=eig(A);
V=zeros(dof);
V(isFree,isFree)=Vr;
dofred = size(Vr,2);
	\end{verbatim}
\end{code}


\begin{code}[5 formas de obtener $\COmega $]
	\begin{verbatim}
	
Phi = zeros(dofred,dofred);
omega = zeros(dofred,1);
omegaray = zeros(dofred,1);
ray = zeros(dofred,1);
for i = 1:dofred
   Dbi=Db(:,i);
   Phi(:,i) = Dbi/sqrt(Dbi'*Mr*Dbi);
   Phii = Phi(:,i); 
   omegaray(i) = sqrt((Dbi' * Kg(isFree,isFree) *Dbi)/aux);% Cook (11.4-13)
   ray(i)= sqrt((Phii' * Kg(isFree,isFree) *Phii)/(Phii' *...
        Mg(isFree,isFree) *Phii)); %Cook (11.7-1)b
   omega(i) = sqrt(Phii'* Kg(isFree,isFree)*Phii); % Idem
end
ESP  = Phi' * Kg(isFree,isFree) *Phi;
omegaespectral = sqrt(diag(ESP));
omegaeig = sqrt(diag(eigVal)); %y una quinta para que tengas
	\end{verbatim}
\end{code}

\begin{code}[Rutina barrido de frecuencia]
	\begin{verbatim}
	
input_ksi = 0.05:0.05:0.3;
Nmodos = 3;
input_omega = 1:1:10000;
for k = 1:Nksi
   ksi = input_ksi(k);
   for f = 1:Nfrec
      Z=zeros(dofred,1);
      for i=1:dofred
         chi = input_omega(f)/omega(i);
         z(i) = (Rmodalenfrecuencia(i,f)/omega(i)^2)/ ...
              sqrt((1-chi^2)^2+(2*input_ksi(k)*chi)^2);
      end
      Amp(f,k) = abs(sum(z));
   end
   semilogy(input_omega/(2*pi),Amp(:,k))% HZ
   hold on
end
	\end{verbatim}
\end{code}