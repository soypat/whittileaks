\part{Conceptos de modelado y de uso de software}
\section{Proceso de modelado}
El primer paso es la generacion del modelo

Primer step, el modelo matematico. se representan las caracteristicas esenciales y lo
que se considera importante o relacionado al problema tratado, que pueden ser los detalles.

No necesariamente se tenga en cuenta el metodo de solucion al momento de modelado matematico
pero si sirve tener conocimiento de como funciona un FEA para saber cuanto detalle incluir.

Es importante conocer la teoria que se maneja y si aplica. Teoria de vigas supone material homogeneo y 
deflexiones pequeñas. Si se va de rango entonces te vas del rango donde es valida tu solucion.

Se recomienda que se empiece con un plan flexible que puede ser modificado a medida que se aprende mas del problema
que empezar de una con FEA. Planning requiere de 
*que resultados son los deseados, que se busca?
* Como serán verificados los resultados
* Hay una tendencia de aceptar los resultados del FEA de una por el tiempo que se invirtio en ellos,
esto se contraresta teniendo en mano los resultados aproximados ANTES de comenzar el FEA. 

*No se suelen usar elementos de deformacion constante porque no muestran la variacion lineal de la 
deformacion que aparece tan seguido en problemas con flexion y porque sufren de problemas numericos. 
(shear locking)

Thin walled sections!
Element Shape / Distribution!