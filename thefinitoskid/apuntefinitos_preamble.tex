\documentclass[11pt, a4paper,titlepage,openany]{book}
\usepackage[spanish, es-tabla, es-nodecimaldot]{babel}
\selectlanguage{spanish}
%% NO AGREGAR PAQUETES ANTES DE BABEL!


%%%%%%%%%%%%%%%%%
% DOCUMENT FONT %
%%%%%%%%%%%%%%%%%
\usepackage[T1]{fontenc}
\usepackage[utopia]{mathdesign}
%\newcommand{\gtrsim}{\text{ mayor o parecido a }}

%%%%%%%%%%%%%%%
% PAGE FORMAT %
%%%%%%%%%%%%%%%
\usepackage[a4paper,top=1.5cm,bottom=1.5cm,left=2cm,right=2cm]{geometry}
\setlength{\parindent}{10pt} %Sangria al comienzo de un paragraph
%%----------------------------------------------------

%%%%%%%%%%%%%%%%%%%
% GENERAL PURPOSE %
%%%%%%%%%%%%%%%%%%%
\usepackage[
hyperfootnotes=false,
urlcolor=blue,
colorlinks=true,
citecolor=red]{hyperref}
%\usepackage{units} % permite usar nicefrac
\usepackage{graphicx,xcolor,caption,subcaption,multirow}
\usepackage{amsmath,array}

%%%%%%%%%%%%%%%%
% BIBLIOGRAPHY %
%%%%%%%%%%%%%%%%
\usepackage{natbib}
\bibliographystyle{unsrtnat}

%%%%%%%%%%%%%
% FOOTNOTES %
%%%%%%%%%%%%%
\usepackage[symbol,perpage]{footmisc}
\renewcommand*{\thefootnote}{\fnsymbol{footnote}}

\let\oldpart\part
\renewcommand{\part}[1]{ \chapter{#1}}
