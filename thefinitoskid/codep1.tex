\bgroup

\section{Parcial 1}

\subsection{Initialization}
\begin{code}
\begin{verbatim}

nod=[0 0;0 L;0 2*L;-L L;-L 2*L;-1.5*L 0;-1.5*L L;-1.5*L 2*L];
N=length(nod);
R=zeros(length(nod)*3,1); %hinges! 
elenod=[8 5;5 3;7 4;4 2;7 6;5 4;5 2;4 1;3 2;2 1];
CB=false(N*3,1); %add hinges
nodo=@(n) [n*3-2 n*3-1 n*3]; %gr8 func
CB(nodo(8))=~~[1 0 1]';
\end{verbatim}
\end{code}

\subsection{Variable Generation}
%\includegraphics[width=\textwidth]{1zunch.pdf}

\begin{code}
\begin{verbatim}

function [nodeDofs,elenod,nudos,Le,phide,Ndof] = ...
gennodedofs(nod,elenod,eletype)
%GENNODEDOFS genera matriz nodeDofs
[Ne,~]=size(elenod);
[N,~]=size(nod);
hinges=length(eletype(eletype==4 | eletype==44 | eletype==3 | eletype==33));
Le=zeros(Ne,1);
phide=zeros(Ne,1);
Ndof=N*3+hinges;
%HINGES (ROTULAS)
hingedlist=zeros(1,hinges);
j=0;
Nt=N+hinges;
nodeDofs=zeros(Nt,3);
nudos=zeros(Nt,1);
for i=1:N
    nodeDofs(i,:)=[i*3-2 i*3-1 i*3];
end
%Termino de llenar nodeDofs y lleno nudos
for i = 1:Ne
    nodestart=elenod(i,1);
    nodeend=elenod(i,2);
    lx=nod(nodeend,1)-nod(nodestart,1);
    ly=nod(nodeend,2)-nod(nodestart,2);
    Le(i)=sqrt(lx^2+ly^2);
    phide(i)=atan2d(ly,lx);%angulo en degrees
    switch eletype(i)
        case {2,22}
            nudos(nodestart)=nudos(nodestart)+1;
            nudos(nodeend)=nudos(nodeend)+1;
        case {3,33}
            j=j+1;
            nudos(nodeend)=nudos(nodeend)+1;
            hingedlist(j)=i; 
            elenod(i,1)=N+j;
            nodeDofs(N+j,:)=[nodestart*3-2 nodestart*3-1 N*3+j];
        case {4,44}
            j=j+1;
            nudos(nodestart)=nudos(nodestart)+1;
            hingedlist(j)=i; 
            elenod(i,2)=N+j;
            nodeDofs(N+j,:)=[nodeend*3-2 nodeend*3-1 N*3+j];
    end
end
\end{verbatim}% las mierditas de los apostrofes joden mucho.
\end{code}

\subsection{Assembly}
\subsubsection*{Assemble bar}
\begin{code}
\begin{verbatim}

klocal=Kb(Ee(i),Ae(i),Le(i));
if eletype(i)==11
    klocal=klocal/2;
end
T=Tbu(phide(i));
kbarrarotada=T*klocal*T';
index=[nodeDofs(elenod(i,1),[1 2]) nodeDofs(elenod(i,2),[1 2])];
kG(index,index)=kG(index,index)+kbarrarotada;
klocalrotada=kbarrarotada;
index=[nodeDofs(elenod(i,1),[1 2]) nodeDofs(elenod(i,2),[1 2])];
elementos(i,[1 2 3 4 5 6])=...
    [index(1:2) nodeDofs(elenod(i,1),3) index(3:4) nodeDofs(elenod(i,2),3)];
if nudos(elenod(i,1))==0
    CB(index(2)+1)=true;
end
if nudos(elenod(i,2))==0
    CB(index(4)+1)=true;
end
\end{verbatim}
\end{code}
\subsubsection{Assemble beam}
\begin{code}
\begin{verbatim}

klocal=Kv(Ee(i),Ae(i),Ie(i),Le(i));
if eletype(i)>11
    klocal=klocal/2;
end
T=Tvu(phide(i));
klocalrotada=T'*klocal*T;
index=[nodeDofs(elenod(i,1),:) nodeDofs(elenod(i,2),:)];
elementos(i,:)=index;
kG(index,index)=kG(index,index)+klocalrotada;
\end{verbatim}
\end{code}

\subsubsection*{3-D Beam with taxed displacements}

\begin{code}
\begin{verbatim}

nod=[4 4 3; 0 4 3...
enod=[2 1;2 1...
fnod=@(n) [n*6-5 n*6-4 n*6-3 n*6-2 n*6-1 n*6];
[Ne,~]=size(enod);
[N,~]=size(nod);
Le=zeros(Ne,1);
kG=zeros(6*N);
for i=1:Ne
    ns=enod(i,1);
    ne=enod(i,2);
    lx=nod(ne,1)-nod(ns,1);
    ly=nod(ne,2)-nod(ns,2);
    lz=nod(ne,3)-nod(ns,3);
    Le(i)=sqrt(lx^2+ly^2+lz^2);%Long.
    px=[lx ly lz]/Le(i);
    vy=[0 pi 5];
    pz=cross(px,vy)/norm(cross(px,vy));
    py=cross(pz,px);
    C=[px',py',pz'];%Para barras no importa orientacion azimuth (acimutal)
    T=blkdiag(C,C,C,C);
    Ts=[Ts T];
    klocal=Kvuw(Ee(i),Ae(i),0,0,0,0,Le(i));
    krotada=T'*klocal*T;
    kG([fnod(ns) fnod(ne)],[fnod(ns) fnod(ne)])=...
      kG([fnod(ns) fnod(ne)],[fnod(ns) fnod(ne)])+krotada;
end
CB=false(6*N,1);
CB([fnod(2) fnod(3) fnod(4) fnod(5)])=true;
CB(fnod(1))=~~[0 0 0 1 1 1];
Kr=kG(~CB,~CB);
R=zeros(6*N);
R(fnod(1))=[0 -10e3 0 0 0 0];
F=R(~CB);
U=Kr\F;
%Ahora imponemos desplazamiento inicial--------
v1=-1.517731e-04;
Dc=zeros(N*6,1);
Dc(fnod(1))=[0 v1 0 0 0 0];
CB(fnod(1))=~~[0 1 0 1 1 1];
R=zeros(N*6,1); %si hay fuerzas ademas de la desconocida las agregamos
Dx=kG(~CB,~CB)\(R(~CB)-kG(~CB,CB)*Dc(CB));
Rx=kG(CB,CB)*Dc(CB)+kG(CB,~CB)*Dx;
R(CB)=Rx
\end{verbatim}
\end{code}

\begin{code}
\begin{verbatim}

klocal=Kb(Ee(i),Ae(i),Le(i));
if eletype(i)==11
    klocal=klocal/2;
end
T=Tbu(phide(i));
kbarrarotada=T*klocal*T';
index=[nodeDofs(elenod(i,1),[1 2]) nodeDofs(elenod(i,2),[1 2])];
kG(index,index)=kG(index,index)+kbarrarotada;
klocalrotada=kbarrarotada;
index=[nodeDofs(elenod(i,1),[1 2]) nodeDofs(elenod(i,2),[1 2])];
elementos(i,[1 2 3 4 5 6])=...
    [index(1:2) nodeDofs(elenod(i,1),3) index(3:4) nodeDofs(elenod(i,2),3)];
if nudos(elenod(i,1))==0
    CB(index(2)+1)=true;
end
if nudos(elenod(i,2))==0
    CB(index(4)+1)=true;
end
\end{verbatim}
\end{code}
\subsubsection*{Assemble beam}
\begin{code}
\begin{verbatim}

klocal=Kv(Ee(i),Ae(i),Ie(i),Le(i));
if eletype(i)>11
    klocal=klocal/2;
end
T=Tvu(phide(i));
klocalrotada=T'*klocal*T;
index=[nodeDofs(elenod(i,1),:) nodeDofs(elenod(i,2),:)];
elementos(i,:)=index;
kG(index,index)=kG(index,index)+klocalrotada;
\end{verbatim}
\end{code}
\clearpage
\subsubsection*{Matrixes}

\begin{code}
\begin{verbatim}

Tb=[cosd(phi) 0;
    sind(phi) 0;
    0 cosd(phi);
    0 sind(phi)];
    
T=[cosd(phi) sind(phi) 0 0 0 0;
    -sind(phi) cosd(phi) 0 0 0 0;
    0 0 1 0 0 0;
    0 0 0 cosd(phi) sind(phi) 0;
    0 0 0 -sind(phi) cosd(phi) 0;
    0 0 0 0 0 1];
    
    k=E*A/L;
klocal=[k -k;
    -k k];
    
    kv= [A*E/L 0 0 -A*E/L 0 0;
    0 12*E*I/L^3 6*E*I/L^2 0 -12*E*I/L^3 6*E*I/L^2;
    0 6*E*I/L^2 4*E*I/L 0 -6*E*I/L^2 2*E*I/L;
    -A*E/L 0 0 A*E/L 0 0;
    0 -12*E*I/L^3 -6*E*I/L^2 0 12*E*I/L^3 -6*E*I/L^2;
    0 6*E*I/L^2 2*E*I/L 0 -6*E*I/L^2 4*E*I/L];
    
     X=A*E/L;
    G=E/(2+2*nu);
    S=G*K/L;
    Y1=12*E*Iz/L^3;
    Y2=6*E*Iz/L^2;
    Y3=4*E*Iz/L;
    Y4=Y3/2;
    Z1=12*E*Iy/L^3;
    Z2=6*E*Iy/L^2;
    Z3=4*E*Iy/L;
    Z4=Z3/2;
    K=[.5*X 0 0 0 0 0,-X 0 0 0 0 0;
        0 .5*Y1 0 0 0 Y2,0 -Y1 0 0 0 Y2;
        0 0 .5*Z1 0 -Z2 0, 0 0 -Z1 0 -Z2 0;
        0 0 0 .5*S 0 0,0 0 0 -S 0 0;
        0 0 0 0 .5*Z3 0, 0 0 Z2 0 Z4 0;
        0 0 0 0 0 .5*Y3, 0 -Y2 0 0 0 Y4;
        ...
        0 0 0 0 0 0, .5*X 0 0 0 0 0;
        0 0 0 0 0 0, 0 .5*Y1 0 0 0 -Y2;
        0 0 0 0 0 0, 0 0 .5*Z1 0 Z2 0;
        0 0 0 0 0 0, 0 0 0 .5*S 0 0;
        0 0 0 0 0 0, 0 0 0 0 .5*Z3 0;
        0 0 0 0 0 0, 0 0 0 0 0 .5*Y3];
    K=K+K';%simetriza las cosas
\end{verbatim}
\end{code}

\egroup