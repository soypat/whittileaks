\clearpage
 \subsection*{Código \Matlab{}}
    %%%%------------------------------------------------------------------CODE
    \subsubsection*{Código Axisimétrico para matriz rigidez (por radian)}
    Se va tratar con un método para el almacenado eficiente de las funciones de forma. Esto acorta el tiempo de corrido en un factor mayor a 100 para problemas medianos/grandes. Se puede tambien hacer para los nodos (código )
    \begin{code}
    \begin{verbatim}

%% ALMACENADO EFICIENTE
Ns=cell(npg,1); %Acelerar proceso.
dNs=cell(npg,1);
NLs=cell(npg,1);
for ipg=1:npg
    ksi=upg(ipg,1);eta=upg(ipg,2);
    Ns{ipg}=eval(subs(N));
    dNs{ipg}=eval(subs(dN));
    NLs{ipg}=eval(subs(NL));
end
    \end{verbatim}
    \end{code}
    
    \begin{code}
    \begin{verbatim}
    
kG=zeros(doftot);%Matriz global
for e=1:Nelem
    Ke=zeros(Ndofpornod*Nnodporelem);
    index=elementos(e,:);
    elenod=nodos(index,:);
    A=0;
    for ipg=1:npg
        ksi=upg(ipg,1); eta=upg(ipg,2);

        J=dNs{ipg}*elenod;
        Djac = det(J);
        dNxy=J\dNs{ipg};
        r = Ns{ipg}*elenod(:,1); 
        
        B = zeros(size(Caxi,2),Ndofpornod*Nnodporelem);
        B(1,1:2:Ndofpornod*Nnodporelem-1) = dNxy(1,:);
        B(2,1:2:Ndofpornod*Nnodporelem-1) = Ns{ipg}(1,:)/r;
        B(3,2:2:Ndofpornod*Nnodporelem) = dNxy(2,:);
        B(4,1:2:Ndofpornod*Nnodporelem-1) = dNxy(2,:);
        B(4,2:2:Ndofpornod*Nnodporelem) = dNxy(1,:);%extra fila por ser axisim.

        Ke = Ke + B'*Caxi*B*wpg(ipg)*r*Djac;%Despues aplicar cargas por radian
    end
    meindof=reshape(DOF(index,:)',1,[]);
    kG(meindof,meindof)=kG(meindof,meindof)+Ke;
end
\end{verbatim}
\end{code}

\subsubsection*{Recuperación de tensiones axisimétricas.}
\begin{code}
\begin{verbatim}
 
for e = 1:Nelem
    index = elementos(e,:);
    elenod = nodos(index,:);
    for inod = 1:Nnodporelem
        J = dNs{inod}*elenod;                  

        dNxy = J\dNs{inod};     
        r = Ns{ipg}*elenod(:,1);
        
        B = zeros(size(C,2),Ndofpornod*Nnodporelem);
        B(1,1:2:Ndofpornod*Nnodporelem-1) = dNxy(1,:);
        B(2,1:2:Ndofpornod*Nnodporelem-1) = Ns{inod}(1,:)/r;
        B(3,2:2:Ndofpornod*Nnodporelem) = dNxy(2,:);
        B(4,1:2:Ndofpornod*Nnodporelem-1) = dNxy(2,:);
        B(4,2:2:Ndofpornod*Nnodporelem) = dNxy(1,:);
        
        meindof = reshape(DOF(index,:)',1,[]);
        
        deformation = B*D(meindof);
        if r==0
            deformation(2)=deformation(1);
        end
        deformation;
        strain(e,inod,:) = deformation;
        S(e,inod,:) = C*deformation; % Stress
    end
end
\end{verbatim}
\end{code}
\subsubsection*{Carga centrifugas (por radian)}
\begin{code}
\begin{verbatim}

R = zeros(doftot,1);
for e = 1:Nelem
    index=elementos(e,:);
    Re = zeros(Ndofpornod*Nnodporelem,1);
    elenod = nodos(elementos(e,:),:);
    for ipg = 1:npg
        ksi=upg(ipg,1); eta=upg(ipg,2);
        J = dNs{ipg}*elenod;
        r = Ns{ipg}*elenod(:,1);
        
        Re = Re + (NLs{ipg}'*[1; 0])*det(J)*wpg(ipg)*r^2*rho*omega^2;
    end
    meindof=reshape(DOF(index,:)',1,[]);
    R(meindof) = R(meindof) + Re;
end
\end{verbatim}
\end{code}

\subsubsection*{Condicionamiento}
\begin{code}
\begin{verbatim}

CK = max(eigvals)/min(eigvals)
N=size(K,2);
S=zeros(N);
for i=1:N
    S(i,i)=sqrt(K(i,i))^(-1);
end
Ks=S*K*S;
scaledeigval=eig(Ks);
CKscaled=max(scaledeigval)/min(scaledeigval)
digitos_perdidos = floor(log10(CKscaled))
\end{verbatim}
\end{code}

\subsubsection*{Desplazamientos impuestos sin \emph{skew}}
\begin{code}
\begin{verbatim}

% Solve sin cargas por despl. impuestos
Dr = Kr\Rr;
%% Buscamos imponer desplazaments iniciales
dx=-sqrt(2);
dy=dx;
Dc=zeros(doftot,1);
CONOC=Fixed;
CONOC([9 10 11]*Ndofpornod)=true;
CONOC([9 10 11]*Ndofpornod-1)=true;
DECONOC=~CONOC;
Dc([9 10 11]*Ndofpornod)=dy;
Dc([9 10 11]*Ndofpornod-1)=dx;

Dx = K(DECONOC,DECONOC)\(R(DECONOC)-K(DECONOC,CONOC)*Dc(CONOC));
Rx = K(CONOC,CONOC)*Dc(CONOC)+K(CONOC,DECONOC)*Dx;
R(CONOC)=Rx;
Dc(DECONOC)=Dx;
D=Dc;
%Voy a recuperacion de tensiones
\end{verbatim}
\end{code}
\subsubsection*{Despl. impuestos con \emph{skew}}
\begin{code}
\begin{verbatim}

%Solve sin despl. impuestos
Dr = Kr\Rr;
%% SKEW SYSTEM Desp. imp.
T = eye(doftot);angulo =45;
for i = [9 10 11]
    nodedof = [i*2-1 i*2];
    T(nodedof,nodedof)=[cosd(angulo) sind(angulo); ...
    -sind(angulo) cosd(angulo)];
end
Dc = nan(doftot,1); Dc([1 2 3 9 10 11]*Ndofpornod)=0;
Dc([9 10 11]*Ndofpornod-1)=-2;
XX=isnan(Dc);CC=~XX; %DesConocidas y Conocidas
%Resuelvo
Kskew = T*K*T'; Rskew = T*R;
Dx = Kskew(XX,XX)^(-1)*(Rskew(XX)-Kskew(XX,CC)*Dc(CC));
Rskew(CC) = Kskew(CC,CC)*Dc(CC)+Kskew(CC,XX)*Dx;

%Ahora son TODOS conocidos jeje
Dc(XX)=Dx; % AhDesde ahora isnan(Dc)=0!

D=T'*Dc;
Desp = reshape(D',Ndofpornod,Nnod)';
% Directo a Recuperacio de tensiones en nodos.
\end{verbatim}
\end{code}

\subsubsection*{Error Q8}

\begin{code}
\begin{verbatim}

%% tensiones en los puntos de superconvergencia
stressSuper = zeros(nel,4,3);
uNod = [-1 -1
1 -1
1 1
-1 1]/sqrt(3);
for iele = 1:nel
nodesEle = nodes(elements(iele,:),:);
for inode = 1:4
% Punto de Gauss
ksi = uNod(inode,1);
eta = uNod(inode,2);
% Derivadas de las funciones de forma respecto de ksi, eta
dN = shapefunsder([ksi eta],'Q8');
% Derivadas de x,y, respecto de ksi, eta
jac = dN*nodesEle;
% Derivadas de las funciones de forma respecto de x,y.
dNxy = jac\dN; % dNxy = inv(jac)*dN

B = zeros(size(C,2),nDofNod*nNodEle);
B(1,1:2:15) = dNxy(1,:);
B(2,2:2:16) = dNxy(2,:);
B(3,1:2:15) = dNxy(2,:);
B(3,2:2:16) = dNxy(1,:);
eleDofs = nodeDofs(elements(iele,:),:);
eleDofs = reshape(eleDofs',[],1);
stressSuper(iele,inode,:) = C*B*D(eleDofs);
end
end
% Extrapolo a los Nodos
a = sqrt(3);
rsExt = a*[-1 -1
1 -1
1 1
-1 1
0 -1
1 0
0 1
-1 0];
stressExtra = zeros(nel,nNodEle,3);
for iele = 1:nel
    for inode = 1:nNodEle
        r = rsExt(inode,1);
        s = rsExt(inode,2);
        N = shapefuns([r s],'Q4');
        stressExtra(iele,inode,:) = N * squeeze(stressSuper(iele,:,:));
    end
end
%% Gauss
[wpg, upg, npg] = gauss([3 3]);
%% calculo de eta_el, e2, U2
invC = C\eye(3);
eta_el = zeros(nel,1);
e2_el = zeros(nel,1);
U2_el = zeros(nel,1);
for iele = 1:nel
    nodesEle = nodes(elements(iele,:),:);
    for inode = 1:nNodEle
        for ipg = 1:npg
            % Punto de Gauss
            ksi = upg(ipg,1);
            eta = upg(ipg,2);
            % Derivadas de las funciones de forma respecto de ksi, eta
            dN = shapefunsder([ksi eta],'Q8');
            % Derivadas de x,y, respecto de ksi, eta
            jac = dN*nodesEle;
            % Derivadas de las funciones de forma respecto de x,y.

            dNxy = jac\dN; % dNxy = inv(jac)*dN
            % funciones de forma
            N = shapefuns([ksi eta],'Q8');
            eleStress = squeeze(stress(iele,inode,:)); %
            tensiones "directas"
            starStress = squeeze(stressExtra(iele,inode,:)); %Str.mejoradas
            e2_el(iele) = e2_el(iele) + (starStress - eleStress)' * ...
            invC * (starStress - eleStress) * wpg(ipg) *det(jac);

            U2_el(iele) = U2_el(iele) + eleStress' * invC * eleStress * ...
            wpg(ipg) * det(jac);
        end
    end
    eta_el(iele) = sqrt( e2_el(iele) / (e2_el(iele) + U2_el(iele)) );
end
etaG = sqrt( sum(e2_el) / (sum(e2_el) + sum(U2_el)) )
%% Configuracion deformada
D = (reshape(D,nDofNod,[]))';
nodePosition = nodes + D(:,1:2);
%Graficacion
limites = [min(min(stressExtra(:,:,2))),max(max(stressExtra(:,:,2)))];
figure(1)
bandplot(elements,nodePosition,stress(:,:,2),limites,'k');
\end{verbatim}
\end{code}