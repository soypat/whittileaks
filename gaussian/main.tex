\documentclass{article}
\usepackage[utf8]{inputenc}

\title{Apuntes Gaussian Processes}
\author{Patricio Whittingslow}
\date{March 2019}
\usepackage{amssymb}
\usepackage{amsmath}
\usepackage{mdframed,xcolor}

\newmdtheoremenv[%
linecolor=gray,
leftmargin=30,%
rightmargin=0,
backgroundcolor=blue!4,%
innertopmargin=2pt,%
ntheorem]{define}{Definición}[section]

\newmdtheoremenv[%
linecolor=gray,
leftmargin=30,%
rightmargin=0,
backgroundcolor=blue!4,%
innertopmargin=2pt,%
ntheorem]{code}{Código}[section]

%% Formatting Commands
\newcommand{\glossentry}[2]{$#1$\indent #2 \par \vspace{.4cm} } %Entradas para glosario

%% Macro commands
\newcommand{\vectorbold}[1]{\mathbf{#1}}

%% Definition Commands
\newcommand{\Dim}{\mathbb{D}}
\newcommand{\T}{\textsc{t}}
\newcommand{\func}{f}
\newcommand{\noise}{\varepsilon}
\newcommand{\GD}{\mathcal{N}}
\newcommand{\sn}{\sigma_{n}}
\newcommand{\xbold}{\vectorbold{x}}
\newcommand{\ybold}{\vectorbold{y}}
\newcommand{\vbold}{\vectorbold{v}}
\newcommand{\fbold}{\vectorbold{f}}
\newcommand{\pbold}{\vectorbold{p}}
\newcommand{\qbold}{\vectorbold{q}}
\newcommand{\cov}{\textrm{cov}}

\begin{document}

\maketitle

\section{Prefacio}

Es importante la palabra lineal. Aquí no hay funciones con cuadrados ni nada por el estilo. Existe una función global lineal que toma como argumento las variables elegidas.
\subsection*{Glosario Términos}

\glossentry{\vectorbold{x}}{Input vector ($\Dim$). Que variables eligo para modelo.}

\glossentry{\vectorbold{y}}{Vector objetivo ($n$). Observaciones/mediciones.}
\glossentry{\vectorbold{w}}{Weight vector ($\Dim$). Aqui van los parámetros a obtener de la regresión lineal. La ``solución" de la regresión lineal.}
\glossentry{\Dim}{Dimensión del problema. Cuantas variables elegí para el modelo.}
\glossentry{n}{Cantidad de observaciones.}
\glossentry{X}{Matriz de diseño ($\Dim \times n$). }
\glossentry{\noise\sim \GD (0,\sn^2) }{Ruido. Sigue una distribución gausiana independiente, idénticamente distribuida con promedio $0$ y varianza $\sn^2$}
\glossentry{|\vbold|=\left(\sum_i\vbold_i^2\right)^\frac{1}{2}}{Longitud euclidiana del vector $\vbold$. }
\glossentry{\sn}{Asumimos que las mediciones están alejadas de la función $\func$ por causa de ruido $\noise$.}

\glossentry{\Sigma_p}{Matriz de covarianza o \textbf{matriz de varianza-covarianza}.}
\subsection*{Glosario Subindices}
% \glossentry{\qbold,\pbold}{Refiere a un punto de datos. $\xbold_\qbold$ sería un vector de entradas.}
\glossentry{i,j,k,p,q}{Refiere a un elemento de un tensor, sea vector,matriz, etc.}
\subsection{Introducción}

Dado datos de entrenamiento ($X,\ybold$) se quiere efectuar predicciones para nuevas entradas $\xbold_*$ usando una función $\func$, por lo tanto, es claro que el problema es \textit{inductivo}. Para lograr esto tenemos que suponer ciertas características sobre nuestra función $\func$. Si no acotáramos $\func$ entonces cualquier función que sea consistente con los datos de entrenamiento valdría. Este es el problema que intenta resolver el aprendizaje de maquina, el \textbf{Machine Learning.}
\section{Regresion Lineal}
Para que sea mas didáctico la aplicación de la regresión lineal se va plantear un problema a resolver:
\textit{Se tiene un cilindro de diámetro $D$ y largo $L$ que enfrenta un fluido de viscosidad $\mu$, densidad $\rho$ y velocidad $U_\infty$. Obtener una expresión para la fuerza $F_D$ que el fluido ejerce sobre el cilindro?}

Cabe destacar que el problema no es lineal y que probablemente la regresión lineal que obtengamos sea valida para unos pocos puntos. Igual intentaremos hallar la regresión lineal.

Primero tenemos que partir de datos, preferiblemente \textbf{muchos} datos. Vamos a obviar la variable $L$, ya que si dividimos la fuerza por el largo obtenemos la fuerza por unidad de largo y reducimos la dimensión del problema sin perder información.

Abajo esta la matriz de diseño para el problema. Se efectuaron $n$ observaciones o mediciones, por ende también se tienen $n$ resultados para la fuerza $y=F_{D_i}$.

\[X=
\begin{bmatrix}
\mu_1 & \mu_2 &\ldots& \mu_n \\
\rho_1 &\rho _2 &\ldots &\rho_n \\
U_{1} & U_{2}&\ldots & U_{n}\\
D_1 & D_2 & \ldots& D_n
\end{bmatrix}
\]
\[
\vectorbold{y} = 
\begin{Bmatrix}
F_1 \\
F_2 \\
\vdots\\
F_n
\end{Bmatrix}
\]

\subsection{No-Bayesiana}
Como las mediciones nunca son exactas (siempre hay error humano o por el sistema de medición) se puede proponer ruido ($\varepsilon$) de distribución gaussiana $\varepsilon \sim \GD (0,\sigma_{n}^{2})$ que se suma a nuestra regresión para obtener los valores de $y$ tal que 
\[
\func(\vectorbold{x})=\vectorbold{x}^{\T} \vectorbold{w} \qquad \qquad y=\func(\vectorbold{x})+\varepsilon
\]

Se presentaron dos nuevos vectores al usuario, $\vectorbold{x}$ y $\vectorbold{w}$, los vectores \textit{input} y de \textit{pesos} (\textit{weight} en inglés), respectivamente. Ambos tienen dimensión del problema, en este caso $\Dim=4$.
\[ \vectorbold{x} =
\begin{Bmatrix}
\mu_i \\
\rho_i \\
U_i \\
D_i
\end{Bmatrix} \qquad \qquad
\vectorbold{w} =
\begin{Bmatrix}
w_a \\
w_b \\
w_c \\
w_d
\end{Bmatrix}
\]
Behold. La regresión lineal entonces quedaría de la forma
\[
\func(\vectorbold{x}) = w_a\mu_i +w_b\rho_i+w_cU_i+w_dD_i =y_i = F_i
\]
Resolver el problema (efectivamente: obtener el vector $\vectorbold{w}$) requiere un uso pesado de la probabilidad, en particular el teorema de Bayes. También se tiene que elegir un \textit{prior}, es decir, dejar expresadas nuestras creencias de los parametros (pesos) antes de mirar las observaciones. Esto toma la forma de la matriz de covarianza $\Sigma_p$. Si suponemos que los parametros se distribuyen según $\vectorbold{w} \sim \GD(\vectorbold{0},\Sigma_p)$ entonces:\footnote{Aún no se ha explicado como se calcula $\Sigma_p$. }

\[
p(\vectorbold{w}|X,\vectorbold{y}) \sim \GD(\Bar{\vectorbold{w}},A)
\]
\begin{equation}
   A = \left(\sigma_n^{-2}XX^{\T}+\Sigma_p^{-1}\right)^{-1}
\end{equation}
\begin{equation}
     \Bar{\vectorbold{w}} = \sigma_{n}^{-2}\left( \sigma_{n}^{-2}XX^{\T}+\Sigma_p^{-1} \right)^{-1}X\vectorbold{y}
\end{equation}

Donde $\Bar{\vectorbold{w}}$ son los promedios de los pesos hallados. $A$ es la \textit{matriz de covarianza de la distribución posterior}.

Se suele referir al promedio de los pesos $\Bar{\vectorbold{w}}$ como MAP, \textit{maximum a posteriori. }

\subsection{Bayesiana}
Se ha hablado de la formulación No-Bayesiana hasta ahora. De que se trata esto de si es Bayesiano o no?  En el la formulación Bayesiana el caso prueba tiene la forma del vector de entrada, y la regresión se calcula para ese vector prueba. Por ende, en el esquema Bayesiano no existe un único vector $\vectorbold{w}$. Un subíndice $_*$ se refiere a un caso prueba.
\begin{equation}
    p(\func_{*}|\vectorbold{x}_*,X,\vectorbold{y}) = \GD \left( \sigma_n^{-2}\vectorbold{x}_*^{\T}AX\vectorbold{y},\vectorbold{x}_*^\T A\vectorbold{x}_* \right)
\end{equation}
Se usará esta formulación para la regresión en el espacio funcional. $A$ se definió anteriormente.

\section{Regresión en el espacio funcional}
Si bien hemos obtenido una regresión cuando calculamos $\Bar{\vectorbold{w}}$, no resuelve el problema debido a que el problema \textit{no es lineal}. 

Esto se  resuelvo proyectando las variables de entrada a un espacio funcional. Nuestra función para la regresión tendría la forma

\begin{equation}
    \func(\vectorbold{x}) = \phi (\vectorbold{x})^{\T} \vectorbold{w}
\end{equation}

esta función $\phi$ mapea el vector de entrada $\vectorbold{x}$ de un espacio $\Dim$ a un espacio funcional $N$. La aplicación de este modelo es análoga a la regresión lineal excepto que donde antes teníamos $X$ ahora tenemos:
\[
\Phi =\Phi(X)
\]

\begin{equation} \label{eq:FeatureSpaceProb1}
    p(f_*|\vectorbold{x}_*,X,\vectorbold{y})\sim \GD \left(\sigma_{n}^{-2}\phi(\vectorbold{x}_*)^\T A\Phi\vectorbold{y},\phi(\vectorbold{x}_*)^\T A\phi(\vectorbold{x}_*) \right)
\end{equation}
donde
\[
A =\left(\sigma_n^{-2}\Phi \Phi^\T +\Sigma_p^{-1}\right)^{-1}
\]
\subsection{Kernel}
Si uno quisiese ahorrar tiempo y no calcular la inversa de $A$ (calculo computacional pesado) podria recurrir a la expresión (donde $\phi(x)=\phi$):

\begin{equation} \label{eq:KernelTrickFeatureSpace}
        p(f_*|\vectorbold{x}_*,X,\vectorbold{y})\sim \GD \left(\phi_*^\T \Sigma_p \Phi (K+\sn^2I)^{-1}\vectorbold{y},\phi_*^\T \Sigma_p \phi_* -\phi^\T_* \Sigma_p\Phi(K+\sn^2I)^{-1}\Phi^\T \Sigma_p\phi_* \right)
\end{equation}
donde $K=\Phi^T\Sigma_p\Phi$.

Note que en la ecuación \ref{eq:KernelTrickFeatureSpace} es prevalente el producto de el espacio funcional en las formas $\phi_*^\T \Sigma_p \Phi$,  $\phi_*^\T \Sigma_p \phi_*$ y $\Phi^\T \Sigma_p\Phi$. Definiendo entonces una función $\psi(\xbold)=\Sigma_p^{\frac{1}{2}}\phi(\xbold)$ obtenemos
\begin{equation}
    k(\xbold,\xbold^{\prime}) = \psi(\xbold)\cdot \psi(\xbold^\prime)
\end{equation}
donde $\xbold$ y $\xbold^\prime$ pueden ser las variables de entrada o casos prueba (denotado con subíndice $*$). Se le suele decir la función kernel\footnote{También llamada función covarianza.} a $k$ y es de sumo interés en la rama del aprendizaje de maquina. Si reescribieramos \ref{eq:KernelTrickFeatureSpace}:
\begin{equation}\label{eq:KernelTrickSubstituted}
p\sim \GD \left(k(\xbold_*,X) (k(X,X)+\sn^2I)^{-1}\vectorbold{y},k(\xbold_*,\xbold_*) -k(\xbold_*,X)(k(X,X)+\sn^2I)^{-1}k(X,\xbold_*) \right)
\end{equation}

Puede ser que llegado a este punto el lector se sienta ansioso por no tener las herramientas de como aplicar \ref{eq:KernelTrickSubstituted}. Para aliviar dichas dudas se presenta una posible función de covarianza, la \textit{squared exponential} (SE).\footnote{también llamada la gausiana o \textit{Radial Basis Function}}
\begin{equation}
    \cov(\func(\xbold_p),\func(\xbold_q)) = k(\xbold_p,\xbold_q) = \exp(-\tfrac{1}{2}\left|\xbold_p-\xbold_q \right|^2/\ell)
\end{equation}
donde $\ell$ es la longitud característica del proceso gausiano.



\begin{define}
Un \emph{Proceso Gausiano} es una colección de cualquier número finito de variables aleatorias, todas distribuidas gausianamente.
\end{define}

\subsection{Prediccion con ruido}
Suponiendo que $y=\func(x)+\varepsilon$ siendo el ruido $\varepsilon \sim \GD(0,\sigma_n^2)$
\[
\cov (y_p,y_q) = k(\xbold_p,\xbold_q) +\sn^2\delta_{pq} \qquad o \qquad \cov (\ybold) = K(X,X)+\sn^2+I
\]

La distribuci'on de $\ybold$ y las salidas según el \textit{prior} elegido son
\[
\begin{bmatrix}
\ybold \\
\fbold
\end{bmatrix}\sim
\GD\left(\mathbf{0},
\begin{bmatrix}
K(X,X)+\sn^2I & K(X,X_*) \\
K(X_*,X) & K(X_*,X_*)
\end{bmatrix}
\right)
\]

\end{document}