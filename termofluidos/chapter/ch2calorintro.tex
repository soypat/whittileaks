\clearpage
\part{Transferencia de Calor}
\section{Principios de transferencia de calor}
\subsection{Conducción de Calor}
\begin{equation} \label{eq:LaEcuacionBasica}
    q_k\propto A\frac{\di T}{\di x}\rightarrow q_k=-kA\frac{\di T}{\di X}
\end{equation}

El signo negativo en \ref{eq:LaEcuacionBasica} es por causa de la segunda ley de la termodinámica que nos dice que el calor fluye de un punto de mayor temperatura a uno de menor temperatura.

Temperatura uniforme cara plana.
$$q_k=-\frac{kA\Delta T}{L} $$
$$R_k=\frac{L}{Ak}$$

Analogía eléctrica:
$$V\sim T $$
$$I \sim \dot{q}_{\text{calor}}$$
$$R\sim\frac{1}{K}= \frac{1}{\bar{h}_cA}\sim \frac{L}{kA} $$
$$C=\frac{Q}{V}\sim c\cdot m \si{}$$
Conducción térmica
$$K_k=\frac{kA}{L}$$
La razón $k/L$ se le dice la conducción térmica unitaria. $L/k$ es la  resistencia térmica unitaria.

Aproximación lineal de conductividad térmica en función de la temperatura
$$k(T)=k_0(1+\beta_kT)$$ donde $\beta$ es un valor empírico.

Usando está ecuación se aproxima el calor intercambiado con un $k$ promedio.

$$q_k=\frac{k_{\textrm{prom}}A\Delta T}{L}$$

Ley de Fourier
$$ k=\frac{q_k/A}{|\di T/\di x|}$$

Dos mecanismos de transferencia de calor por conducción
\begin{itemize}
    \item Vibraciones de la estructura reticulada
    \item Transporte de electrones (el mas efectivo de los dos)
\end{itemize}
Los metales conducen calor de las dos formas mientras que los solidos no-metálicos solo conducen por vibraciones.
\subsection{Convección}
El mecanismo de transferencia de calor sigue siendo la conducción pero se hace la distinción porque la transferencia de calor está prácticamente determinada por el campo de
velocidad.% \citepalias{vieytes2018filminas}. 
$$q_c=-k_{\textrm{fluido}}A\left| \spartial{T}{y}\right|_{y=0} $$
$$q_c=\hcb A\Delta T $$
El coeficiente $h_c$ local esta dado por
$$ \di q_c=\hcb \di A(T_s-T_{\inf}) $$

$$\bar{h}_c=\frac{1}{A}\int_A\int h_c\di A$$
Conductividad térmica para transferencia de calor por convección:
$$K_c=\hcb A$$
Si el cuerpo está en presencia de una corriente externa la razón del número de Grashof sobre el cuadrado de Reynolds puede ser usada para determinar que modo de convección domina
\[
\frac{\Grashof}{\Reynolds^2}
\begin{cases}
\gg 1&\quad \!\textrm{Convección forzada despreciable} \\
\approx  1&\quad\! \textrm{Convección forz. y natural combinadas}\\ 
\ll 1 & \quad \! \textrm{Convección natural despreciable}
\end{cases}
\]
\subsection{Radiación}
Cuerpo negro $\varepsilon=1$. Cuerpo \textbf{rodeado completamente} por superficie a $T_2$ da $\fforma=1$. El factor de forma ($\fforma$) se tiene que tomar en cuenta cuando ambos cuerpos \textbf{no} son cuerpos negros y además tienen una relación geométrica.
$$q_r=A_1\fforma_{1\boldsymbol{-}2}\sigma(T_1^4-T_2^4) $$
$$K_r=\frac{A_1\fforma_{1\boldsymbol{-}2} \sigma(T_1^4-T_2^4)}{T_1-T_2}$$
Coeficiente de transferencia de calor por radiación: $\hrb=\frac{K_r}{A_1}$
\subsection{Termodinámica}
Primera ley de Termo. $e$ como energía especifica.
$$e\dot{m}_\textrm{entrada}+q+\dot{q}_G-(e\dot{m})_\textrm{out}-W_{\textrm{salida}}=\spartial{E}{t} $$

Para sistemas cerrados:
$$q+\dot{q}_G-W_{\textrm{salida}}=\spartial{E}{t}$$

Según la termodinámica, la entalpía se define como $h=\ei+\frac{p}{\rho}$

$$\frac{\Di h}{\Di t}=\frac{\Di \ei }{\Di t}+\frac{1}{\rho}\frac{\Di p}{\Di t}- \frac{p}{\rho^2} \frac{\Di \rho}{\Di t} $$
luego tenemos que 

\begin{equation}\label{eq:entalpia}
    \rho \frac{\Di h}{\Di t} = -p\spartial{v_j}{x_j} +\spartial{ }{x_j}\left( k\spartial{T}{x_j}\right) +\Phi +\frac{\Di p}{\Di t}+ \frac{p}{\rho} \spartial{v_j}{x_j} 
\end{equation}
\section{Ecuaciones de conducción de calor}
\subsection{Coordenadas cartesianas}
Calor entrante a VC+generación de calor en el VC=Calor saliente de VC+Calor guardado en el VC
\begin{align*} \numberthis \label{eq:condcartesian}
-kA\spartial{T}{x}\bigg|_x+\dot{q}_GA\Delta x &= \\ -kA\spartial{T}{x} & \bigg|_{x+\Delta x}+\rho A \Delta  x c \spartial{T(x+\Delta x/2,t)}{t}
\end{align*}
Expandiendo las derivadas parciales con Taylor...
$$\spartial{B}{x}\bigg|_{x+\di x}=\spartial{B}{x} \bigg|_x+\dpartial{B}{x}\bigg|_x \di x $$
$$\spartial{B}{t} x+\left[ \left( \frac{\Delta x}{2},t\right) \right] =\spartial{B}{t}\bigg|_{x} + \frac{\partial^2 B}{\partial x \partial B}\bigg|_x \frac{\Delta x}{2}\ldots $$

Tenemos que la ecuación \ref{eq:condcartesian} se puede simplificar

\begin{equation}\label{eq:condcart}
k\dpartial{T}{x}+\dot{q}_G=\rho c\spartial{T}{t}
\end{equation}
La difusividad térmica $\alpha=\frac{k}{\rho c}$
\section{Ecuación del Calor}
\begin{equation}\label{eq:HEATorENERGY}
    \rho \left(\spartial{cT}{t}+v_j\spartial{cT}{x_j}\right) = \spartial{ }{x_j}\left(k\spartial{T}{x_j}\right) -p \spartial{v_j}{x_j}+\Phi+\dot{q}_G
\end{equation}
Un proceso de difusión en donde la energía térmica se transfiere de un extremo caliente a otro frío, por medio de un intercambio de energía intermolecular.

Para describir el fenómeno la conducción se aprovecha de la \emph{ecuación  de conducción}. Para llegar a ésta partimos desde \ref{eq:einterna}, definiendo la disipación como $\Phi=S_{ij}\spartial{v_i}{x_j}$. Escrita en coordenadas cartesianas
\begin{equation*}
    \Phi\!=\!\mu\! \left[ 2\! \left(\spartial{u}{x}\right)^2 \! \!+ 2\! \left( \spartial{v}{y} \right)^2\!\!+2\!\left( \spartial{w}{z}\right)^2 \!+\!\left( \spartial{v}{x}\!+\spartial{u}{y}\right)^2 \!\!+\!\!\! \right.
\end{equation*}
\begin{equation}\label{eq:disipcartesianas}
    \!\left.\! \left(\spartial{w}{y} \!\!+\! \spartial{v}{z}\right)^2 \!\!+\left( \spartial{u}{z}+\spartial{w}{x} \right)^2 +\frac{\lambda}{\mu}\left( \spartial{u}{x}\!+\spartial{v}{y}\!+\!\spartial{w}{z} \right)^2 \right]
\end{equation}

Para un solido suponemos que tiene una temperatura uniforme $\ei = cT$ donde $c$ es el calor especifico, y perfil velocidades nulo $v_i=0$. Por lo tanto $\frac{\Di e}{\Di t}=\spartial{cT}{t}+\cancel{v_i\spartial{cT}{x_i}}$. \ref{eq:einterna} se puede reescribir de tal forma

\begin{equation} \label{eq:calorconduccion}
    \rho \frac{\Di c T}{\Di t}\equiv \rho c \spartial{T}{t} = \spartial{ }{x_j}\left( k\spartial{T}{x_j} \right) + \dot{q}_G
\end{equation}

El termino $\spartial{ }{x_j}\left( k\spartial{T}{x_j} \right)$ es la curvatura del perfil de temperaturas. Entonces $\spartial{T}{t}$ depende del calor generado y de la curvatura.

Para un caso unidimensional de un material con propiedades débilmente dependientes de la temperatura se simplifica
\begin{equation}\label{eq:conduccion1D}
    \rho c \spartial{T}{t} = k\dpartial{T}{x} + \dot{q}_G 
\end{equation}

\subsection{Forma adimensional}
Se definen valores de referencia que gobiernan el proceso de transferencia de calor $T_r$, $L_r$ y $t_r$.
$$\theta=\frac{T}{T_r},\qquad\xi =\frac{x}{L_r}\qquad\tau=\frac{t}{t_r}$$
La forma unidimensional de la ecuación de la conducción se puede expresar así:
$$ \dpartial{\theta}{\xi}+\dot{Q}_G=\frac{1}{\Fo} \spartial{\theta}{\tau}$$

donde $\dot{Q}_G=\frac{\dot{q}_GL_r^2}{kT_r}$ y $\Fo$ es el número de Fourier $\Fo=\frac{\alpha t_r}{L_r^2}=\frac{k/L_r}{\rho cL_r/t_r}$
\subsubsection{Método Vieytes}
El cambio de variables ahora es

$$\theta=\mathscr{H}(\tau) X(\xi)\qquad \xi = \frac{x}{L_r}\qquad  \tau=\frac{tk}{\rho cL^2}$$
La ecuación de calor (\ref{eq:calorconduccion}) ahora es
\begin{equation}\label{eq:calorvieytesconduccionadimensional}
    \spartial{\theta}{\tau}=\dpartial{\theta}{\xi}+\beta f(\xi,\tau)
\end{equation}
donde $\beta=\frac{\theta_0 L^2}{k\dot{q}_0}$ es adimensional. $\theta_0$ es una escala de temperatura, $\dot{q}_0$  una escala para la potencia generada en volumen no mecánicos, $f(\xi,\tau)$ la distribución de potencia generada la cual supondremos igual a $0$.

Recordando la materia $93.44$, supondremos que la solución es de variables separables.
Reemplazando $\theta$ en la ecuación \ref{eq:calorvieytesconduccionadimensional}
$$\frac{1}{\mathscr{H}} \spartial{\mathscr{H}}{\tau}=\frac{1}{X}\dpartial{X}{\xi}$$
Haciendo la cuenta llegamos a que
$$\theta(\xi,\tau)=\overbrace{\left(C_1\cos \lambda \xi +C_2 \sin \lambda \xi \right)}^{X}\overbrace{e^{-\lambda^2 \tau}}^{\mathscr{H}}$$
\subsection{Coordenadas cilíndricas y esféricas}
\noindent Cilíndricas
\begin{equation}\label{eq:coordcilindricas}
\frac{1}{r} \spartial{ }{r}\bigg( r\spartial{T}{r} \bigg)+\frac{1}{r^2}\dpartial{T}{\phi}+\dpartial{T}{z}+\frac{\dot{q}_G}{k}=\frac{1}{\alpha}\spartial{T}{t}
\end{equation}
Esféricas
\begin{align*}\numberthis \label{eq:coordesfericas}
\frac{1}{\alpha}\spartial{T}{t}= \frac{1}{r^2} \spartial{ }{r}\bigg( r^2\spartial{T}{r} \bigg)+\frac{1}{r^2\sin^2\theta}\spartial{T}{\theta}\bigg( \sin\theta \spartial{T}{\theta}\bigg)&\\
+\frac{1}{r^2\sin\theta}\dpartial{T}{\phi}+\dpartial{T}{z}+&\frac{\dot{q}_G}{k}
\end{align*}

\subsection{Conducción inestable o transitoria}
En sistemas que no están en régimen estacionario el termino respecto tiempo de la ecuación de calor es no-nulo. Esto trae complicaciones.

Para tratar estos problemas se hacen las siguientes simplificaciones
\begin{enumerate}
    \item La temperatura solo es función del tiempo $T=f(t)$
    \item Sistemas con resistencia interna despreciable de lo cual resulta temperatura uniforme en todo el sistema $T(x,y,z)=cte$
    \item $\hhb$ constante durante todo el proceso
    \item $T_\infty$ no varía con el tiempo
\end{enumerate}

El número de Biot es relevante al modelar estos sistemas. 
$$\Biot =\frac{R_\interna}{R_\externa}=\frac{\hhb \ell}{k_s}$$
donde $\ell=\frac{V}{A}$ es la razón de volumen a superficie del sistema.

Cuando $ \Biot <0,1$ el error por modelar el sistema (placa, cilindro o esfera) con resistencia interna despreciable sera menor a $5\%$. A este análisis se le dice \emph{lumped heat capacity method}.