\section{Convección Interna Laminar}
\subsection{Correlaciones empíricas}

Se comienza definiendo el diámetro hidráulico para poder aplicar las correlaciones a secciones que no son cuadradas ni circulares
\begin{equation}
    D_H=4\cdot \frac{\textrm{Sección de flujo}}{\textrm{Perimetro mojado}}
\end{equation}

Luego se define un valor que se usa seguido como criterio para las correlaciones $\Lambda$:
\[
\Lambda = \frac{\Reynolds_{D_H}\Prandtl D_H}{L}
\]

{\bf Ductos \emph{cortos} circulares o rectangulares.}

Si se trata el ducto rectangular corto con la hipótesis de placas planas
\begin{equation}
    \overline{\Nusselt}_{D_H}=-\frac{\Lambda}{4}\ln \left( 1 - 2,654 \Prandtl^{0,167} \Lambda^{-,5}\right)
\end{equation}