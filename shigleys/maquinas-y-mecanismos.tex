% !TeX spellcheck = es_ES
% !TeX spellcheck = pdflatex

\title{Resumen de Elementos de Maquina}
\author{pwhittingslow@itba.edu.ar}

\documentclass[twocolumn,10pt]{article}
\usepackage[utf8x]{inputenc}
\usepackage[spanish]{babel}
\usepackage[a4paper,top=2cm,bottom=2cm,left=2cm,right=2cm,marginparwidth=1.75cm,headheight=28pt]{geometry}
%Letra general
\usepackage{mathastext}

\renewcommand{\familydefault}{\sfdefault}
\usepackage[scaled=1]{helvet}
\usepackage[format=plain,
            labelfont={bf,it},
            textfont=it]{caption}
\usepackage{color}
\usepackage{fontawesome}
\usepackage{siunitx}
\newcommand{\ansp}{\!\!\!\!\!\!\!}%Anti-space
%Math
\usepackage{amsmath}
\usepackage{amsfonts}
\newcommand{\rmit}[1]{{\fontfamily{cmr}\selectfont #1}}
%commands
\newcommand{\glossentry}[2]{$#1$\indent #2 \par \vspace{.4cm} } %Entradas para glosario
\newcommand{\agma}{\textrm{AGMA }}
\newcommand{\bigpar}[1]{\bigg(
#1 \bigg) }
\newcommand{\dprime}{ {\prime \prime} }
\newcommand{\tprime}{ {\prime \prime \prime} }
\newcommand{\fit}{\textit{\textrm{f }}}
\newcommand{\corr}{{\textrm{\color{red} Revisar}}}
\newcommand{\Kf}[1]{ K_{\textit{\textrm{f}}_{\textrm{#1}}} }
\newcommand{\Kfs}[1]{ K_{\textit{\textrm{fs}}_{\textrm{#1}}} }
\newcommand{\siga}[1]{ \sigma_{{\textrm{a}}_{\textrm{#1}}} }
\newcommand{\sigm}[1]{ \sigma_{{\textrm{m}}_{\textrm{#1}}} }
\newcommand{\pe}{\textit{\textrm{p}}}
\newcommand{\gut}{{\color{green}\faCheck}}
\newcommand{\Fall}{\rmit{Fall}}
\newcommand{\Rise}{\rmit{Rise}}
\newcommand{\Dwell}{\rmit{Dwell}}
\begin{document}
\maketitle

\section{Fatiga}
Los tres métodos de calcular la vida a fatiga son \emph{tensión-vida (también conocido como fatiga-vida), deformación-vida} y \emph{método de mecánica de fractura elástico-lineal} que asume que ya hay una fractura presente en el material. Para \emph{High-cycle fatigue} ($N>10^3$) se suele usar el metodo de \emph{tension-vida} porque la aproxima adecuadamente con efectuar un número alto de ensayos sobre muestras (resultados estadísticos). \par
El método \emph{deformación-vida} es el que mejor aproxima la naturaleza de la fatiga. Su desventaja es que se necesitan realizar varias idealizaciones que agregan a la incertidumbre por eso no se suele usar.\par

\textbf{Tipos de fatiga}
\begin{itemize}
    \item Invertido (Completamente Alternante) $\sigma_m=0$
    \item Repetido $\sigma_m=\sigma_a\Rightarrow \sigma_{m\acute{i}n}=0$
    \item Fluctuante o Variable $|\sigma_m|>\sigma_a$
\end{itemize}

\subsection{Materiales dúctiles}
\textbf{Resistencia a corte}. Los materiales dúctiles tienen menor resistencia al corte. La siguiente relación vale para la gran mayoría de materiales dúctiles.
$$S_{ut_s}=0,67\,S_{ut}\qquad S_{y_s}=0,577\,S_y $$

\subsubsection{Esfuerzo Invertido}
El limite de resistencia a la fatiga ($S_e^\prime$) se determina en un ensayo. Como no podemos ensayar el acero en el parcial se estima con la siguiente relación:
\[ S_e^\prime =
  \begin{cases}
    0,5S_{ut}   & \quad S_{ut}\leq 1400 \si{\mega \pascal}\\
    700\si{\mega \pascal}    & \quad S_{ut}>1400 \si{\mega \pascal} 
  \end{cases}
\]

Resistencia a la fatiga para un determinado $N$ numero de ciclos:
$$ S_\fit=aN^b $$
$$a=\frac{(\fit S_{ut})}{S_e} \qquad b=-\frac{1}{3}\log\bigpar{\frac{\fit S_{ut}}{S_e}} $$
Donde \fit es la fracción de resistencia y $\sigma_F^\prime $ es el esfuerzo de rotura real del material:
$$\fit = \frac{\sigma_F^\prime}{S_{ut}(2\cdot 10^3)^b} \qquad \sigma_F^\prime=S_{ut}+345MPa$$
$$\text{Despejando b:}\;  b=-\frac{\log (\sigma_F^\prime/S_e^\prime)}{\log(2N_e)}$$
El exponente $b$ se puede obtener de tabla ya que la vida de resistencia a la fatiga ($N_e$) es un valor experimental.\par
Limite de resistencia a la fatiga (vida infinita) corregida por los factores de condición de superficie ($k_a$), tamaño respecto ensayo ($k_b$), carga ($k_c$), temperatura ($k_d$), confiabilidad ($k_e$) y efectos diversos ($k_f$):
$$ S_e=k_a k_b k_c k_d k_e k_f S_e^\prime $$
\subsubsection{Esfuerzos fluctuantes}
Langer (limita falla por fluencia estática, lo infradimensiona). $n$ es el factor de seguridad:
$$S_a+S_m=S_y\qquad \sigma_a+\sigma_m=\frac{S_y}{n}$$
Soderberg (Sobredimensiona):
$$\frac{S_a}{S_e}+\frac{S_m}{S_y}=1\qquad \frac{\sigma_a}{S_e}+\frac{\sigma_m}{S_y}=\frac{1}{n}$$
Goodman modificada (funciona bien)
$$\frac{S_a}{S_e}+\frac{S_m}{S_{ut}}=1  \qquad \frac{\sigma_a}{S_e}+\frac{\sigma_m}{S_{ut}}=\frac{1}{n}$$
Gerber (se acerca bien a los datos experimentales):
$$ \frac{S_a}{S_e} + \bigg( \frac{S_m}{S_{ut}} \bigg)^2 =1 \qquad  \frac{n\sigma_a}{S_e}+\bigpar{\frac{n\sigma_m}{S_{ut}}}^2 =1 $$
ASME (the best):
$$ \bigpar{\frac{S_a}{S_e}}^2+\bigpar{\frac{S_m}{S_y}}^2=1 \qquad \bigpar{\frac{n\sigma_a}{S_e}}^2+\bigpar{\frac{n\sigma_m}{S_y}}^2=1$$

\textbf{Compresión.} La compresión dificulta el avance de las fallas  $\Rightarrow$  mejora la resistencia a la fatiga. Aun así, ante la falta de datos publicados se resigna esta ventaja.
\subsection{Materiales frágiles}
Criterio Smith-Dolan
$$ \frac{n\sigma_a}{S_e}=\frac{1-n\frac{\sigma_m}{S_{ut}}}{1+n\frac{\sigma_m}{S_{ut}}} \qquad \frac{S_a}{S_e}=\frac{1-\frac{S_m}{S_{ut}}}{1+\frac{S_m}{S_{ut}}}$$
\subsection{Criterios a torsión}
Cuando la carga es de torsión combinada con flexión se elige $k_c=1$ y se usa la tensión efectiva de Von Mises ($\sigma_m^\prime$ y $\sigma_a^\prime$) %Revisar la seccion
\[ k_c =
  \begin{cases}
    1       & \quad flexi\acute{o}n\\
    0,85    & \quad axial \\
    0,59      &\quad torsi\acute{o}n
  \end{cases}
\]
\subsection{Concentradores de tensiones}
Para fatiga se usa un factor de concentrador de tensiones reducido $\Kfs{}$. Para obtenerlo primero se obtiene la sensibilidad a la muesca $q$:
$$q=\frac{1}{1+\frac{\sqrt[]{a}}{\sqrt[]{r}}}$$

\subsection{Combinaciones de carga}
Von Mises(Energía de deformación máxima) tomando en cuenta concentradores de tensiones $\Kf{}$ y $\Kfs{}$:
$$ \ansp \ansp \sigma_a^\prime =\sqrt[ ]{ \bigpar {\Kf{flex} \sigma_{a_{flex}} +\Kf{axial} \frac{\siga{axial}}{0,85}}^2+3\bigpar{ \Kfs{torsión}Z_{a_{flex}} }^2 }  $$
$$\ansp\ansp\sigma_m^\prime =\sqrt[]{ \bigpar{  \Kf{flex}\sigm{flex} +\Kf{axial}\sigm{axial} }^2+3\bigpar{\Kfs{torsión}Z_m}^2 } $$
\section{Engranes}
\subsection*{Intro a trenes de engranajes}
Se requiere un tren de engranajes (varios engranes en un mecanismo) para relaciones de transmision mayores a 6:1 por sus limites de tamaño e interferencia.
Según su geometría pueden ser de tres tipos:
\begin{itemize}
    \item Simples
    \item Compuestos
    \item Epiciclicos
\end{itemize}
\[ m_\omega = \pm \frac{productodientesimpulsores}{producto dientes impulsados}
\]
Para un {\bf tren simple} de mas de dos ruedas se tiene que todas las ruedas que no son la primera ni la ultima son dientes impulsados e impulsores a la misma vez. por ende la relacion de velocidad siempre es la razon entre el primer y ultimo engrane. Se las suele usar para llegar a un eje lejano o para invertir el sentido de giro.
\[ 
m_\omega = \left(-\frac{N_2}{N_3}\right)\left(-\frac{N_3}{N_4}\right) \ldots \left(-\frac{N_{n-1}}{N_n}\right) = \pm \frac{N_2}{N_n}
\]

{\bf Tren compuestos} son todos aquellos trenes que no son simples. Ahora si podemos aprovechar el full potencial de los engranes. Con estos logramos reducciones mayores a 6:1.

{\bf Tren compuesto revertido} Eje entrada y eje de salida colineales. Suerte resolviendo esto en el excel:
\[
M_1 (N_2+N_3)=M_2 (N_4 + N_5)
\]

{\bf Trenes epicíclicos.} Hay velocidad relativa entre los ejes. Se usa en motores híbridos donde es difícil aportar energía de dos fuentes a un eje en común. Osea que necesitas controlar ese grado de libertad adicional.

Tren epicíclico planetario. En general la salida se da con el engrane planetario.
{\sc Ojo el balanceo}
Cajas automatizadas.
\[
\omega_{engrane}=\omega_{brazo}+\omega_{brazo/brazo}
\]

Método de Willis
\[
R=m_\omega para \omega_{br} = 0
\]
Uso el CIR de forma analítica y tengo la expresión:
\[R=\frac{\omega_l - \omega_{br}}{\omega_f - \omega_{br}}
\]
{\bf Eficiencia de engranes especiales partiendo de la eficiencia de trenes simples.} De engranes interiores:

$L_{int}=\frac{b-1}{b+1}L_{ext}\qquad b=\frac{N_{int}}{N_{ext}}=\frac{N_{mayor}}{N_{Menor}}$
y engranes helicoidales:

$L_{hel}=L_{ext} 0,8 \cos \lambda $
y para trenes epicíclicos:
$E_0 = E_1\cdot E_2 \cdot \ldots \approx 1-L_1 -L_2 - \ldots $

con $T_1=\frac{T_{brazo}}{\rho E_0 -1}\qquad T_2 =\frac{\rho E_0 T_{brazo}}{\rho E_0 -1}$ donde $\rho$ es la relación básica de transimisión. 

\subsection*{Glosario de Engranes}
\glossentry{N}{Numero de dientes.}
\glossentry{m_G}{Razón de dientes. Siempre menor que 1.}
\glossentry{d_p}{Diámetro primitivo (Pitch diameter).}
\glossentry{m_n=\frac{d_p}{N}}{Modulo.}
\glossentry{P_n}{Paso diametral normal(Diametral pitch).}
\glossentry{P_d}{Paso diametral transversal.}

\glossentry{\pe_x}{Paso circular axial: Paso entre dientes medido sobre el circulo de paso.}
% \glossentry{\pe_b}{Paso circular transversal: Paso entre dientes medido perpendicular al helicoide.$\pe_b=\pe_x$ para dientes rectos }
\glossentry{\pe_N}{Paso de base normal.}
\glossentry{F,b}{Ancho de cara.}
\glossentry{\phi_n}{Angulo de presión normal. $20-25^\circ$}
\glossentry{\phi_t}{Angulo de presión transversal.}
\glossentry{\phi_r}{Angulo de presión transversal de operación.}
\glossentry{\psi_b,\psi}{Angulo de Hélice.}

% \glossentry{x}{}
%\glossentry{x}{}
%------------------------ENGRANES HELIC
\subsection{Engranes Helicoidales}
$$P_n=\frac{N}{d_p\cos\psi} \quad \pe_b=\pi m_n \cos \phi_n \qquad \pe_n=\pi m_n$$
$$ P_n=\frac{P_t}{\cos\psi}  \quad \pe_n=\pe_t\cos\phi \quad \phi_t=\tan^{-1} \bigpar{\frac{\tan\phi_n}{\cos \psi}}$$
La relación de contacto transversal $m_F$ nos dice cuantos dientes hay en contacto a la vez. Idealmente seria lo mas alto posible pero a relación de contacto cuanto mas alto mas le afectan las inexactitudes de montaje. Por eso se lo suele mantener por debajo a $1,2$.
$$ m_F=\frac{q}{\pe}=\frac{Z}{\pe_b}$$
donde $q$ es el arco de acción y $Z$ es longitud de la linea donde los dientes están en contacto.\par
\subsection{Fuerzas en los engranes helicoidales}
\begin{align*}
W_t &= \frac{2T}{d_p} \\
W_r &= W\sin \phi_n \\
W_t &= W \cos \phi_n \cos \psi \\
W_a &= W \cos \phi_n \sin \psi 
\end{align*}


\subsection{Dimensionamiento de Engranes}
Ecuación de flexión de Lewis (metrica e imperial):
$$\sigma = \frac{nK_v W_t}{Fm_nY}\qquad \quad \sigma = \frac{nK_v W_t P_n}{FY}$$
Ojo, el factor $K_v$ tiene $V$ medida en feet por minuto o metros por segundo y depende del terminado superficial del engrane. $V=\omega \frac{d_p}{2}$\par
Para obtener la durabilidad al pitting se usa:
$$\sigma_c=-nC_p\bigpar{\frac{K_vW_t}{F\cos \phi_n}^{\frac{1}{2}} (r_1^{-1}+r_2^{-1})} $$
$$C_p=\sqrt[]{ \frac{1}{\pi\bigg(\frac{1-\nu_p^2}{E_p}+\frac{1-\nu_c^2}{E_c} \bigg)} }$$
$$r_1=\frac{d_{p_p}\sin\phi_n}{2}\qquad r_2=\frac{d_{p_c}\sin\phi_n}{2} $$
\[ \sigma_c\big|_{N=10^8} =
  \begin{cases}
    2,76HB-70\, MPa    & \quad M\acute{e}trico \\
    0,4HB-10\, kpsi       & \quad Imperial
  \end{cases}
\]
\section{Ejes}
\subsection{Esfuerzos en ejes}
En general los esfuerzos axiales no son fuente de preocupación.
$$\sigma_a=\Kf{} \,\frac{M_ac}{I}\qquad \sigm{}=\Kf{}\, \frac{M_mc}{I}$$
$$\tau_a =\Kfs{}  \frac{T_a c}{J} \qquad \tau_m=\Kfs{} \frac{T_m c}{J} $$
Si se supone un eje solido con sección circular:
$$\sigma_a=\Kf{} \frac{32M_a}{\pi d^3}\qquad \sigm{}=\Kf{} \frac{32M_m}{\pi d^3}$$
$$\tau_a =\Kfs{}  \frac{16T_a}{\pi d^3}\qquad \tau_m=\Kfs{} \frac{16T_m}{\pi d^3}$$
Aplicando Von Mises (despreciando esfuerzos axiales):
$$\siga{}^\prime =\sqrt[]{\siga{}^2+3\tau_a^2} \qquad \sigm{}^\prime =\sqrt[]{\sigm{}^2+3\tau_m^2}$$
Goodman despejado para diámetro (caso particular):
$$d = \sqrt[3]{\frac{16n}{\pi} \bigpar{\frac{2\Kf{}M_a}{S_e}+\frac{\sqrt[]{3}\Kfs{} T_m}{S_e}}} $$
\section{Rodamientos}
$\pe=3$ para rodamientos de bola rígidos y es $\pe=10/3$ para cilíndricos. $L$ es los ciclos de trabajo esperados. $L_{nm}$ es una versión modificada por {\scshape\textrm{SKF} }.
$$L_{10}=\bigpar{\frac{C}{P}}^\pe\qquad L_{nm}=a_1a_{skf}L_{10}  $$
$$L_{horas}=\frac{10^6}{60n_{rpm} }L$$
Donde $a_1$ es el factor de confiabilidad y $a_{skf}$ es el factor del fabricante.
$$ P=P_{eq}=X P_{radial}+Y P_{axial} $$
$X$ e $Y$ son obtenidos de un catalogo y son característicos del modelo de rodamiento. $P$ es la fuerza aplicada al rodamiento\par
El factor $K=\frac{\nu_{min} }{\nu}$ es el cociente entre la viscosidad mínima permitida (obtenida de tabla con $D_{medio}$ y velocidad en rpm) y la viscosidad del lubricante elegido. $\eta_c$ esta gobernado por la condición de trabajo (aceite contaminado, limpio, etc.)

%--------------------PERNOS
\section{Pernos}
\subsection{Analogía del resorte}
Se tiene dos elementos abulonados juntos con un perno con precarga. La precarga impone una deformación sobre el perno $\delta$, los elementos abulonados tambien se van a deformar $\delta_1+\delta_2=\delta$ entre los dos. Este comportamiento se puede modelar como si los elementos estuvieran en serie (comparten deformación) y el perno en paralelo con estos.
\subsection{Pernos en tracción}
Cuando se quiere que una union permanezca junta bajo tracción y que sea desmontable se tiene que hacer un análisis de rigidez del conjunto bulon-elementos para conocer la precarga que hay que fijar.

La constante de rigidez (``resorte")del perno esta dada por la sumatoria en serie de su parte roscada y su parte lisa:
$$k_b=\bigg( \frac{1}{k_{liso}}+\frac{1}{k_{roscado}} \bigg)^{-1}=\frac{A_dA_tE}{A_d L_t +A_tL_d} $$

Los elementos a unir también se van a deformar según su constante de rigidez:
$$ \frac{1}{k_m}=\frac{1}{k_{m1}+\frac{1}{k_{m2}} } \ldots$$

pero como la deformación no ocurre sobre un volumen bien definido, se tiene que resolver por elementos finitos para obtener resultados que reflejen la realidad. 

De los modelos propuestos el que mejor aproxima la rigidez del material de una unión abulonada es considerando que la zona deformada tiene la forma de un tronco cónico, o un \emph{frustrum}. 

$$k=\frac{0,5774 \pi Ed}{\ln \frac{(1,155t+D-d)(D+d)}{(1,155t+D+d)
(D-d)}} $$
Donde $t$ es el ancho del elemento, $D$ es el diámetro del cono de deformación de partida, el diámetro del agujero para el perno es $d$ y $E$ es el modulo de Young. Una expresión que aproxima  

Se puede calcular la rigidez para dos elementos del mismo material y del mismo ancho con la aproximación (valores $A$ y $B$ de tabla 8--8)

$$k_m=A\cdot Ed e^{\frac{Bd}{l}} $$

Para conocer las tensiones en el perno se usan estas constantes para obtener $C$, denominada como la \emph{constante de rigidez de la unión abulonada} o la fracción de carga compartida por el perno.

$$ C=\frac{k_b}{k_b+k_m}$$

La fuerzas resultante sobre el perno y los elementos son (respectivamente)
\begin{align*}
F_b&=C\cdot P+F_i \\
 F_m&=(1-C)P-F_i \qquad F_m<0
\end{align*}

donde $P$ es la carga aplicada y $F_i$ es la precarga.



$$T=KF_id $$
\subsection{Pernos en corte}
Se tiene que verificar cizallamiento del perno, corte  de los elementos abulonados, aplastamiento de los elementos abulonados y desgarramiento de los elementos.
$$\tau_{\textrm{perno}}=\frac{F_{\textrm{corte}}}{A}$$
\subsubsection{Método Alemán}
\subsubsection{Método americano}
Verificar aplastamiento, corte perno y corte pieza (transversal y longitudinal)
\subsection{Fatiga}
Siempre va ser fluctuante, y si haces todo mal y l
%-------------------TORNILLOS DE POTENCIA
\section{Tornillos de Potencia.}
Largo de una vuelta de rosca: $\pi d_r$ 
$$ \sigma_x=\frac{6F}{\pi d_r n_t p}$$
$$\sigma_z=-\frac{4F}{\pi d_r^2} $$
$$\tau_{yz}= \frac{16T}{\pi d^3_r} $$

%----------------SOLDADURA
\section{Soldadura}
\subsection{Solicitaciones simples}
Unión a tope:
$$\tau=\frac{F}{hl} $$

Para una unión con filetes transversales el Shigley dice que \emph{no se tiene una aproximación analítica que prevea los esfuerzos existentes}. Por lo tanto se emplea un modelo simple y conservador el cual \textbf{supone} que la fuerza $F$ completa produce un esfuerzo cortante en la sección mínima de la garganta:
$$\tau=\frac{\sqrt[]{2}F}{hl} $$
\subsection{Torsión}
\textbf{Corte primario} en las soldaduras:$\tau^\prime = \frac{V}{A} $

Donde $A$ es el área de la garganta de todas las soldaduras.

\textbf{Corte secundario} en las soldaduras donde $I_p$ es el segundo momento de inercia polar de la unión respecto el centroide del grupo y $r$ es la distancia al punto de interés de la soldadura medida desde el centroide:

$$\tau^\dprime= \frac{Mr}{I_p} $$
$$I_p=\frac{hI_{pu}}{\sqrt[]{2}} $$
\subsection{Flexión}
Se va tener el mismo $\tau^\prime=\frac{V}{A}$
Y ademas el esfuerzo nominal cortante sobre la garganta va ser de 

$$\tau = \frac{Mc}{I} $$
$$I=\frac{hI_{u}}{\sqrt[]{2}} $$
Donde $c$ es la distancia al punto de interés medido desde el centroide. El segundo momento de inercia $I$ se calcula respecto el centroide en el eje donde hay momento flector.
\subsection{Resistencia de uniones soldadas}
Lo que mas importa al momento de soldar dos piezas es la habilidad y rapidez del soldador, las propiedades del material de aporte y de las piezas vienen después. 

En general se usan materiales que tienen buenas propiedades laminado en caliente. Esto se debe a que en la zona de la soldadura cualquier tratamiento térmico anterior, o si el material es laminado en frió, se reemplaza por las propiedades de una barra laminado en caliente.

Agregar concentradores de tensiones
%------------------RESORTES
\section{Resortes}
Longitud solida == carga bloque
Sea el diametro \emph{exterior} del resorte $D_e$ y el diámetro del alambre $d$. El diámetro medio esta dado por la cuenta $D=D_e-d$. $C=\frac{D}{d}$ usado para calcular el factor de corrección $K_B = \frac{4C+2}{4C-3} $, el cual no se debe comparar a un factor de concentrador de tensiones. $K_W=\frac{4C-1}{4C-4}+\frac{0,615}{C}$. Recordar $G=\frac{E}{2(1+\nu)}$.
$$\tau_{max} =k_{B,W}\, \frac{8F.D}{\pi d^3} $$

$$k=\frac{Gd^4}{8N_aD^3} $$
$$\delta =\frac{F}{k} $$
corte de cargas
% $$S_s=\frac{r^2S}{} $$
saarstahl materials specification sheet
SAE 9254 Resortes
IRAM 9260
A Leslabay le gusta usar $K_w$ en vez de $K_b$. Kb equivalente depende del area visto de arriba.


entender las tbas de la pag. 46->usar de filminas de clase. No te conviene usar R sola.
\subsection{Fatiga en resortes}
$S_{ut}=\frac{A}{d^m}$

%------------------CORREAS.
\section{Correas}
$T_1$ Ramal tenso, $`T_2$ es el Ramal flojo(devolucion).

Condicion Limite:
$$\frac{T_1}{T_2}=e^{\mu \alpha } $$
Es decir, si $T_2e^{\mu\alpha}\leq T_1$ entonces ocurre el resbalamiento.



Angulo de abrace $\Delta\alpha = \arcsin (R_2\cdot R_1 / C)$

Proy Horizontal: $\mu dQ=\cos \frac{D\alpha}{2}dT$
$dQ dfc$

Fuerza centrifuga por radian: $\bigg(   \frac{q V^2}{g}\cdot d\alpha \bigg)$

$$ \frac{T_1-F_c}{T_2-F_c}=e^{\mu \alpha}$$
Teorema de Prony Generalizado:
$$ T_1\leq T_2 e^{\mu \alpha}-F_c(e^{\mu \alpha}-1)  $$
Siempre vamos a ver poleas que estan en el punto optimo y no resbalan (cumple Prony): $P=(T_1-T_2)$

Tensión de montura $T_0$, se denomina la tension en la correa cuando esta detendida. Esta puede ser calculada de forma simplista como la semisuma de las tensiones:

$$T_0=\frac{T_1+T_2}{2} $$

\subsection{Correas Shigley}

\subsection{Mecanismos de desgaste}
Una vez que se ajusta un material 
\begin{itemize}
\item Corrosion (me genera muchisimas particulas)
\item Corrientes electricas (eddy)
\item rozamiento (velocidad)
\end{itemize}

Cuidado con usar radios chicos -> flexion. y radios grandes -> centrifugated

La potencia que puede transmitir una correa se ve dada por la sección de la correa, el angulo de contacto (lo mas alto posible, te lo limita la relación de transmisión)

Las poleas no se rompen, resbalan. Por eso cuidado cuando se diseña para picos de torque. El peor momento para la correa es cuando pasa sobre la polea mas chica.

bajar fuerza centrifuga: bajo peso de correa/m o bajo velocidad angular(bajar velocidad me sube la fuerza tangencial-> encontrar balance).

Las correas se compran en kit del mismo lote asi se distribuyen la carga equivalentamentemente.
\subsection{Selección de correas (Optibelt)}
Se busca(n) una(s) correa(s) que se les entregue $P_{motor}$ a un régimen de carga dado con $n_1$ revoluciones de entrada y $n_2$ revoluciones de salida deseadas. 

Primero se obtiene la potencia calculada $P_{calc}=P\cdot c_2$, esto depende del tipo de carga. Usando $max(n_1,n_2)$ y $P_{calc}$ se obtiene el tipo de correa (diag 1). Eligo la polea pequeña (pag. 104-111) luego la grande (numero redondo?) para que $i=\frac{n_1}{n_2}\sim \frac{d_1}{d_2}$. 

Sugiero un valor de $C\in \bar{S}$ (dist. entre centros) tal que $\bar{S}=\left \{  c\in \mathbb{R}^+ / 0,7(d_1+d_2)<c<2,0(d_1+d_2) \right \}$. Recomendado empezar con $C=1,8(d_1+d_2)$. Con $C$ calculo $L_p$ y luego busco $L_{p_{real}}$ parecido a $L_p$ igual al desarrollo interior. Con $L_{p_{real}}$ calculo $C_{real}$ verificando que $\in\bar{S}$.

Ahora busco los valores restantes para poder calcular la cantidad de correas necesarias $z$
$$z=\frac{P_{calc} }{P_{real}}=\frac{c_2\cdot P_{motor}}{P_{nominal}\cdot c_1 \cdot c_3} $$
donde $P_{nominal}=P_{tabla}+P_{mult}$ obtenidos de $pag 104-111$ con los datos de la polea pequeña ($P_{mult}$ es el ``incremento por multiplicación''... obtenido con $i_{real}$). $c_3\sim 1$.

$c_1$ es el factor de angulo. Lo busco en tabla (optibelt 17) con $\beta$. $D-d\equiv |d_1-d_2|$

$$\beta =2\arccos\left( \frac{D-d}{2C}\right) $$


\section{Cadenas}
Aumentar

%---------------LUBRICACIÓN
\section{Lubricación}
Diagrama de stribeck tiene una zona donde se minimiza el coeficiente de roce, despues de este punto al aumentar la velocidad aumentan las perdidas a la viscosidad del aceite (calor generado) se puede mitigar este problema con aditivos que modifican el indice de viscosidad:$\frac{\partial \mu}{\partial T}$

Seis condiciones basicas de la lubricación.
\begin{itemize}
\item Reducir rozamiento
\item reducir desgaste
\item Absorber o amortiguar impactos
\item Disminuir temperatura
\item Reducir la Corrosión 
\item Prevenir la contaminación externa
\end{itemize}
Lubricacion hidrodinamica se basa en la viscosidad del lube y la velocidad en el mov. rel. entre elementos y el huelgo radial entre los elementos.

Lubricacion mixta
lubricacion 

parafinicos
Naftenicos
Aromaticos

La lubricación

\section{Levas}
\glossentry{R_b}{Radio base.}
\glossentry{R_f}{Radio del seguidor.}
\glossentry{R_p}{Radio primario.}
\glossentry{s}{Desplazamiento del seguidor.}
\glossentry{\theta}{Ángulo de la leva.}
\glossentry{\phi}{Ángulo de contacto/presión.}
\glossentry{\rho_\rmit{paso}}{Radio de curvatura de la curva de paso del seguidor.}
\glossentry{\varepsilon}{Excentricidad de la leva respecto al seguidor.}
\glossentry{\omega}{Velocidad angular de la leva.}
\glossentry{\omega_n}{Frecuencia natural del conjunto leva-seguidor.}
\glossentry{\beta}{Ángulo total de un segmento \Rise{}, \Fall{} o \Dwell.}
\begin{itemize}
    \item[$s=$] $\left[\rmit{longitud}\right]$
    \item[$\dot{s}=$] $\left[\frac{\rmit{longitud}}{\rmit{tiempo}}\right]$
    \item[$\ddot{s}=$] $\left[\frac{\rmit{longitud}}{\rmit{tiempo}^2}\right]$ 
    \item[$v=$] $\left[\frac{\rmit{longitud}}{\rmit{radian}}\right]$
    \item[$a=$] $\left[\frac{\rmit{longitud}}{\rmit{radian}^2}\right]$ 
\end{itemize}

\[
\rho_{\rmit{paso}} = \frac{ \left[(R_p+s)^2+v^2 \right]^{3/2} }{(R_p+s)^2 +2v^2-a(R_p-s)}> \overbrace{R_F \cdot2,5}^{\textrm{\tiny Regla empírica}}
\]
$F$ es la fuerza de contacto
\[
F = m \cdot \ddot{s}+c\cdot\dot{s}+k(s+x_0)
\]

\[
\phi = \arctan\left( \frac{v-\varepsilon}{s+\sqrt{R_p^2-\varepsilon^2}}\right)
\]
$F_a$ es la fuerza axial sobre el seguidor, Nortóno la llama $F_c$ para confundirnos:
\[
F_a=F\cdot \cos \phi
\]
$F_t$ es la fuerza tangencial
\[
F_t = F_a \tan \phi = F \sin \phi 
\]

el ``\textit{dampin ratio}"{} $\zeta$:
\[
\zeta  = 0,06  = \frac{c}{2\sqrt{mk}}
\]
luego la frecuencia natural (ojo, $c$ depende de $k$)
\[
\omega_d = \sqrt{\frac{k}{m}-\left( \frac{c}{2m} \right)^2}
\]
Se tienen que cumplir las siguientes condiciones: $F$ siempre mayor a cero y la aceleración durante el \Fall{} no debe vencer el resorte (si no se ``despega"{} la leva). Finalmente tiene que trabajar por debajo de la frecuencia natural para evitar la separación del conjunto leva-seguidor por vibraciones.

\begin{align}
   \fbox{ $F>0$}& \\
   \fbox{ $k\cdot \left(s(t_{\ddot{s}_{\min}})+x_0\right)>m\cdot\left| \ddot{s}_{\min} \right|$}& \quad \text{donde}\quad \ddot{s}_{\min}<0 \\
   \fbox{$\frac{\omega_f}{\omega_d}<\frac{\sqrt{2}}{2}=\frac{1}{\sqrt{2}}$}&
\end{align}
$s(t_{\ddot{s}_{\min}})$ es el desplazamiento evaluado en el lugar de aceleración mínima.

\[
T = \frac{F \cdot \dot{s}}{\omega\cdot \eta}
\]

%% %%%%%%
%% Seguidor PLANO
\subsection{Seguidor plano}
\[
\rho_{base} = R_b +s +a >0
\]
donde
\[
R_{b_{\min}} > \min\left[ \rho_{\min} - (s+a)  \right]
\]
\end{document}
